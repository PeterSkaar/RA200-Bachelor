%Ting som må i glossory. 
% - Mission Zero?
% - Forrigling, blokkdiagram, PI&D
% Koble inn kommandoen som brukes av \printglossaries for å sette sidestil
%Et verktøy som oversetter høynivåkode til maskinkode eller bytekode for kjøring på PLS-er og andre styringssystemer.
\thispagestyle{fancy}

\newglossaryentry{Globalvariabel}{
    name={global variabel},
    description={Ein variabel som er tilgjengeleg i heile programmet},
    plural={global variabelen}
}
\newglossaryentry{Kompilator}{
    name={Kompilator},
    description={Eit verktøy som omset høgnivåkode til maskinkode},
    plural={Kompilatoren}
}
\newglossaryentry{NEK}{
    name={NEK},
    description={Norsk Elektrotekniske Komité},
    plural={NEK}
}

\newglossaryentry{SFC}{
    name={SFC},
    description={Sequential Function Chart},
    plural={SFC}
}

\newglossaryentry{LD}{
    name={LD},
    description={Ladder Diagram},
    plural={LD}
}

\newglossaryentry{ST}{
    name={ST},
    description={Strukturert tekst},
    plural={ST}
}

\newglossaryentry{CFC}{
    name={CFC},
    description={Continuous Function Chart},
    plural={CFC}
}

\newglossaryentry{TIA}{
    name={TIA},
    description={Totally Integrated Automation},
    plural={TIA}
}

\newglossaryentry{Hygienisering}{
    name={hygienisering},
    description={Behandling som har som hovudmål å redusere faren for overføring av smittestoff til menneske, dyr og plantar ved disponering eller anna handtering av slam},
    plural={hygienisering}
}

\newglossaryentry{diffuser}{
    name={diffuser},
    description={Redskap for å spre strøymen av eit stoff},
    plural={diffusera}
}

\newglossaryentry{RA}{
    name={RA},
    description={Reinseanlegg},
    plural={RA}
}

\newglossaryentry{Batch}{
    name={batch},
    description={Ein gitt mengde av noko},
    plural={batchar}
}

\newglossaryentry{forrigling}{
    name={forrigling},
    description={forrigling er en mekanisk eller elektrisk konstruksjon som forhindrer at farlige og/eller uønskede hendelser kan forekomme},
    plural={forriglingar}
}
\newglossaryentry{SCD-Toolbox}{
    name={SCD-Toolbox},
    description={},
    plural={Visios}
}
\newglossaryentry{MicrosoftVisio}{
    name={Microsoft Visio},
    description={Ein programmvare utvikla av Microsoft Corporation, som brukas til å lage diagram og illustrasjonar},
    plural={Visios}
}
\newglossaryentry{HMI}{
    name={HMI},
    description={'Human Machine Interface' er eit grensesnitt mellom menneske og maskin.},
    plural={alarmar}
}
\newglossaryentry{alarm}{
    name={Alarm},
    description={Varsling som er vist på HMI og som krever operatørens merksemd, forårsaka av en diskret endring av tilstanden.},
    plural={alarmar}
}
\newglossaryentry{MJK}{
    name={MJK},
    description={Leverandør av prøvetakar stasjon.},
    plural={LaTeXes}
}
\newglossaryentry{MissionZero}{
    name={Mission Zero},
    description={Renasys sitt slagord.},
    plural={LaTeXes}
}
\newglossaryentry{PID}{
    name={P\&ID},
    description={'Piping and instrumet diagram'},
    plural={LaTeXes}
}
\newglossaryentry{blokkdiagram}{
    name={Blokkdiagram},
    description={Ein visuell representasjon av ein funksjonsblokk som viser strukturen og funksjonen til funksjonsblokka.},
    plural={Blokkdiagrammer}
}

\newglossaryentry{latex}{
    name={LaTeX},
    description={LaTeX er eit høgnivå-typsettingssystem brukt for å lage vitenskaplege og tekniske dokument},
    plural={LaTeXes}
}
\newglossaryentry{github}{
    name={GitHub},
    description={Dette er ei teneste for syknronisering av filer},
    plural={GitHubs}
}
\newglossaryentry{IO}{
    name={I/O},
    description={Eit begrep som er brukt for ein felles betegnelse for inngangar og utgangar i ein pls},
    plural={I/O}
}
\newglossaryentry{Renasys}{
    name={Renasys},
    description={Oppdragsgivar},
    plural={Renasys}
}
\newglossaryentry{Sunnfjord Kommune}{
    name={Sunnfjord kommune},
    description={Oppdragsgivar},
    plural={Sunnfjord kommune}
}
\newglossaryentry{PLS}{
    name={PLS},
    description={PLS står for programmerbar logisk styring og er namnet på ein type eining ein kan bruke til å styre elektriske og automatiske system},
    plural={PLS}
}

\newglossaryentry{PAS}{
    name={PAS},
    description={'Public available specification,' og har som mål av IEC å framskynde standardiseringa innan områder som har rask utviklande teknologi.},
    plural={PLS}
}

\newglossaryentry{SCD}{
    name={SCD},
    description={'System controll diagram', og er ein metode for å spesifisere og dokumnetere kontrollapplikasjoner for styresystemer.},
    plural={SCD}
}

\newglossaryentry{IEC}{
    name={IEC},
    description={International Electrotechnical Commission},
    plural={IEC}
}
\newglossaryentry{Codesys}{
    name={Codesys},
    description={Ein programvareplattform for industriell automasjonsteknologi.},
    plural={Codesys}
}
\newglossaryentry{MA}{
    name={MA},
    description={Monitor Analogue},
    plural={Monitor Analogue}
}
\newglossaryentry{MB}{
    name={MB},
    description={Monitor Binary},
    plural={Monitor Binary}
}
\newglossaryentry{SBE}{
    name={SBE},
    description={Switch Binary Eletrical},
    plural={Switch Binary Eletrical}
}
\newglossaryentry{SBV}{
    name={SBV},
    description={Switch Binary Value},
    plural={Switch Binary Value}
}
\newglossaryentry{FB}{
    name={FB},
    description={funksjonsblokk},
    plural={funksjonsblokker}
}
\newglossaryentry{FC}{
    name={FC},
    description={Funksjon},
    plural={FC}
}
\newglossaryentry{SBR}{
    name={SBR},
    description={Sekvensiell batchreaktor},
    plural={SBR}
}
\newglossaryentry{HVL}{
    name={HVL},
    description={Høgskulen på Vestlandet},
    plural={HVL}
}