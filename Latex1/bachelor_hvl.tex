\documentclass[11pt, a4paper]{report}
%-----------NORSK LaTeX--------------

% Denne pakken inkluderer skandinaviske bokstaver( æ, ø og å)
\usepackage[utf8]{inputenc}

% Denne pakken bruker norsk typesetting (innholdsliste osv.)
\usepackage[norsk]{babel}

%--------DOKUMENTOPPSETT------------

% Lar deg tilpasse layouten i prosjektet
\usepackage{fancyhdr}

% Definere egne farger 
\usepackage[usenames, dvipsnames]{color}

% Avstand på marger til sidene på arket
%\usepackage{geometry}

% Linker i dokumentet (figurer, seksjoner osv)
\usepackage[hidelinks]{hyperref}

% Kildeliste-setup
%\usepackage[natbibapa]{apacite} % For apalike eller apacite
\usepackage{natbib} % For Chicago, Vancover, Harvard Style, Nature Style osv

% Spacing mellom bokstaver
\usepackage{microtype}

% Laster inn forskjellige skrifttyper
\usepackage{uarial} % Arial
\usepackage[scaled]{helvet} % Helvetica
%\usepackage{tgbonum} % Usikker
%\usepackage{mathptmx} % Times New Roman
%\usepackage{lmodern} % Latin Modern
%\usepackage[charter]{mathdesign} % Charters
\usepackage[T1]{fontenc}

% Avsnitt mellom paragrafer
\usepackage{parskip}

% Linjeavstand i dokumentet
\usepackage{setspace}
\setlength{\parskip}{1em}

% Fet skrift på alle figurnr
\usepackage[labelfont=bf]{caption}
\usepackage{capt-of}
\usepackage{caption}
\usepackage{tikz}

\usepackage{chngcntr}
\counterwithin{figure}{chapter}
\counterwithin{table}{chapter}

% Definerer mellomrom mellom seksjons-, figur- og tabellnummer i innholdsliste til teksten begynner
\renewcommand{\numberline}[1]{#1 \,}

% Kan skrive i kolonner
\usepackage{multicol}

% For appendix
\usepackage{titletoc}
\usepackage{appendix}

%----------ANDRE PAKKER--------------

% Pakker fra AMS. Kan skrive matteformler
\usepackage{amsfonts, amssymb, amsthm}
\usepackage{mathtools}

% For grafikk (lime inn bilder, pdf-dokumenter)
\usepackage{graphicx}
\usepackage{float}
\usepackage{pdfpages}
\usepackage{subcaption}

% For inportering av csv-filer som tabeller. ( Ikkje i bruk)
%usepackage{csvsimple}

% Pakke for citations
%\usepackage{biblatex}
% Pakke for at url linker skal fungere i bl.a. referansar
\usepackage{url}

% Generere tekst
\usepackage{kantlipsum}

% Definere egne objekt
\usepackage{tikz}

% Lar deg lage lister på en mer oversiktlig måte
\usepackage[sharp]{easylist}

% Boks rundt tekst
\usepackage{fancyvrb}

% Test pakker
\usepackage{array}
\usepackage[margin=1in]{geometry} % Juster margene etter behov
\usepackage{titlesec}
\usepackage{tabularx}
\usepackage[table]{xcolor}
\usepackage{lastpage}

% pakke for ordliste
\usepackage[automake]{glossaries-extra}

% For å kunne justere bilete
\usepackage{adjustbox}

% 
\usepackage{tocloft}


% Sider til SCD
\usepackage{afterpage}
\usepackage{adjustbox}
%---------------------------------------------------

% Legger til bookmarks i pdf
\hypersetup{
    bookmarks=true, % Aktiver bokmerker
    bookmarksnumbered=true, % Inkluder nummer i bokmerker
    pdfstartview={FitH}, % Åpner PDF med passende bredde
    %colorlinks=true, % Farger lenker i stedet for å bruke bokser
    %linkcolor=blue, % Intern lenkefarge
    %filecolor=magenta, % Fil-lenkefarge
    %urlcolor=cyan % URL-lenkefarge
}



% Velger arial som skrifttype i dokumentet 
%\renewcommand{\rmdefault}{qpl} % 
%\renewcommand{\rmdefault}{phv} % Arial
\renewcommand*\familydefault{\sfdefault} % Helvetica(sans-serif)


% Definerer egne farger
\definecolor{mybla}{RGB}{25, 49, 90}
\definecolor{mylysbla}{RGB}{0, 136, 176}
\definecolor{renasysGreen}{RGB}{4, 255, 4}
\definecolor{purewhite}{RGB}{255,255,255}
\definecolor{lightgray}{gray}{0.8}
\definecolor{myblack}{gray}{0.1}

% Velger norsk språk på innholdsfortegnelse, figurliste, tabelliste og Referanseliste
% Legger disse så til i innholdsfortegnelsen.
\addto\captionsnorsk{\renewcommand{\contentsname}{Innhaldsliste}}
\addto\captionsnorsk{\renewcommand{\listfigurename}{Figurliste}}
\addto\captionsnorsk{\renewcommand{\listtablename}{Tabelliste}}
\addto\captionsnorsk{\renewcommand{\bibname}{Referansar}}
% Nynorsk!
\addto\captionsnynorsk{\renewcommand{\contentsname}{Innhaldsliste}} 
\addto\captionsnynorsk{\renewcommand{\listfigurename}{Figurliste}}
\addto\captionsnynorsk{\renewcommand{\listtablename}{Tabelliste}}
\addto\captionsnynorsk{\renewcommand{\bibname}{Referansar}}


% ----- Justering av avstander mellom chapter / section / subsection / subsubsection

% Justeringa av chapter, 
% første verdi : Horisontalverdi frå venstre marg
% Andre verdi : avstanden frå toppen av sida til tittelen
% Tredje verdi : Avstand til neste tekst
\titleformat{\chapter}[hang]
  {\normalfont\bfseries\LARGE}{\thechapter}{1em}{}
\titlespacing*{\chapter}{0pt}{-50pt}{0pt}

% Juster avstanden for \section
\titleformat{\section}
  {\normalfont\Large\bfseries}{\thesection}{1em}{}
\titlespacing*{\section}{0pt}{10pt}{10pt}

% Juster avstanden for \subsection
\titleformat{\subsection}
  {\normalfont\large\bfseries}{\thesubsection}{1em}{}
\titlespacing*{\subsection}{0pt}{10pt}{10pt}

% Juster avstanden for \subsubsection om nødvendig
\titleformat{\subsubsection}
  {\normalfont\normalsize\bfseries}{\thesubsubsection}{1em}{}
\titlespacing*{\subsubsection}{0pt}{10pt}{10pt}

% ------------------- jusetring av lister ------------------

% Denne vil endre skriftstørrelse ...
% Legg til \hfill før LARGE om overskrifta skal sentrerest
\renewcommand{\cfttoctitlefont}{\LARGE\bfseries}
\renewcommand{\cftaftertoctitle}{}
\renewcommand{\cftloftitlefont}{\LARGE\bfseries}
\renewcommand{\cftafterloftitle}{}
\renewcommand{\cftlottitlefont}{\LARGE\bfseries}
\renewcommand{\cftafterlottitle}{}

% Avstand mellom toppen av sida og ned til overskrifta t.d. innhaldsliste
\setlength{\cftbeforetoctitleskip}{-20pt}  % Adjust the value as needed, negative reduces the space
\setlength{\cftbeforeloftitleskip}{-20pt}  % Adjust the value as needed
\setlength{\cftbeforelottitleskip}{-20pt}  % Adjust the value as needed

% Avstand mellom Overskrift og kapittel
\setlength{\cftaftertoctitleskip}{15pt}
\setlength{\cftafterloftitleskip}{15pt}
\setlength{\cftafterloftitleskip}{15pt}


% -------------------------------------------------------------

% Her definerer ein pagestylen til hovud opplegget
\fancypagestyle{fancy}{
  % Clear the header and footer
  \fancyhf{}
  % Set the right side of the footer to be the location of the page number
  \fancyfoot[R]{Side \thepage\ av~\pageref{LastPage}} % Her sett ein fra side til side
  \renewcommand{\headrulewidth}{0pt} % Remove header line
  \renewcommand{\footrulewidth}{0pt} % Remove footer line
}

% Her definerer ein pagestylen til dei sidene som skal ha romartall
\fancypagestyle{romanpages}{
  % Clear the header and footer
  \fancyhf{}
  % Set the right side of the footer to be the location of the page number
  \fancyfoot[R]{\thepage}
  \renewcommand{\headrulewidth}{0pt} % Remove header line
  \renewcommand{\footrulewidth}{0pt} % Remove footer line
}


% ------ Justering av avstander i dokumentet for itemize ---------------------

\let\olditemize\itemize
\renewcommand{\itemize}{
  \olditemize
  \setlength{\itemsep}{-10pt} 
}

% ------ Justering av avstander i dokumentet for enumerable ---------------------

% Save the original definition of enumerate
\let\oldenumerate\enumerate

% Redefine the enumerate to adjust spacing
\renewcommand{\enumerate}{
  \oldenumerate
  \setlength{\itemsep}{-7pt}  % Adjust this value as needed
}

% -------------------------------------------------------------------



% ------------------------- KODE FOR ORDLISTE ---------------------

% Kode for å lage ei meir kompakt ordliste
\newglossarystyle{compact}{%
  % base this style on the list style:
  \setglossarystyle{list}%
  % adjust the space between entries:
  \renewenvironment{theglossary}{%
    \begin{description}%
    \setlength{\itemsep}{-10pt}% adjust this value as needed
    \setlength{\parskip}{0pt}%
  }%
  {%
    \end{description}%
  }%
}

\setglossarystyle{compact}

% Kode for å lage stil for ordlista
\newglossarystyle{tablestyle}{
  % Start the glossary as a tabular environment
  \renewenvironment{theglossary}{
	\setlength{\LTleft}{0pt} % Align the table with the left margin
    \begin{longtable}{lp{\glsdescwidth}}
    }{
    \end{longtable}
  }
  \renewcommand*{\glossaryheader}{
    \bfseries Term & \bfseries Description \\
    \endfirsthead
    \bfseries Term & \bfseries Description \\
    \endhead
  }
  % No heading between groups
  \renewcommand*{\glsgroupheading}[1]{}
  % No space between groups
  \renewcommand*{\glsgroupskip}{}
  % Define how each entry should appear:
  \renewcommand*{\glossentry}[2]{
    \glstarget{##1}{\glossentryname{##1}} &
    \glossentrydesc{##1} \\
  }
}

% Setter Stil for Ordlista
\setglossarystyle{tablestyle}


\makeglossaries
\loadglsentries{Ordliste.tex}

% Stil for referansar
\bibliographystyle{unsrt}

% Endre namn på overskrifta til referanselista

% dette må stå her slik at romartalla viser korrekt i rapporten
\makeatletter
\let\ps@oldplain\ps@plain
\renewcommand\ps@plain{\pagestyle{romanpages}}
\makeatother


% ---------------------------------- END kode for ordliste -----------------------------

% ------------------------------------ Dokumentet starter her ----------------------------

\begin{document} % Dokument start



	%------------------FREMSIDE------------------------

	% Linker inn fremsiden fra adms (administrative sider) fremside-filen.
	% Førstesiden skal ha følgende marger.
\newgeometry{top=2.5cm,bottom=2.5cm,right=3cm,left=5cm}

% Ønsker ikke sidetall på denne siden.
\thispagestyle{empty}


% Legger inn banner
\begin{center}
	\begin{tikzpicture}[remember picture, overlay]
	\node [xshift=0cm, yshift=11.5cm] at (current page.center) 
	{\includegraphics[width=\paperwidth]{Bilder/frontpage.png}};
	\end{tikzpicture}
\end{center}



\vspace{6cm}

{\fontsize{36}{50}\selectfont \textbf{\textcolor{mybla}{Programmering PLC}}}


\vspace{1cm}
% Forfattarar
{\fontsize{26}{30}\selectfont \textbf{\textcolor{mylysbla}{Roar Bøyum}}}\\
{\fontsize{26}{30}\selectfont \textbf{\textcolor{mylysbla}{Vegard Aven Ullebø}}}\\
{\fontsize{26}{30}\selectfont \textbf{\textcolor{mylysbla}{Peter Søreide Skaar}}}\\ 


\vspace{1.3cm}


\setstretch{0.7}

{\fontsize{20}{20}\selectfont\textcolor{mylysbla}{Name of Masterprogram}} 

\vspace{-0.4cm}
{\fontsize{20}{20}\selectfont\textcolor{mylysbla}{Department/Institute/Program}} 

\vspace{-0.4cm}
{\fontsize{20}{20}\selectfont\textcolor{mylysbla}{Supervisor (in agreement with supervisor)}} 

\vspace{-0.4cm}
{\fontsize{20}{20}\selectfont\textcolor{mylysbla}{Submission Date}} 



% Endrer skriften for resten av dokumentet
{\fontfamily{qpl}\selectfont}


\vspace{1cm}

\setstretch{0.8}
{\fontsize{9.5}{20}\selectfont
	I confirm that the work is self-prepared and that references/source references to all sources used in the work are provided, cf.\ \href{https://lovdata.no/dokument/SF/forskrift/2016-12-21-1851#KAPITTEL_10}{Regulation relating to academic studies and examinations at the Western Norway University of Applied Sciences (HVL), §~10}.}
	

	%------------------------------------------------------

	\newpage

	% Nye marger
	\newgeometry{top=2.5cm,bottom=3cm,right=2.2cm,left=2.2cm}

	% Linjeavstand
	\setstretch{1.3}

	% Set the right side of the footer to be location of the page number

	% Ønsker tallnummerering videre.
	%-------------------- 			FORORD / Samandrag			---------------------	
	\pagenumbering{Roman}
	\setcounter{page}{2}
	\chapter*{}
\thispagestyle{romanpages}

% Sentrere bildet helt nøyaktig på siden
\noindent\begin{adjustbox}{center=\paperwidth}
	\hspace*{-0.25\textwidth} % justering venstre høgre på bilde
    \includegraphics[width=1.1\textwidth]{Bilder/Framside RA200.jpg}
\end{adjustbox}

	\chapter*{Forord}
\thispagestyle{fancy}
Ønskjer å takke, Hansa Mango IPA for ei fattig trøst til ein fattig student.


% Ønsker ikke sidetall på denne siden.
\thispagestyle{empty}
	\chapter{Samandrag}
\thispagestyle{romanpages}

Bacheloroppgåva, som vi har løyst saman med \gls{Renasys} og \gls{Sunnfjord Kommune}, handlar om utbetring av styresystemet på 
avlaupsreinseanlegget i Sande i Sunnfjord. 
Anlegget er teknisk utdatert noko som gjer at ei oppgradering av styresystemet må undersøkast.

Rapporten presenterer fire ulike løysningsalternativ
der ulik grad av dybde blir vurdert. \newline
Rapporten legg til grunn val av løysningsalternativ C som presenterer planlegging av eit nytt styresystem.
Vidare undersøkar rapporten generell verkemåte til eit avlaupsreinseanlegg og forklarer og dokumenterer reinseanlegget på Sande.

Oppgåva gir innblikk i dei relevante stega i planlegging av eit nytt styresystem der eigen dokumentasjon er nytta som grunnlag for vidare arbeid. 
Utføring av programmering, simulering og testing er løyst med eigne funksjonsblokker og metodar i programmeringsverktøyet \gls{Codesys},
der anerkjende \gls{IEC} standardar er undersøkt og tatt i bruk.
Vidare blir styresystemet og programmeringsarbeidet skildra og dokumentert.

Resultatet av bacheloroppgåva gir vår arbeidsgivar ny og forbetra dokumentasjon av anlegget 
og grunnlaget for eit nytt, fleksibelt og teknisk moderne styresystem. 
Rapporten undersøker også generelle oppgraderingar og ytterlegare forbetringspotensial for reinseanlegget.

	%--------------------Innhaldsliste / Figurliste / tabell liste---------------------	
	% innholdsliste
	\cleardoublepage
	\pagestyle{romanpages}
	\tableofcontents
	\thispagestyle{romanpages}  % Apply to the first page of the Table of Contents

	\cleardoublepage
	\pagestyle{romanpages}
	\listoffigures
	\thispagestyle{romanpages}  % Apply to the first page of the List of Figures
	
	\cleardoublepage
	\pagestyle{romanpages}
	\listoftables
	\thispagestyle{romanpages}  % Apply to the first page of the List of Tables

	% Turn on the style
	\pagestyle{fancy}


	\clearpage
	\pagenumbering{arabic}
	%\setcounter{page}{5}

	%------------------------------------------------------

	%------------------ Lim inn sider til documentet her ------------------------



	% Kapittel 3 Innleiing
	\chapter{Innleiing}
\thispagestyle{fancy}

\section{Om oss}
Vi er tre studentar som studerar Automatiseringsteknikk med robotikk ved Høgskulen på Vestlandet (\gls{HVL}), campus Førde.
Vi har alle fagbrev som elektrikar og fann lett tonen i starten av studiet. \newline
Gjennom tre år har vi brukt vår breie kompetanse innen industri, programmering og elektronikk
til å danne ei god gruppe.

\textbf{Roar Bøyum} bur i Sogndal og har arbeidd med teknologi og prosjektstyring i maritime sektor.

\textbf{Vegard Aven Ullebø} bur i Vadheim og har variert erfaring frå industri og offshore-arbeid. 

\textbf{Peter Søreide Skaar} er busatt i Førde og balanserar studiet med ei deltidsstilling i Renasys. 

Peter sin tilknyting til oppdragsgivar har gitt oss viktig innsikt i bransjen og 
Vegard og Roar sine tekniske kunnskapar og praktiske tilnærming har vore nyttige for prosjektet og hjelpt oss undervegs.


\newpage



	\section{Oppdragsgivar}
\textbf{Renasys AS i samarbeid med Sunnfjord kommune}

Renays\citep{Renasys} kan skildrast som ein innovativ og nyskapande oppstarts bedrift som arbeider med banebrytande teknologi innan mekanisk finpartikkelfiltrering av avlaupsvatn.
Bedrifta har 15 tilsette fordelt på forskjellige lokasjonar, dei har kontor på Øyrane i Førde og Sandnes i Rogaland. 
Etter å ha arbeidd konfidensielt over lengre tid, gjekk Renasys offentleg ut med teknologien sin i løpet av 2023. 
Dei tilbyr reinsetjenester til kommunar og interkommunale selskap innan avlaup og maritim sektor.

Samarbeidet med Sunnfjord kommune\citep{SunnfjordKommune} er retta mot ``Mission Zero'' som er eit ambisiøst mål om 
null utslepp, null avfall, null energi og ein generell forbetring av avlaupssektoren i Noreg. \newline
Sunnfjord kommune er ansvarleg for vann, veg og avlaup i sitt område og har engasjert 
Renasys for å utforske forbetringar ved reinseanlegget på Sande.
\newline
\newline


\begin{figure}[htbp]
    \centering
    \begin{subfigure}[b]{0.3\textwidth}
        \centering
        \includegraphics[width=1\textwidth]{Bilder/renasys.png}
    \end{subfigure}
    \hfill
    \begin{subfigure}[b]{0.3\textwidth}
        \centering
        \includegraphics[width=0.7\textwidth]{Bilder/SK.png}
    \end{subfigure}
    \caption{Logo oppdragsgivar}\label{fig:Oppdragsgivar}
\end{figure}

\section{Om Vedlegga}
Til denne rapporten er det utarbeidd eit eige dokument som inneheld alle relevante vedlegg. 
Desse vedlegga omfattar ytterlegare tekniske detaljar knytte til oppgåva vår.\newline
Vedlegga er organiserte for å gje lesaren ein djupare innsikt i arbeidet vi har utført.

	% Kapittel 4 Analyse av problemet
	\chapter{Analyse av oppgåva}
\thispagestyle{fancy}
Sande reinseanlegg ligger 25 minuttar med bil frå Førde sentrum og er det 
nest største reinseanlegg i Sunnfjord kommune. Anlegget er dimensjonert for 1500 personekvivalentar
og vart bygd i 2003. Anlegget har fleire utfordringar, og ein av desse utgjer ei problemstilling som egnar seg godt til ei
bacheloroppgåve innan automasjon.
	\section{Problemstilling}
Reinseanlegget er teknisk utdatert og er avhengig av modernisering. Stryringssystemet er over tjue år gammalt
og består hovudsakeleg av eldre og utgåtte komponentar. Med eldre komponentar aukar risikoen for svikt, 
og reservedelar som passar kan være vanskeleg å finne.

WaterCare, som leverte styresystemet er i seinare tid også blitt avvikla noko som gjer at kompetansen 
og moglegheiten for å gjere endringar i styresystemet er vankeleg. 
Mykje grunna denne utfordrina har ikkje anlegget følgt den teknologiske utviklinga 
og små problem har samla seg opp til å bli større utfordringar.

Samtidig med desse faktorane er dokumentasjonen til reiseanlegget mangelfull, noko som gjer at enkle arbeidsoppgåver blir vansklege og tunge.
I eit værste tilfelle vil styringseinheten på anlegget svikte og med det som er nevnt tidlegare vil det være krevjande
å få anlegget i drift igjen. Dette er eit kritisk problem innan avlaupshandtering og kan ikkje sjåast vekk ifrå.
\newline

%\begin{figure}[htbp]
%    \centering
%    \includegraphics[width=0.8\textwidth]{Bilder/BeijerSkjerm.JPG}
%    \caption{Beijer HMI}\label{fig:HMI}
%\end{figure}


\begin{figure}[htbp]
    \centering
    \begin{subfigure}[b]{0.5\textwidth}
        \centering
        \includegraphics[width=1\textwidth]{Bilder/BeijerSkjerm.JPG}
        \caption{Beijer HMI}\label{fig:subfig1}
    \end{subfigure}
    \hfill
    \begin{subfigure}[b]{0.3\textwidth}
        \centering
        \includegraphics[angle=-90,width=1\textwidth]{Bilder/Styreskap.JPG}
        \caption{Styreskap}\label{fig:subfig2}
    \end{subfigure}
    \caption{Styresystem}\label{fig:Styresystem}
\end{figure}
	\newpage
\section{Løysningsforslag}

Fells faktor for løysningsalternativ til reinseanlegget vil vere å modernisere
styresystemet med ein ny moderne styringseining (PLS). 
Dette vil være med å løyse mykje av utfordringane som anlegget har idag. 
Det er fleire måtar og utføre dette på men vi velger å ta med dei som er mest relevant.
\newline
\subsection{Løysningsalternativ A: Bytte PLS}
Den enklaste løysninga er å oppgradere det eksisterande styresystem ein til ein. 
Dette inneber å bytte ut den eksisterande PLS-en med ein nyare PLS frå same leverandør, som inneheld lik programvare.
Dette vil minske faren for kritisk komponentsvikt. Sidan PLS-programmet er det same, 
vil det likevel ikkje bli enklare å gjere endringar i det.

\subsection{Løysningsalternativ B: Bytte PLS, konvertere kode}
Eit steg vidare frå løysningsalternativ A, er også å konvertere det eksisterande programmet over til eit meir
framtidsretta programmeringsspråk der kompetansen er lettare tilgjengeleg. Dette alternativet inneber å konvertere logikk for logikk,
med dei nødvendige tilpassingane som trengst for den nye plattformas unike krav og funksjonar.
\newline \newline
Utfordringa med dette alternativet er at ein risikera å dra med seg eventuelle feil frå eksisterande
kode over til det nye programmeringsspråket. Det kan også være problematisk å gjere endringar
i den konverterte koden ettersom den overordna strukturen i programmet
kanskje ikkje blir fullt ut bevart eller riktig tolka i det nye programmeringsspråket.

\subsection{Løysningsalternativ C: Bytte PLS, ny kode}
Eit tredje og meir komplett løysning bygger på dei to førre alternativa, men forheld seg rundt problematikken med endring av program.
Dette alternativet forhold seg ikkje til val av PLS eller bruk av den gamle koden.
\newline \newline
Ved å sette seg inn i anleggets teknologisk verkemåte vil vi kunne få eit bilete av korleis anlegget skal fungere teoretisk.
På denne måten vil ein kunne opprette ein ny funksjonsbeskriving til anlegget og bygge eit nytt program utifrå denne funksjonsbeskriving.
Programmerings messig vil være som å starte på nytt, men dette vil gje sikkerheit for at anlegget driftast optimalt og at den nye programmet følger
dagens standard. Dette tilsvara ein arbeidsprosess av eit nytt anlegg.

\newpage
\subsection{Løysningsalternativ D: Nytt anlegg, ny teknologi}
I tillegg til ein full gjennomgang av styresystemet, 
kan ein undersøke nye løysningar for å optimalisere, forbetre og utvide heile reinseanlegget.
I samråd med Renasys arbeider Sunnfjord kommune med å undersøke moglegheita og kostnadane for eit slikt alternativ,
men første steg i ein slik prosess er uansett eit løysningsalternativ som forholder seg til problema rundt styresystemet.

	\section{Val og drøfting}

Då vi starta med å definere dei ulike løysningsalternativa i forprosjektet, var vi
ikkje heilt sikre på kva retning vi skulle velje. Alle løysningsalternativa synast å vere gode,
men med varierande arbeidsmengde. Vi kom fram til at løysningsalternativ C hadde ei arbeidsmengd
som best passa ei bacheloroppgåve.

Dette var ein av grunnane til at vi valte løysningsalternativ C.
I tillegg til dette er det fleire aspekt som gjer denne løysinga til den beste for anlegget.

Ved å følgje eit løysningsalternativ som bygger på ein arbeidsprosess for eit nytt anlegg 
kan vi forhindre å overføre feil og manglar frå eksisterande program. 
Dette gir oss også moglegheita til å utforske og finne betre løysningar, 
samtidig som vi oppgraderer styresystemet med ein forbetra oppbygging og heilskap.
Vidare legg dette til rette for ein enklare og meir fullstendig dokumentasjonsjobb, 
som er essensiell for å oppretthalde kvaliteten og berekrafta til prosjektet over tid.

Med tanke på arbeidsmengd, valde vi å halde oppgåva teoretisk.
Det var uvisst om reinseanlegget følgde gjeldande sikkerheitskrav og normer,
og om starten av eit praktisk inngrep ville utløyse krav om forbetringar.
Dette var noko vi ikkje ønskja å ta stilling til i vår bacheloroppgåve, og det blei avklart
med både Renasys og Sunnfjord kommune at oppgåva skulle haldast teoretisk.


	
	% Kapittel 5 Krav og mål
	\chapter{Krav og mål}
\thispagestyle{fancy}
\gls{Renasys} hadde som krav at vi etablerte ein kravspesifikasjon i samråd med \gls{Sunnfjord Kommune}.
Vi hadde eit møte med kommunen der dei hadde nokon konkrete ønsker til oppgåva.

\begin{enumerate}
    \item Overgang frå system med høge lisenskostandar og utskifting av pls
    \item Forbetring og utvikling av ny dokumentasjon, som inkludera ei ny funksjonsbeskriving 
    \item Klargjering av styresystemet for integrering av ny funksjonalitet og komponentar
\end{enumerate}

\section{Krav}
Vi tok utgangspunkt i ønska frå \gls{Sunnfjord Kommune} for å forme kravspesifikasjon vår.
Deretter utarbeida vi ei liste med krav som naturleg delte seg i tre
hovuddelar: dokumentasjon, programmering og simulering med verifisering. Denne oppdelinga vart framlagt for oppdragsgivarane våra og godkjent av dei. 


%, fjernet "som vart gjort i samråd med rettleiar"

\begin{enumerate}
    \item Dokumentasjon med detaljert funksjonsbeskriving som inneheld:
        \begin{itemize} 
        \item Beskriving av anleggets verkemåte
        \item \gls{blokkdiagram}
        \item \gls{forrigling}, \gls{alarm}- og \gls{IO}-lister
        \item Oversikt over objekta
        \item Elektriske teikningar og \gls{PID}
        \item Vedlikehaldsmanual
        \end{itemize}
    \item Programmering, vi skal:
        \begin{itemize}
        \item Bruke den ny funksjonsbeskriving som grunnlag for programmeringa.
        \item Anvende open kjeldekode og ikkje låse oss til spesifikke leverandørar.
        \item Følge \gls{IEC} 61131\textemdash3 standard for programmering.
        \end{itemize}
    \item Simulering med verifisering
        \begin{itemize}
        \item Utvikle eit simuleringsverktøy for å teste og verifisere programmet
        \end{itemize}
\end{enumerate}

\newpage
\section{Mål}
Vi har satt følgande mål for systemet:

\begin{itemize}
    \item Programmere fire \gls{IEC} blokker: \gls{MB},\gls{MA},\gls{SBE} og \gls{SBV}
    \item Utvikle eit program som er enkelt og vedlikehalde
    \item Implementera tilstandsmaskin for alle prosess sekvensane
\end{itemize}

\section{Ny funksjonalitet og sensorar}
Følgande funksjonar og sensorar er ikkje nødvendig for programmets grunnleggande verkemåte, Men 
er ønska av \gls{Sunnfjord Kommune} om tiden strekker til.  
\begin{itemize}
    \item Temperatur, nivå og trykksensorar(reintvan inn).
    \item Ventil tilbakemeldingar og oksygenmåling
    \item Mengdemåling for overlaup og frekvensstyring av pumper
    \item Integrasjon av \gls{MJK} prøvetakar og energimåling
\end{itemize}



	% Kapittel 6 - Anleggets Verkemåte
	\chapter{Verkemåten til anlegget}
\thispagestyle{fancy}

For å setje oss inn i korleis spesefikt Sande reinseanlegg fungerar var vi nøydd til å forstå
korleis eit generelt avlaupsreinseanlegg er bygd opp. Dette var det naturlege første
steget inn mot løysningsalterntivet vi hadde valgt. 



	\section{Generell verkemåte}

Eit avlaupsreinseanlegg er bygd opp av tre hovuddelar, primær, sekunder og tertiærreinsing,
samt ein del for behandling av slam \citep{Regjeriga}.
Desse delane kan løysast på forskjellige måtar, men hovudoppgåvene er dei same
i alle reinseanlegg.

``Primærreinsing'' handlar om å skilje organisk og uorganisk materiale.
I eit avlaupsreinseanlegg tilsvarer dette å skilje avlaupsvatnet, 
som ein vil behandle, frå sand, Q-tips, våtserviettar og anna material som ein ikkje ønskjer vidare i prosessen.\newline
``Primærreinsing'' er eit viktig steg for å bevare pumper og anna prosessutstyr.

``Sekundærreinsing'' handlar om å fjerne mest mogleg suspanderte stoff og organisk materiale frå vatnet.
``Sekundærreinsing'' er i kvart anlegg avhengig av kva ``reinseprinsipp'' som er brukt. 
Dette tilsvarer kva teknolgisk metode som nyttast for å utføre steget.
``Sekundærreinsing'' refererast generellt som biologisk reinsing.

``Tertiærreinsing'' handlar om å fjerne resterande forureiningar i vatnet.
Dette steget varierer frå anlegg til anlegg og er
avhengig av kva krav reinseanlegget har på sitt utsleppsvatn.

Slamebehandling handler om å fjerne og behandle det oppbygde organiske materialet (slam)
som skillast ut i sekundær og tertiærreinsing 

\begin{figure}[htbp]
    \centering
    \includegraphics[width=1\textwidth]{Figurar/Generellverkemåte.png}
    \caption{Generell verkemåte for eit avlaupsreinseanlegg}\label{fig:GenerellVerkemåte}
\end{figure}

Då vi hadde fått oversikt over verkemåten til eit generellt avlaupsreinseanlegg, retta vi fokuset mot Sande.
Dette valde vi å dele opp i tre hovuddelar. Kva reinseprisipp er brukt, korleis Sande reinseanlegg
fungerar i praksis og om anlegget har nokre særeigna preg.

	\newpage
\section{Teknisk verkemåte}
\thispagestyle{fancy}
Sande reinseanlegg er konstruert og basert på \gls{SBR}-teknologi.

\gls{SBR} står for ``Sequence \Gls{batch} Reactor'', på norsk ``sekvensiell \gls{batch}reaktor''.\newline
\gls{SBR} er en reinsemetode der alle prosessar føregår i same reaktortank. 
Reaktor nyttar biologisk reinsing, ved hjelp av aktivert slam som inneheld mikroorganismar, for å koagulere 
og fjerne løyste og ikkje sedimenterbare partiklar samt stabilisere organisk materiale. 
Avlaupsvatn tilførast reaktor i ``batcher'' for å bli reinsa og uttappa. 
Kvar avlaups-batch går gjennom ein reaktorsyklus som består av følgjande fem delsekvensar \citep{Statsforvalter}.
\newline

\begin{figure}[htbp]
    \centering
    \includegraphics[width=1\textwidth]{Figurar/SBR-V2.png}
    \caption{\gls{SBR}-prossessen}\label{fig:SBR-Prosessen}
\end{figure}


\begin{enumerate}
    \item \textbf{\makebox[3cm][l]{Pause}:} Reaktoren er klar og ventar.
    \item \textbf{\makebox[3cm][l]{Innpumping}:} Reaktoren mottar avlaupsvatn, normalt ifrå ein utjamningstank.
    \item \textbf{\makebox[3cm][l]{Reaksjon}:} Reaktoren luftast periodisk for å tilføre oksygen til mikroorganismane.
    \item \textbf{\makebox[3cm][l]{Sedimentering}:} Reaktoren sedimenterer ved hjelp av gravitasjon. Overskuddslam fjernast.
    \item \textbf{\makebox[3cm][l]{Uttapping}:} Reaktoren drenerar reinsavatn mot resepient.
\end{enumerate}

Meir detaljert informasjon om \gls{SBR} og anleggets teknolgiske prinsipp er tilgjengeleg i anleggets
funksjonsbeskrivelse. (Vedlegg A)


	\newpage
\section{Praktisk verkemåte}
\thispagestyle{fancy}

Sande reinseanlegg består av ``primærreinsing'' via grovrist, ein mottakstank 
som samlar varierande tilstrøymingar for å gi resten av anlegget homogene forhold og
``sekundær'' og ``tærtierreinsing'' ved to reaktorar som anvender \gls{SBR}-teknologi.

Avlaupsvatn vil opphalde seg i eller på veg mot ein av desse fire hovuddelane medan det er i anlegget.
Ferdig behandla avlaupsvatn blir drenert ut til resipient (elva Gaula). 

\begin{figure}[htbp]
    \centering
    \includegraphics[width=1\textwidth]{Figurar/Sande verkemåte.png} %% FIKS FIGUR
    \caption{\gls{RA}200 flytskjema}\label{fig:SandeVerkemaate}
\end{figure}


\subsection{Grovrist}
Innløpet på anlegget renn først gjennom grovrista. Grovrista på reinseanlegget
er ein ``HUBER ROTAMAT Ro9'' som er ei grovrist som nyttar ein mekanisk skrue.
Den fungerer som ein liten tank og ein intern nivågivar startar
skruen ved innkommande avlaupsvatn. Skruen tek med uorganisk materiale og fjernar det til eigen avfallshandtering.
Dersom grovrista er ute av drift vil vatn renne vidare til mottakstanken via eit overløpsrøyr.

\begin{figure}[htbp]
    \centering
    \begin{subfigure}[b]{0.3\textwidth}
        \centering
        \includegraphics[angle=-90,width=1\textwidth]{Bilder/Huber.JPG}
        \caption{Motor og avfallshandtering}\label{fig:Huber}
    \end{subfigure}
    \hfill
    \begin{subfigure}[b]{0.3\textwidth}
        \centering
        \includegraphics[angle=-90,width=0.6\textwidth]{Bilder/Huber2.JPG}
        \caption{Gods og skrue}\label{fig:Huber2}
    \end{subfigure}
    \caption{Grovrist levert av Huber}\label{fig:HuberGrovrist}
\end{figure}

\newpage
\subsection{Mottakstank}
Frå grovrista renn vatn med sjølvfall mot mottakstanken som ligg som lågaste punkt på anlegget.
Mottakstanken er 120 $m^3$ og er ein felles lagringsplass for vatn før det går vidare mot reaktorane.
Mottakstanken har fire sensorar som heng frå taket.

Nivået i mottakstanken blir primært målt med trykkgivar LT01. For at vatn skal pumpast vidare mot ein
reaktor i riktig sekvens må trykkgivar indikere at nivået er høgt nok. LS02 fungerar som backup. \newline
I toppen av mottakstanken er det ei overløpskasse som drenerer mot resipient, her vil det
ved normale omstendigheter ikkje renne anna ein reinsa vatn. Trykkgivar for overløp måler
dersom ureinsa vatn renner i resipientrøyret.


\begin{figure}[htbp]
    \centering
    \includegraphics[width=1\textwidth]{Figurar/Mottakstank.png}
    \caption{P\&ID mottakstank}\label{fig:Mottakstank}
\end{figure}

\begin{figure}[htbp]
    \centering
    \includegraphics[width=1\textwidth]{Bilder/Bilde pumper.jpg}
    \caption{Matepumper}\label{fig:Matepumper}
\end{figure}

\newpage
\subsection{Reaktor}

Frå mottakstanken blir vatn pumpa opp til reaktor dersom den er i riktig sekvens.
Vatn blir pumpa ved hjelp av to matepumper som rullerer, der kvar pumpe kan levere til kvar reaktor. 
Reaktorane er 165 $m^3$ og står på bakkenivå og strekker seg opp mot taket på bygget.
Reaktorane er utstyrt med nivåmåling via trykk og sensor er plassert to meter over botn.

I reaksjonssekvensen blir reaktorane periodisk lufta. Dette er for å lage aerobe og anaerobe fasar
for mikroorganismane som vidare gir betre reinsing.
For å best spreie oksygenet i den aerobe fasen
er det satt inn eit \gls{diffuser}-oppsett i botn.
\Gls{diffuser}ane er laga av ein membran med små hol som dannar bobler når lufta kjem i 
kontakt med avlaupsvatn.
Lufting av reaktoren gir også effektiv omrøring utan behov for ektra mekanisk inngrep.

\begin{figure}[htbp]
    \centering
    \begin{subfigure}[b]{0.3\textwidth}
        \centering
        \includegraphics[width=1\textwidth]{Figurar/DiffusereMedOgUtanLuft.png}
        \caption{Illustrasjon \gls{diffuser}}\label{fig:Illustrasjon diffuser}
    \end{subfigure}
    \hfill
    \begin{subfigure}[b]{0.3\textwidth}
        \centering
        \includegraphics[width=1\textwidth]{Figurar/DiffuserFraTopp.png}
        \caption{Oppsett av \gls{diffuser}}\label{fig:Oppsett diffuser}
    \end{subfigure}
    \caption{\Gls{diffuser} system}\label{fig:Illustrasjon-Diffuser}
\end{figure}

På reinseanlegget er det tilleggskrav for fjerning av fosfor \citep{Regjeriga}. På grunn av desse tilleggskrava
er det sett inn eit tertiærreinsesteg ved hjelp av simultanfelling.
Simultanfelling er ein fellesbetegnelse på kombinert biologisk og kjemisk reinsing.

I slutten på reaksjonssekvensen tilsettest polyaluminium klorid, og kjemikaliet binder seg til
løyst fosfor og danner sedimenterbare partiklar \citep{Pax18}. Desse sedimenterbare partiklane synk 
så i sedimenteringssekvensen og utsleppskravet på fosfor oppretthaldast.

\begin{figure}[htbp]
    \centering
    \includegraphics[angle=-90, width=1\textwidth]{Bilder/BildeReaktor.jpg}
    \caption{Reaktor}\label{fig:reaktorsoner}
\end{figure}

\newpage

Reaktorane er delt opp i tre forskjellige soner. Desse er med på å skilje
dei forskjellige substansane når reaktoren er ferdig med ein syklus.
Alle \gls{SBR}-reaktorar har lagra aktivert slam i botn \citep{Statsforvalter}. Det er her alle mikroorganismane akkumulerast 
og gir grunnlag for god biologisk reinsing. Denne sona blir kalla slamsona.\newline
Sjølve bruksvolumet til reaktoren er alt over uttaksventilen, 
og det er dette volumet som blir fylt og behandla under kvar innpumpingsekvens.

Mellom dei to sonene er det ei sikkerheitssone.
Denne sona er med for å ta hand om varierande sedimenteringseigenskapar
og eventuelt overskotsslam.\newline

\begin{figure}[htbp]
    \centering
    \includegraphics[width=0.5\textwidth]{Figurar/Reaktorsoner.png}
    \caption{Illustrasjon av reaktorsoner}\label{fig:Reaktorsonar}
\end{figure}

\subsection{P\&ID}

Arbeidet rundt anleggets verkemåte var meir problematisk enn forventa.
Hovudgrunnen til dette var at det ikkje fantest noko røyrgateskjema eller teknisk planteikning. \newline
Vi såg det som naudsynt å etablere ein \gls{PID} for å betre dokumentere og vise samanhengen til anlegget.

\begin{figure}[htbp]
    \centering
    \includegraphics[angle=90,width=1\textwidth]{Figurar/PID.drawio.png}
    \caption{\gls{PID}}\label{fig:HMI}
\end{figure}
	\newpage
\section{Spesefikke anleggs forskjeller}
\thispagestyle{fancy}

Sjølv om Sande reiseanlegg anvender \gls{SBR}-teknologi så er det enkle spesefikke
punkter der dette reiseanlegget avviker ifrå normalen. 
Sande reinseanlegg bruker eksempelvis ein spesiel form for slambehandling sett i norsk perspektiv.

Vanleg slambehandling er deponering av slam som handterast og sendast til \gls{Hygienisering} \citep{Slam}.
På Sande blir ikkje slammet lagra, men jamnlig spreid ut over eit designert område. På dette området er
det planta siv som skal ta opp slammet og det resterande vatnet blir naturleg filtrert og drenert.
Desse områda kallast sivbed.

Sivbeda er konstruert med fleire dreneringslag som gjer at resterande vatn skiljast ut i designerte soner.
Her kan desse handterast vidare etter ønska behov. 
Grunna denne slambehandlingsmetoden er det på Sande reiseanlegg heller slamfjerning frå reaktor
i reaksjonsfasen. Dette blir gjort for å ha mindre konsentrert slam.

Sande reiseanlegg har fire sivbed og den kombinerte størrelsen er 676 $m^2$. Sivbeda er lokalisert på utsida av reiseanlegget (sjå figur~\ref{fig:Sivbed}).\newline
Videre har kvart sivbed sin eigen respektive ventil (sjå figur~\ref{fig:SivbedVentilar}), og under slambehandlinga vil kun ein ventil vere aktiv. 
 

\begin{figure}[htbp]
    \centering
    \includegraphics[width=1\textwidth]{Figurar/Sivbed.png}
    \caption{Illustrasjon sivbed}\label{fig:Sivbed}
\end{figure}

\newpage

\begin{figure}[htbp]
    \centering
    \includegraphics[width=1\textwidth]{Bilder/SatelittFoto.png}
    \caption{Satelitt foto frå Google Earth \citep{Google} }\label{fig:Sivbed}
\end{figure}

\newpage

Vatnet frå desse forskjellige dreneringslaga blir samla til eit pumpehus/komme.
Dette pumpehuset står ca femti meter ifrå sjølve reinseanlegget og er utstyrt med to pumper og nokre nivåvipper.
Pumpehuset er delt i to og skiljer på vatnet som kjem ifrå dei forskjellige dreneringslaga.\newline
På Sande reiseanlegg er djupaste dreneringssona klassifisert som reinsa vatn (sivbed reject) og blir sendt ut til resepient.
Den øvre dreneringssona er forsatt klassifisert som skitten og blir returnert til mottakstanken. \newline \newline \newline \newline \newline

\begin{figure}[htbp]
    \centering
    \includegraphics[width=1\textwidth]{Figurar/Pumpehus.drawio.png}
    \caption{\gls{PID} pumpehus}\label{fig:Pumpehus}
\end{figure}

\begin{figure}[htbp]
    \centering
    \includegraphics[width=1\textwidth]{Bilder/SivbedSande.jpg}
    \caption{Sivbed ventilar}\label{fig:SivbedVentilar}
\end{figure}

\newpage




	\newpage
\section{Dokumentasjon}
\thispagestyle{fancy}

Som tidlegare spesifisert i kapittel 3 var ein stor del av oppgåva vår og dokumentere
anlegget og verkemåten til anlegget. Vi bestemte oss for å gjere ei dokumentasjonsfornying,
altså at vi henta alt det som var av tidlegare dokumentasjon og kombinerte det med vår nyerfarte kunnskap.

Vi oppretta ein funksjonsbeskrivelse som bygger vidare på driftsinstruksen som var levert av 
Watercare i 2003. Dokumentet er tiltenkt ein slags bruksanvisning på heile anlegget,
der sikkerheit, prosess, verkemåte og programmering er sentrale tema.

Funksjonsbeskrivelsen skal være forståeleg for alle intresserte parter og inneheld 
mange av våre nye figurar og avsnitt. Dokumentet innheld detaljert kvifor og korleis
ting heng saman og gir derfor ein forståelse for anlegget som ikkje var mogleg med dokumentasjonen som var tilgjengeleg før.

Funksjonsbeskrivelsen inneheld alt om reinseanlegget på Sande men er delt opp i kapittel for å ikkje
overvelde lesar med mykje unødvendig informasjon. Desto lenger ned i dokumentet ein kjem jo meir avansert blir informasjonen,
og programmering av anlegget ligger under udjupa teknisk beskrivelse.

Dokumentet er delt opp i desse kapittela

\begin{enumerate}
    \item Introduksjon
    \item Verkemåte
    \item Teknisk beskrivelse
    \item Drift og vedlikehald
    \item Feilsøking
    \item Utdjupa teknisk beskrivelse
    \item Teknisk underlag
\end{enumerate}

Alt som er nemnt her i kapittel 4 står meir detaljert under kapittel Teknisk beskrivelse i funksjonsbeksrivelsen, og
er sentralt for å best forstå korleis anlegget fungerer.

Funksjonsbeskrivelsen i sin heilhet ligger som vedlegg (SETT INN VEDLEGG)

	% Kapittel 7 - Tilrettelegging for programmering
	\chapter{Tillrettelegging for programmering}
\thispagestyle{fancy}
\label{sec:5} 

Som tidlegare beskrevet under krav og mål, ønskte vi å bruke ei løysning som ikkje hadde høge lisenskostnadar for vår sluttkunde. 
Sunnfjord kommune var interessert i å ikkje låse seg til ein fast leverandør, men heller ha moglegheita til å ha valet mellom fleire 
leverandørar innan PLS. Dette ilag med moglegheiten til å breie vår eigen kompetanse 
gjorde at vi såg vekk ifrå Siemens TIA-portal, som vi hadde lært igjennom PLS emnet, 
og begynte å sjå i andre endar.
Programmet ønska vi å skrive i Structured Text (ST) men undersøkte også 
typar av grafiske-diagram baserte språk for å lettare vise samanhengar i programmet.

Det var også viktig at alle på gruppa kunne programmere samtidig og at ein felles programmeringsstandard skulle nyttast.
Det var viktig at parallelt arbeid ikkje skulle by på synkroniserings problem og at det fantest ei god løysning for dette.


	\section{Codesys}
\thispagestyle{fancy}
\gls{Codesys} laga av \gls{Codesys} Group tilbyr ein open kjeldekodeløysning for prosjeket og har ingen lisenskostnadar. \citep{CodesysLisens}. 
I tillegg så kan prosjektfilene nyttast på fleire typar PLS einingar \citep{CodesysPLS}. 
Dette gir vår oppdragsgivar fleksibilitet i korleis dei ønskjer å implementere løysningsforslaget.

\gls{Codesys} nyttar programeringsspråkstandarden satt av \gls{IEC} 61131-3 som omfattar \gls{ST}, ``Sequential Function Chart'' (\gls{SFC}) og ``Ladder Diagram'' (\gls{LD}). 

\gls{Codesys} har nyleg fått støtte for integrering av \gls{github} i programvaren \citep{CodesysGIT}. 
Dette gjer det enklare å halde versjonskontroll og for gruppemedlem å programmere saman. 
\gls{github} har vi nytta på andre prosjekt og det har vi god erfaring med.

Vidare så har \gls{Codesys} støtte for bibliotek gjennom \GLS{Codesys} Store \citep{CodesysStore}. 
Ved å nytte kjende bibliotek, får vi tilgang til ei samling av gjenbrukbare kodeblokker og funksjonar.
Dette er bilbioteka vi ønskjer å nytte

 
\begin{itemize}
    \item \GLS{Codesys} Building Automation \citep{BuildingAutomation}
    \item SysTime \citep{DateAndTime}
    \item Util \citep{Util} \newline \newline
\end{itemize} 

\begin{figure}[htbp]
    \centering
    \includegraphics[width=0.6\textwidth]{Bilder/Codesys.png}
    \caption{Codesys programmvare}\label{fig:Codesys}
\end{figure}

\newpage

	\section{IEC}
\thispagestyle{fancy}
\label{sec:5.2}


\gls{IEC} \citep{IEC} er ein internasjonal, ikkje statleg orginasjon som utviklar og publiserer tekniske standardar innan elektrofag. 
Norge er representert i IEC ved Norsk Elektrotekniske Komité (NEK) \citep{IEC-SNL}. 
IEC har standarar som dekker programmering av PLS som går heilt tilbake til 1993\citep{Wiki-93}. 
Den nåverande standaren som omfamnar PLS er IEC 61131\citep{IEC-61131}. Dette er ein standard spesielt designet for programmerbare kontrollarar, og er delt opp i 10 delar, der del 3 tar for seg programmeringsspråk. 

Vårt program er i hovudsak tiltenkt og programmerast etter IEC 61131-3 og IEC \gls{PAS} 63131\citep{IEC-63131}. 
Der IEC PAS 63131 er ein standard utarbeida av IEC som gir oss grunnlag for å lage \gls{SCD} samt å bruke forhandsdefinerte funksjonstemplater for funksjonsblokker. 
IEC PAS 63131 er laga med formål at leverandørindustrien og oljeselskap skal ha eit felles rammeverk for bruk på norsk sokkel, og er utarbeida etter NORSOK I-005:2013.
Ved å bruke desse standardane så gir det oss eit robust og fleksibelt rammeverk for å programmere anlegget. 
Ved bruk av dei forhandsdefinerte funksjonsblokkane har vi moglegheit til å enkelt knytte i hop fleire delar av programmet våra, og ha fleksibilitet ved å enkelt kunne endre og legge til funksjonar i programmet. 
Dette er noko vi har heile vegen har fokusert på, då vår sluttkunde har ytra eit ønske om fleire tilleggsfunksjonar i programmet, utover det som er der i dag. 

Vi bestemte oss for å fokusere på fire funksjonsblokker (MB, MA, SBV og SBE, sjå appendiks \ref{sec:IEC-Blokker})  
i frå IEC 63131 for å få dekka behovet for dei komponentane vi hadde, og for å kunne programmere anlegget i sin heilheit.
Sjølv om blokkene kunne verke overkvalifiserte valde vi likevel å ha dei med, da nokre av tilleggsfunksjonane eventuelt kunne nyttast seinare
og at det gav oss ein klar retning å arbeide mot.
\newpage
	
	\section{SCD}
\thispagestyle{fancy}


 \gls{SCD} (System Control Diagram) er eit grafisk dokumentasjons verktøy beskrevet i IEC PAS 63131.\newline
 \gls{SCD} blir brukt for å vise relasjonane mellom prosessens komponentar og programmet som styrer dei og skiljer seg frå eit \gls{PID}
 som dokumenterer alt fysisk utstyr.

 Vi ynskjer å bruke SCD som eit planleggingsverktøy for programstrukteren. SCD inneheld IEC templat noko som gjer at ein kan
 planlegge programmering av anlegget før ein har skreve sjølve IEC blokk-koden
 Diagrammet gir oss mulighet til å visualsere, teikne og kople IEC blokkene mot komponentane på anlegget og 
 gir oss også ein unik moglegheit for å kunne dokumentere arbeidet og vil gi ein grafisk representasjon
 av styringsform og løysningar som blir valgt.

 Vi har kontakta MIDTechology\citep{MIDT} som er som er eit selskap som utviklar programvare for SCD og har fått utlevert studentlisensar
 til programvaren. Vi vil bruke dette programmet for planlegging og dokumentering av koden til reiseanlegget på Sande.

 \begin{figure}[htbp]
    \centering
    \includegraphics[width=0.35\textwidth]{Bilder/Visio_eksempel.png}
    \caption{Eksempel av SCD}\label{fig:SCD eksempel}    
\end{figure}

\newpage
	\section{Tilstandsmaskin}
\thispagestyle{fancy}

Vi identifiserte tidleg i prosessen eit ynskje om å lage ei tilstandsmaskin som hadde overordna styring over kvar reaktor. 
Sidan ein \gls{SBR}-reaktor er prinsipielt oppbygd av forskjellige sekvensar, virka det logisk å implimentere ei tilstandsmaskin.
Tilstandsmaskina vil ha ansvar for å oppretthalde korrekt tilstand og avansere vidare når gitte kriterier er nådd.

\begin{figure}[htbp]
    \centering
    \includegraphics[width=1\textwidth]{Figurar/Tom tilstandsmaskin.png}
    \caption{Tilstandsmaskin prinsipp}\label{fig:Tilstandsmaskin prinsipp}    
\end{figure}

\newpage



	% Kapittel 8 - Programmering
	\chapter{Programmering}
\thispagestyle{fancy}
Etter å ha danna oss eit godt grunnlag og eit bilete av korleis vi ønskja programmet i kapittel~\ref{sec:7} 
byrja vi med programmeringa med eit tomt prosjekt i \gls{Codesys}. Første steget i var å 
settje oss inn i, og programmere etter \gls{IEC} funksjonsblokk malane.



	\section{IEC Funksjonsblokker} \label{IEC Seksjon}
\thispagestyle{fancy}

Vi gjore eit utval av funksjonsblokk maler basert på dei komponentane vi hadde identifiserte i reinseanlegget.
For å laga eit robust program valde vi og fokusere på: \gls{MA}, \gls{MB}, \gls{SBE} og \gls{SBV}.
Sjølv om funksjonsblokk malane kunne verke noko overkvalifiserte valde vi likevel å ha dei med, då nokon av tilleggsfunksjonane eventuelt kunne nyttast seinare
og at det gav oss ein klar retning å arbeide mot.

\gls{IEC} har sentrale begreper som vi ynskjer å utdjupe nærmare. Grunna mangel på gode norske begreper 
og for å forhindre forvirring har vi valgt å beskrive begrepa slik dei er definert i normen.

\begin{itemize}
    \item \textbf{Lock:} Action overruling any other signal while being true
    \item \textbf{Force:} Action overruling any other signal
    \item \textbf{Disable Transition:} Transistion high/low function not avaliable
    \item \textbf{Blocking:} Prevention of certain functions or operations 
    \item \textbf{Suppression:} Disable alarm annunciation as well as any associated automatic actions
\end{itemize}

Alle funksjonsblokk maler som er designet etter \gls{IEC} \gls{PAS} 63131 har alle moglegheit for handtering og varsling av feil 
Desse funksjonane er veldig viktig i videre arbeid mot feilhandtering og alarmliste.

\newpage

\subsection{Monitor Binary}
\gls{MB} funksjonsblokka (Vedlegg C.1) blir brukt til automatisk overvåking, alarmhandtering, framvising og låsing av binære prosess variablar.
Funksjonsblokka inkluderer alarm ``suppression'' og ``blocking'' funksjonalitet. Den har moglegheit for invertering av 
inngangssignal og moglegheit for tidsforseinking av utgangssignal via parameter.

Funksjonsblokka er brukt i programmet for å overvaka alle digitale nivåvipper og trykkbrytarar i prosessen.


\begin{figure}[htbp]
    \centering
    \begin{subfigure}[b]{0.45\textwidth}
        \centering
        \includegraphics[width=1\textwidth]{Bilder/MBBlokkIEC.png}
        \caption{IEC}\label{fig:Monitor Binary blokk IEC}
    \end{subfigure}
    \hfill
    \begin{subfigure}[b]{0.45\textwidth}
        \centering
        \includegraphics[width=0.7\textwidth]{Bilder/MBBlokkIProgrammet.png}
        \caption{Bruk i programmet}\label{fig:Monitor Binary blokk i programmet}
    \end{subfigure}
    \caption{Monitor Binary}\label{fig:Monitor Binary}
\end{figure}

\subsection{Monitor Analogue}
\gls{MA} funksjonsblokka (Vedlegg C.2) er brukt for skalering, visning, overvåking og alarmhandtering av \newline
analoge inngangsvariablar i ein prosess.
Funksjonsblokka inneheld ``suppression'' og ``blocking'' funksjonalitet.

Funksjonsblokka er brukt i programmet for å overvåke analoge trykknivågivarar samt å skalere og vise desse som ein fyllingsgrad i prosent.

\begin{figure}[htbp]
    \centering
    \begin{subfigure}[b]{0.45\textwidth}
        \centering
        \includegraphics[width=1\textwidth]{Bilder/MABlokkIEC.png}
        \caption{IEC}\label{fig:Monitor Analogue blokk IEC}
    \end{subfigure}
    \hfill
    \begin{subfigure}[b]{0.45\textwidth}
        \centering
        \includegraphics[width=0.7\textwidth]{Bilder/MABlokkIProgrammet.png}
        \caption{Bruk i programmet}\label{fig:Monitor Analogue blokk i programmet}
    \end{subfigure}
    \caption{Monitor Analogue}\label{fig:Monitor Analogue}
\end{figure}

\newpage

\subsection{Switch Binary Eletrical} 

\gls{SBE} funksjonsblokka (Vedlegg C.3) blir brukt for binærkontroll (av/på) av straumningselement for elektrisitet, varme eller væske. Den
kontrollerte komponenten kan være av typen motor, pumpe, varmeelement, vifte osv.
Funksjonsblokka er i dette programmet brukt til å styre motorar, pumper og blåserar.

Funksjonsblokka beskriver korleis ein styrer ein komponent.
Det er utgangen Y, som sender ein opne/stenge kommando (høg/lav) til komponenten. Blokka har fleire funksjonar, der den
tar utgangen og samanliknar med tilbakemelding \gls{XGH} som gjer korrekt \gls{BCL}/\gls{BCH} status. 

Funksjonsblokka inkluderar alarm ``suppression'', ``blocking'', ``safeguarding'' og ``transition'' funksjonalitet.

\begin{figure}[htbp]
    \centering
    \begin{subfigure}[b]{0.45\textwidth}
        \centering
        \includegraphics[width=1\textwidth]{Bilder/SBEBlokkIEC.png}
        \caption{IEC}\label{fig:Switch Binary Eletrical blokk IEC}
    \end{subfigure}
    \hfill
    \begin{subfigure}[b]{0.45\textwidth}
        \centering
        \includegraphics[width=0.5\textwidth]{Bilder/SBEBlokkIProgrammet.png}
        \caption{Bruk i programmet}\label{fig:Switch Binary Eletrical blokk i programmet}
    \end{subfigure}
    \caption{Switch Binary Eletrical}\label{fig:Switch Binary Eletrical}
\end{figure}

\newpage

\subsection{Switch Binary Valve}

\gls{SBV} funksjonsblokka (Vedlegg C.4) skal brukast til binær av/på kontroll av eit straumningselement ved å endra straumen av medium (varme eller væske). 
Typisk komponentar som styrast er bl.a.\ ventilar og spjeld.
Funksjonsblokka er i dette programmet brukt til å styre ventilar.

Funksjonsblokka styrer ventilen ved hjelp av dei binære inngangane \gls{XH} og \gls{XL}.\@
Desse inngangane styrer ein utgang Y, som sender opne/stenge-kommando (høg/låg) til ventilaktivatoren.
Alternativt kan dei pulsmodulerte utgangane \gls{YH} og \gls{YL} kan også nyttast.

Funksjonsblokka har også inngangar \gls{XGH} og \gls{XGL} som gjer tilbakemelding om ventilen er heilt open eller stengd
som bekrefter ventilen sin posisjon.

%\textbf{Kontrollfunksjonane i funksjonsblokka inkluderar:}
%\begin{itemize}
    %\item Generering av feilstatus (YF) om det oppstår ein intern eller ekstern feil.
    %\item Blokka set utgangen Y i samsvar med parameter når feil blir oppdaga.
    %\item Blokka set utgangen Y basert på tilbakemelding i ``outside mode'' når ingen eksterne inngangar blir brukte (XOH/XOL).
%\end{itemize}

Funksjonsblokka inkluderar alarm ``suppression'', ``blocking'', ``safeguarding'' og ``transition'' funksjonalitet.

\begin{figure}[htbp]
    \centering
    \begin{subfigure}[b]{0.45\textwidth}
        \centering
        \includegraphics[width=1\textwidth]{Bilder/SBVBlokkIEC.png}
        \caption{IEC}\label{fig:Switch Binary Value blokk IEC}
    \end{subfigure}
    \hfill
    \begin{subfigure}[b]{0.45\textwidth}
        \centering
        \includegraphics[width=0.5\textwidth]{Bilder/SBVBlokkIProgrammet.png}
        \caption{Bruk i programmet}\label{fig:Switch Binary Value blokk i programmet}
    \end{subfigure}
    \caption{Switch Binary Value}\label{fig:Switch Binary Value}
\end{figure}

\newpage



\newpage

%\begin{figure}[htbp]
%    \centering
%    \begin{subfigure}[b]{0.45\textwidth}
%        \centering
%        \includegraphics[width=1\textwidth]{Bilder/4_20mA_Scaling.png}
%        \caption{Skalering av mA mot prosent}\label{fig:Skalering av mA mot prosent}
%    \end{subfigure}
%    \hfill
%    \begin{subfigure}[b]{0.45\textwidth}
%        \centering
%        \includegraphics[width=0.95\textwidth]{Bilder/27327_prosent_Scaling.png}
%        \caption{Skalering av prosent til verdi}\label{fig:Skalering av prosent til verdi}
%    \end{subfigure}
%    \caption{Dei forskjellige skaleringane av inngangssignal}\label{fig:Skalering av prosent til verdi}
%\end{figure}



	\section{Generelle funksjonsblokker}
\thispagestyle{fancy}

% Legge inn forklaring av "felleskoden", utrekningar', sivebedrotasjon', dataprossesing', Høgbelastningsprogram', ProvessedWater.
% Alt under her er er ikkje skrevet i stein.

Undervegs i programmeringa av \gls{IEC} blokkene såg me også nødvendigheita av nokon generelle funksjonsblokker
som kunne gjenbrukast fleire gonger i programmet. Dette er då hensiktsmessigt sidan ein då slepp og skrive lik 
funksjonalitet fleire gonger, men heller kallar ei \gls{FB} som kann gjere denne same jobben.

\subsection{fbTimer}
Timer (Sjå appendix xXx) \gls{FB} kan brukast om ein treng ein tids forseinking i programmet.
Her kan ein nytta tidsforsinka inn, tidsforsinka ut eller ein kombinasjon av begge.

\subsection{fbAnalougeAlarm}
Analogue alarm (sjå appendix xXx) \gls{FB} brukast til å overvåke, tidsforsinke, behandle grenser, 
gje alarmar og legge på hystereser på ferdig skalerte analoge inngangsverdiar.

\subsection{fbDigitalAlarm}
Digital alarm (Sjå appendix xXx) \gls{FB} kan overvåke, tidsforsinke og gje alarmer. Det er valbart om blokka skal trigge på høg eller låg
inngang basert på ein parameter.

\subsection{fbSwap}
Swap \gls{FB} (sjå appendiksar) får ein inngangsverdi og sekvensielt bytter på og bruke to utgangsverdiar. Det vil sei at blokka hugsar på kva utgang som vart brukt sist,
og vil bruke den andre utgangen ved neste kall. Blokka har også moglegheit for feilhandtering.

\subsection{fbCalculations}
Calculations \gls{FB} (Sjå appendiks xXx) gjer nokre rekneoppgåver som ligger i bakgrunnen og køyrer kontinuerleg. 
Det er i hovudsak utrekningar av volum i reaktor, mottakstank og drenert volum i frå reaktortank.

\subsection{fbTimeMeter}
Time meter \gls{FB} (Sjå appendiks xXx) er ei blokk som teller tid så lenge den blir kalla på og lagrar desse verdien i forskjellige tidsformater.
Denne er brukt i programmet til å telle gangtid for forskjellige eletriske komponentar.

\subsection{fbHighLoad}
High Load \gls{FB} (Sjå appendiks xXx) lokka blir brukt til og overvåke antatt innstraumning i mottakstanken basert på nivå endringa i tanken. 
Blokka kalkulerar gjennomsnittleg tilstraumning ved å sample tank nivået kvart minutt over totalt 30 minutt.
Denne kalkuleringa blir samanlikna med eit parameter PXR 'setpunkt' for å sette annlegget i høgsbelastningsmodus.
Blokka gjer også antatt tilstrøymning per time.

\subsection{fbSivbedRotation}
Anlegget har fire ventilar til sivebedet som det er viktig å oppretthalde rotasjon imellom. Anlegget tappar frå ein reaktor om gongen, 
og \gls{FB} Sivbedrotation vel kva ventil som skal aktiverast.
Rotasjon blir angitt av ein parameter som fastset kor mange syklusar kvar ventil skal nyttast. 
Det er også mogleg å ta ein ventil ut av rotasjonen, slik at den ikkje blir inkludert i syklusen.

\subsection{fbDataprocessing}
fbDataprocessing (Sjå appendiks xXx) blir brukt i samband med innsamling av driftsdata. 
Når nokon av objekta som er tiltenkt å ha driftsdata køyrer eller er aktive, blir desse sendt til fbDataprocessing, og derifrå kallar den opp fbTimeMeter som lagrar og teller driftsdata for objektet.

\subsection{fbProcessedWater}
fbProcessedWater (Sjå appendiks xXx)
Det er laga ei eiga blokk for å halde oversikt over driftsdata som omhandlar behandla vann for anlegget. 
Denne blokka gir informasjon for totalt behandla vann, nåverande år, nåverande månad, nåverande veke, nåverande dag, førre år, færre månad, færre veke og førre dag.
	\section{Tilstandsmaskin}
\thispagestyle{fancy}

Då \gls{IEC}-blokkene var ferdige, byrja vi på tilstandsmaskina som skulle styre sjølve \gls{SBR}-prosessen. 
Vi hadde allereie danna oss eit bilete, men kunne no byrje å nytte kunnskapen frå anleggets verkemåte
til å grovt fylle inn dei hendelsane og aksjonane som foregikk mellom tilstandane. 
Tilstandsmaskina er bygd opp av dei fem reaktorsekvensane som eksisterar i eit \gls{SBR}-anlegg.

Dette er ein enkel modell av korleis tilstandsmaskina er programmert, men gir eit godt innblikk i \newline funksjonaliteten. \newline \newline \newline \newline \newline

\begin{figure}[htbp]
    \centering
    \includegraphics[width=1\textwidth]{Figurar/Simpel tilstandsmaskin.png}
    \caption{Enkel model av tilstandsmaskin}\label{fig:SimpelTilstandsmaskin}
\end{figure}


\newpage

I sjølve programmeringa av tilstandsmaskina blei den oppretta som ei eiga funksjonsblokk, noko som gav oss moglegheita å nytte blokka for begge reaktorane.
Tilstandsmaskina er laga med fem inngangar og seks utgangar, og baserer seg på ``switch/case'' logikk.

Tilstandsmaskina sender ut høg på den respektive utgangen som samsvarer med reaktortilstanden den er i. Dersom tilstandsmaskina får tilbake
høg på den respektive tilstandsinngangen avanserer \newline tilstandsmaskina.
Det er også mogleg å hente ut aktiv tilstand ved hjelp av ein heiltallsverdi (1-5). \newline \newline \newline \newline

\begin{figure}[htbp]
    \centering
    \includegraphics[width=0.6\textwidth]{Bilder/Tilstandsmaskin.png}
    \caption{Tilstandsmaskin implementert i programmet}\label{fig:TilstandsmaskinIProgram}
\end{figure}



	\newpage
\section{Tilstandslogikk}
\thispagestyle{fancy}

Styring og logikk som skulle skje i kvar tilstand, valde vi å samle i ei funksjonsblokk som vart kalla tilstandslogikk og fikk navn etter tilstanden
den skulle styre t.d. Reaksjon. % Desse funksjonsblokkene er skrevet og løyst spesefikt for deira arbeidsoppgåver i programet.

Kvar tilstandslogikkblokk får inn ``external enable'' (\gls{XE}) frå tilstandsmaskina som startar tilstandslogikken. Når sekvensen er ferdig sender
funksjonsblokka høg på utgang Y som returerast til tilstandsmaskina som avanserer til neste tilstand og tilstandslogikkblokk.

Det er tilstandslogikken som har ansvar for å samarbeide med \gls{IEC}-blokkene som handterar
start/stopp av elektrisk utstyr, kontrollerer tilbakemeldingar, feilmeldingar og skriv akutelle parameter.

\begin{figure}[htbp]
    \centering
    \begin{subfigure}[b]{0.6\textwidth}
        \centering
        \includegraphics[width=1\textwidth]{Bilder/fbInnpumping.png}
        \caption{Innpumping}\label{fig:fbInnpumping}
    \end{subfigure}
    \hfill
    \begin{subfigure}[b]{0.3\textwidth}
        \centering
        \includegraphics[width=1\textwidth]{Bilder/fbReaksjon.png}
        \caption{Reaksjon}\label{fig:fbReaksjon}
    \end{subfigure}
    \caption{Tilstandslogikkblokker implmentert i programmet}\label{fig:ReaksjonsFasen}
\end{figure}




	\newpage
\section{Oppbygging av programmet}
\thispagestyle{fancy}

\subsection{Programmeringsmetode}
For å sette saman alle funksjonsblokkene vi hadde programmert i \gls{ST}, valde vi å bruke \gls{Codesys} ``Continuous Function Chart'' (\gls{CFC}).
\gls{CFC} er eit grafisk programmeringsspråk som nyttar symbol og koplingar for å gjere programmet meir visuelt.

Alle samankoplingar av blokker valde vi å gjere i \gls{CFC}. Ved å bruke ein grafisk metode sikra vi oss god lesbarheit og
visuell forståelse av programmet. 

Alle inngangar og utgangar er leselege og enkle og forstå. \gls{CFC} i lag med god dokumentasjon vil kunne gi personar utan programmeringsbakgrunn
god forståelse av korleis programmet er oppbygd, utan å måtte lese kodelinjer.
\gls{CFC} gir eit godt grunnlag for feilsøking og analyse, dette bygger vidare på filosofien med eit enkelt og fleksibelt program.

Dersom antall koplingar og linjer gjorde programmet vanskeleg å lese var det også
mogleg å opprette ``source'' og ``links'' som oppretta ein trådlaus forbindelse gjennom ein unik ID.

\begin{figure}[htbp]
    \centering
    \includegraphics[width=1\textwidth]{Bilder/ReaktorPRG.png}
    \caption{Eksempel \gls{CFC} - Styring reaktor 1}\label{fig:CFCReaktor}
\end{figure}

\newpage

\subsection{Hovuddel}

Programmet er delt opp i tre hovuddelar, ei tilstandsmaskin for kvar reaktor og ein del for samling av felles reaktorfunksjonar.
Alle delane blir utført kvar \gls{PLS} syklus. 
Tilstandsmaskina har det overordna ansvaret og passar på kva tilstandslogikkblokk som nyttast.

Fellesfunksjonar er ei samling av funksjonsblokker og utrekningar som er felles for reaktorane, og er uavhengig av tilstandsmaskinene.
Driftsovervaking er ein sentral del av fellesfunksjonar, der gangtider og mengde prosessert vatn er døme utførte berekningar.

I nokre tilfelle, som ved rullering av sivbed, var vi avhengig at begge tilstandsmaskinene hadde same informasjon.
Dette løyste vi ved å lage ei funksjonsblokk ``fbSivbedRotation'', i felles funksjonar, som hentar inn og behandlar antall slamuttak for å rotere sivbed når ei gitt grense er nådd.
Denne informasjonen blir deretter sendt til kvar tilstandsmaskin som sørger for at begge reaktorane har same aktive sivbed.

\begin{figure}[htbp]
    \centering
    \includegraphics[width=1\textwidth]{Figurar/Oppbygging_Program.png}
    \caption{Illustrasjon oppbygging program}\label{fig:OppbyggingProgram}
\end{figure}

\newpage

\subsection{Styring tilstandslogikk}

Som tidlegare skildra er det tilstandslogikken som samarbeider med \gls{IEC} blokkene. Dette samarbeidet valde vi også å gjere i eit
\gls{CFC} vindu som gjorde kall og koplingar meir visuelt. \gls{CFC} vindauget fikk namn etter kva sekvens i \gls{SBR}-prosessen
den hadde ansvar for å styre. 

Oppbygginga av desse sekvensstyringane er gjort med inngangsblokker (\gls{MA} og \gls{MB}) øvst og utgangsblokker (\gls{SBE} og \gls{SBV}) i botn.
Mellom desse kjem tilstandslogikkblokka som inneheld styringslogikk.

\begin{figure}[htbp]
    \centering
    \includegraphics[width=1\textwidth]{Bilder/Heile_innpump.png}
    \caption{Eksempel \gls{CFC} - styring innpumping}\label{fig:CFCInnpumping}
\end{figure}

I denne figuren er ekstra inngangar, parameterinngangar og utgangar fjerna for å betre kunne visualisere koplingane og samarbeidet mellom
\gls{IEC} blokkene og tilstandslogikk.

\newpage


	\newpage
\section{Alarm og feilhandtering}
\thispagestyle{fancy}

% Skriver litt om introduksjon om alarm og feilhandtering
Alarm og feilhandtering utgjer ein sentral del av eit velfungerande styresystem. Det er avgjerande
at anlegget effektivt handterer og varslar om uønskte hendingar, slik at 
driftspersonalet blir varsla og nødvendige tiltak kan raskt setjast i verk.

% Skrive litt om codesys alarmhantering
For å varsle om alarm og feil i vårt program, har vi nytta oss av \gls{Codesys} sine
innebygde alarmhandteringsfunksjonar \citep{CodesysAlarm}. Desse funksjonane gir oss høve til å 
kontrollere, gruppere og prioritere alarmar, samt å sende informative tekstar til driftspersonalet.

% Skriver om alarmar både gamle og nye.
\gls{IEC}-blokkene som vi har utvikla, gir oss moglegheita til å detektere ulike typar hendingar
som feil, alarmar og forvarsel.
Dei forskjellige blokkene har ein variert mengde med feil som skal kunne detekterast og varslast.
Forløpet er slik at feil blir detektert og ein boolsk variabel \gls{YF} blir sett til ``true'', 
samt at ein heiltalsverdi \gls{YFI} indikera kva type feil som har inntruffe.\newline
Alle desse moglege typane varslingar, som er omtala i vedlegg E, 
er samla i programmet vårt saman 
med dei aktive varslingane som er i bruk på Sande reinseanlegg i dag.

Vi har valt å dele opp alle varslingane i fire forskjellige grupper:

\begin{itemize}
    \item \textbf{Feil}          (Der systemet ikkje fungera, t.d. sensorfeil og blokkfeil)
    \item \textbf{Alarmar}       (Kritiske prosessparametrar, t.d. straumbrot og veldig høge nivå)
    \item \textbf{Forvarsel}     (prosessparametrar som nærmar seg kritisk nivå)
    \item \textbf{Informasjon}   (Prosessinformasjon med nytteverdi)
\end{itemize}

Ved å kategorisere varslingar i fire ulike grupper med forskjellige prioriteringsnivå,
vil systemet og driftspersonalet enklare kunne forstå samanhengen og alvoret i varslingane. \newline
Korleis dei ulike alarmane blir definert og prioriterte må evaluerast i samråd med \gls{Sunnfjord Kommune}.




\begin{figure}[htbp]
    \centering
    \includegraphics[width=0.4\textwidth]{Bilder/Alarmeksempel.png}
    \caption{Eksempel på alarmlogg}\label{fig:Alarmlogg}
\end{figure}

\newpage


	\section{Utfordringer}
\thispagestyle{fancy}

\subsection{Blokker}
Vi opplevde noko utfordringar rundt det å ha fleire blokker som styrer den same komponenten, samt det å  
skriva til ein felles global variabel.
Denne variabelen kan då bli skriven ``true'' eller ``false'' i frå fleire plassar i programmet, noko som gjer at 
variablenen sin tilstand vil vere tilfeldig basert på korleis kompilatoren til \gls{Codesys} les koden.

Vi møtte denne utfordringa fleire plassar i programmet, t.d. i pumpestyringa,
der kvar reaktor skulle kunne styre den same pumpa.
Dette kunne vi løyse ved å bruke ein unik global variabel for kvar blokk, 
og deretter skrive til ei funksjonsblokk som styrer den globale utgangsvariabelen som går til komponenten.\newline
For pumpestyringa laga vi ei blokk som tok inngangen frå begge pumpene og satt utgangen til riktig tilstand.  

\subsection{Mangel på sensorikk}

Anlegget har ei avgremsa mengd med sensorikk, 
noko som har ført til at vi må berekne, estimere og programmere rundt mangelen på sensorar.

Som døme har anlegget ein funksjon for å aktivere
høgbelastningsmodus ved høg tilstrøyming av veske til annlegget, 
men anlegget har ingen form for strøymningsmålar.\newline
Difor har vi vore nøydde til å kalkulere ein teoretisk tilstrøymingsverdi basert på auken av volumet i mottakstanken.

Sjølv om slike løysningar kan fungere, er det ikkje optimalt for anleggets drift.
Estimering av prosessverdiar vil redusere nøyaktigheita i styresystemet og fører til at
mykje unødvendig prosessorkraft vil bli brukt på berekningar som enkelt kunne vore erstatta av ein sensor.\newline
Vidare har det også krevd mykje ekstra tid å programmere desse funksjonane.



% Trenger dette stå her?
%Anlegget har ikkje nokon form for gjennomstrøymingsmålar 
%og har difor ein teoretisk utrekna strøymingsverdiar. 
%Verdiane er utrekna frå tankvolum, og det gjelder for mottakstank og reaktortank. 
%Dette kan vere noko uheldig da dette gir oss noko manglande nøyaktigheit i målingane våra, 
%som forplantar seg vidare til databehandlinga.
%Vi har tatt utgangspunkt i planteikningane for å få så nøyaktige mål som mogleg, 
%i tillegg til at vi har fått verifisert tank (mottakstank) 
%innvending med video for å konstatere at våra mål og utrekningar er korrekte.




	% Kapittel 9 Dokumentasjon
	\chapter{Dokumentasjon}
\thispagestyle{fancy}

I detta kapittelet blir det lagt fram en oversikt over våra dokumentasjon for arbeidet vi har gjort. 
Sentralt i denne prosessen ligger våra programmering
	\section{Dokumentasjon av funksjonsblokker}
\thispagestyle{fancy}

Alle funskjonar og funksjonsblokker som er utarbeida i programmet, har sitt eige\newline dokumentasjonsdokument.
Dette dokumentet beskriv blant anna bruken, funksjonaliteten, inngangar, utgangar og parametrar.
Dokumentet inneheld også versjonshistorikk som skal oppdatterast ved endring. I tillegg omfattar det lister over omgrep, forkortingar og 
programmeringsbiblioteker som er nødvendige for å ta blokkene i bruk.
Dokumentet er utvikla etter ein eigen mal og er standarisert for alle blokkene som er inkludert i programmet.

Dokumentasjon til alle funksjonar og funksjonsblokker er tilgjengeleg via (vedlegg B)
	\section{Forrigling}
\thispagestyle{fancy}

Det er nytta «interlock» av funksjonar nokre plassar i programmet. 
Sidan vi brukar IEC blokkar til styring av alle våra objekt, har vi funksjonalitet i blokkene som gjer det enkelt for oss å handheve forriglingar mellom komponentar. I tillegg har vi overordna kontroll med at tilstandsmaskina gir signal til XE, slik at dei komponentane som ikkje er i bruk i sekvensen, skal ikkje kunne vere tilgjengeleg.

Objekter som krever forriglingar:

Sett inn tabell/bilde over forriglinger, kva forigla mot, korleis.

Forrigling mellom innpumping og pause, slik at ein reaktor kun kan vere i innpumping samtidig.

Naturlig forrigling med tilstandsmaskin.

--SKRIVE OM INTERLOCK AV INNPUMPINGSEKVENSEN


	\section{Tilstandsovergangsbetingelsar}
\thispagestyle{fancy}

For at tilstandsmaskina skal få lov til å endre tilstand, må nokre spesefikke betingelsar vere oppfylt.
Fleire av betingelsane er tidsstyrte, og alle desse tidene er justerbare via parameter.\newline
For å kunne gi ei klarare framstilling av logikken har vi valgt å presentere betingelsane grafisk.
Vi har utarbeidd eit skjema som dokumenterer alle overgangsbetingelsane i tilstandsmaskina.\newline \newline

\begin{figure}[htbp]
    \centering
    \includegraphics[scale=0.5]{Figurar/Tilstandsovergang.drawio.png}
    \caption{Grafisk presentasjon av overgangsbetingelsar}\label{fig:Tilstandsovergangsbetingelsar}
\end{figure}

\newpage		
	\section{IO-liste}
\thispagestyle{fancy}

Når det gjelder \gls{IO} liste, så har vi ikkje noko konkret å visa til, enn den gamle lista sidan våra løysningsforslag baserer seg på ein teoretisk løysning.
Vi har difor tatt utgangspunkt i ein minimum I/O som allereie ligger til stades i frå det tidlegare anlegget (sjå Appendix~\ref{sec:IOliste}). 

\includepdf[pages=1, scale=0.75, pagecommand={\thispagestyle{empty}}]{Appendix/IOliste.pdf}
\includepdf[pages=2, scale=0.75, pagecommand={\thispagestyle{empty}}]{Appendix/IOliste.pdf}



%I tillegg så har vi laget til moglegheit for fleire inngangar basert på ønsket tilleggsmål gitt av arbeidsgivar. 
%Dette er tilrettelagt for til dømes tilbakemelding av ventilar, flowmåler og temperaturgivera.
%Dette er funksjonalitet som er veldig enkle å legge til i ettertid, da programmet er bygget opp med dette i tankane.

(Vi har den gamle IO, men den nye er ikkje konstruet da vi ikkje har ein fysisk pls å planlegge mot.)

\newpage

	\section{Objektliste}
\thispagestyle{fancy}

Vi måtte danne oss eit bilde over alle komponentane som eksisterte og var tilkopla styresystmet. Dette for å gje oss ei o
oversikt over komponentane vi skulle styre, meg også kva \gls{IEC} blokker \ref{IEC Seksjon} vi trengde og programmere.

\begin{figure}[htbp]
    \centering
    \includegraphics[scale=0.6, page=1]{Appendix/Objektliste_midlertidig.pdf}
    \caption{Objektliste}
    \label{fig:Objektliste}
\end{figure}

\newpage




% Place the PDF as a full page


% Manually add a caption on the next page or in a suitable position
\begin{tikzpicture}[remember picture, overlay]
    % Include the PDF page rotated, positioned at the center of the page
    \node[inner sep=0pt] at (current page.center) {
        \includegraphics[angle=90, width=\paperwidth, keepaspectratio]{Bilder/SCD.pdf}
    };

    % Place the caption at the bottom of the page
    \node[anchor=south, yshift=10mm] at (current page.south) { % Adjust yshift to position the caption
        \begin{minipage}{\textwidth}
            \centering
            \captionof{figure}{SCD innpumpingssekvens}
            \label{fig:SCD}
        \end{minipage}
    };
\end{tikzpicture}


\newpage
\includepdf[pages=1, scale=0.75, pagecommand={\section{Objektliste}\thispagestyle{empty}}]{Appendix/Objektliste_midlertidig.pdf}

\newpage

	\newpage
\section{SCD}
\thispagestyle{fancy}

\gls{SCD} utgjer grunnlaget for det meste av programmeringsdokumentasjonen i oppgåva.
Diagrammet er delt opp sidevis og sekvensvis og viser styring mellom program og komponentar i kvar enkelt sekvens,
samt ei side for resterande fellesstyring.

Prosessen i \gls{SCD}en (blå linjer) er basert på \gls{PID} som blei laga etter gjennomgangen av anlegget i kapittel \ref{sec:6}.
\gls{SCD} tar omsyn til programmerbart utstyr og har derfor ikkje med manuelle ventilar, tilbakeslagsventilar eller liknande.

For å representere eigne funksjonsblokker var vi nøydd å leggje dei inn i \gls{SCD}en.
\gls{SCD} verktøyet hadde moglegheit for å leggje til eigendefinerte blokker med respektive inngangar og utgangar.

Funksjonsblokker og \gls{IEC} funksjonstemplata nytta i \gls{SCD} er dei same som vi har laga og nytta i programmeringsdelen. 
Dei stipla linjene viser koplingar mellom blokker og komponentar. \newline
I enkelte diagram førte antal linjer til at diagrammet vart vanskeleg å lese. 
Vi har derfor nytta koplingar med unik ID, liknande som i CFC vindauget.

Tilstanslogikkblokker har fått eigne forkortingar i \gls{SCD}en.
Dette er dei relevante forkortingane for å forstå blokkene og diagrammet.

\begin{itemize}
    \item \textbf{SM}:   Tilstandsmaskin
    \item \textbf{FBP}:  Funksjonsblokk pause
    \item \textbf{FBI}:  Funksjonsblokk innpumping
    \item \textbf{FBR}:  Funksjonsblokk reaksjon
    \item \textbf{FBS}:  Funksjonsblokk sedimentering
    \item \textbf{FBU}:  Funksjonsblokk uttapping
    \item \textbf{FBPH}: Funksjonsblokk pumpehus
\end{itemize}

Heile \gls{SCD} er tilgjengelig i vedlegg. (Vedlegg H). \newline
\newpage

\begin{tikzpicture}[remember picture, overlay]
    % Include the PDF page rotated, positioned at the center of the page
    \node[inner sep=0pt] at (current page.center) {
        \includegraphics[page=2,angle=90, scale=1, keepaspectratio]{Bilder/SCD.pdf}
    };

    % Place the caption at the bottom of the page
    \node[anchor=south, yshift=10mm] at (current page.south) { % Adjust yshift to position the caption
        \begin{minipage}{\textwidth}
            \centering
            \captionof{figure}{SCD av innpumpingssekvens}
            \label{fig:SCD}
        \end{minipage}
    };
\end{tikzpicture}


	% Kapittel 10 Simulering og verifisering
	\chapter{Simulering og verifisering}
\thispagestyle{fancy}

Simulering er ein sentral i utvikling av eit styresystem.
Det gir oss moglegheita til å verifisere at programmet møter kravspesifikasjon og har ønska effekt.
Testing og simulering bidreg til å redusere risiko for feil, forbetre kvalitet og auke pålitelegheita til programmet.



	\section{Kontinuerlig simulering}
\thispagestyle{fancy}

Undervegs i programmeringa har vi utført kontinuerlege testar og simuleringar av blokkene vi har utvikla, noko
som har våre ein viktig del av arbeidsmetodane vi har brukt i oppgåva. 

Kontinuerleg simulering er små testar som blir utførte medan ein programmerar.
Det er ikkje ein isolert testfase med klare start- og stoppunkt, men heller små kontinuerlege testar og verifiseringar som gir oss ein indikasjon
på om arbeidet følgjer spesifikasjon.

I denne oppgåva nytta vi kontinuerleg simulering i programmeringsfasen av \gls{IEC} blokkene.
Desse blokkene hadde mange funksjonar som ikkje nødvendigvis var avhengige av kvarandre,
som gjorde det mogleg å gjennomføra små, enkle testar på dei ulike områda utan at heile blokka var ferdigstilt.\newline
Som eit døme, skal \gls{RX} resette ein utgang \gls{Y}. Dette blir implementert og deretter testa ved hjelp av simulering.
Kva som til slutt skal sette utgangen \gls{Y} høg er ikkje relevant i dette tilfellet. Vi har likevel implementert 
og testa at ein reset vil fungere.


\begin{figure}[htbp]
    \centering
    \includegraphics[width=0.8\textwidth]{Bilder/kontinuerligSimulering.png}
    \caption{Kontinuerleg testing ved manipulasjon av verdiar}\label{fig:KontinuerlegSimulering}
\end{figure}


\newpage

	\section{Simuleringsblokker}
\thispagestyle{fancy}

For å skape eit meir realistisk miljø for testinga lagde vi nokre simuleringsblokker.
Desse blokkene etterliknar driftssituasjonar og gjer simuleringa enklare og meir effektiv.
Vi lagde hovudsakleg to slike blokker; ei simulerer fylling av ein tank
og ei for tilbakemelding frå ventilar.

Blokk for tanksimulering er enkel og skriven spesifikt for Sande reinseanlegg, som omfattar ein mottakstank
og to reaktorar.
Det er lite truleg at desse simuleringsblokkene vil bli brukt vidare, men dei gav
oss den realismen vi trengte.

Desse blokkene forenkla også sjølve arbeidet med simuleringa.
Før vi utvikla blokka for tilbakemelding på ventilar, 
gav vi manuelt kvar ventil \gls{XGH} eller \gls{XGL} basert på den faktiske stillinga (open/stengd).
Dette fungerer greit dersom ein testar ei ventilblokk, men
når simuleringa omfattar større delar av programmet blir denne jobben tungvindt og tidkrevjande.

På desse blokkene har vi tatt oss meir friheit i namngiving, feilhandtering osv.,
fordi blokkene ikkje skal brukast når programmet er ferdigstilt.


\begin{figure}[htbp]
    \centering
    \begin{subfigure}[b]{0.3\textwidth}
        \centering
        \includegraphics[width=0.9\textwidth]{Figurar/TankSim.png}
        \caption{Tank}\label{fig:TankSim}
    \end{subfigure}
    \hfill
    \begin{subfigure}[b]{0.3\textwidth}
        \centering
        \includegraphics[width=0.9\textwidth]{Figurar/ValveSim.png}
        \caption{Ventil}\label{fig:ValveSim}
    \end{subfigure}
    \caption{Simuleringsblokker}\label{fig:SimuleringsBlokker}
\end{figure}

\newpage
	\section{Simuleringsvindu}
\thispagestyle{fancy}

Når ein tester og simulerer eit prosjekt kan det bli mange variablar og komponentar
og halde styr på. Derfor er det viktig at ein sorterer og behalder den informasjonen ein
ønsker og klarer å representere denne på ein god måte.

Codesys Visualization er eit grafisk verktøy der ein kan representere informasjon
ved hjelp av grafiske elementer. Dette gav oss mulighet til presentere tankar, ventilar og pumper
ved hjelp av bilder og lys. 
Dette gjorde det mykje lettare for oss og oppdage eventuelle feil eller fastslå at ting fungerte.

Vi brukte visualiseringsverktøyet til å lage eit fullskala simuleringsvindu der kvar komponent
var knytt til sitt spesefikke grafiske symbol. Dette vinduet brukte vi ilag med simuleringa
av programmet og knytta dei relevante inngangssignala mot knapper. 
Dette gjorde at vi til slutt hadde eit grensesnitt mot programmet og enkelt kunne simulere
forskjellige driftssitasjonar og enkelt konkludere med resultat.

Simuleringsvinduet er ikkje tiltenkt å være noko 
\gls{HMI} og tar ikkje hensys til aktuelle normer og krav

\begin{figure}[htbp]
    \centering
    \includegraphics[width=0.8\textwidth]{Bilder/Codesys symbol.png}
    \caption{Eksempel Codesys visualisering}\label{fig:reaktorsoner}
\end{figure}

\newpage

	\section{Full simulasjon}
\thispagestyle{fancy}
\begin{figure}[htbp]
    \centering
    \includegraphics[width=1\textwidth]{Bilder/Simuleringsbilde.png}
    \caption{Oppsett av simuleringsvindu}\label{fig:Simulering}
\end{figure}
\subsection{Oppsett}
Med simulasjonsblokkene klare og eit simuleringsvindu og vise resultat på var vi
no klare for å sette opp ein fullstending simulasjon av anlegget.
Vi satte inn ventil tilbakemeldingsblokka på alle ventilar i programmet og oppretta eit
nytt CFC vindu som kopla tank simuleringa av mottakstanken og reaktorane ilag med resten av
programmet.

Simulasjonen er satt opp slik at tilstrøymning til anlegget kan stillast av ein glidar.
Dersom ein av reaktorane er i innpumpings sekvens simulerast vi flytting av avlaupsvatn
basert på pumpekurver og løftehøgder.

Reaktortilstand er representert ved ein teksvariabel og
dei aktuelle tidsparameterane, som eksempelvis tid på reaksjonssekvens,
er redusert for å akselerere simuleringen.

Vi har også lagt til ein alarmlogg for alle moglege alarmar anlegget kan generere, 
uavhengig om dei er aktuelle å ha med i sluttprogrammet eller ikkje.
Dette gjer oss verdifull innsikt i korleis programmet fungera.



\subsection{Resultat}


Før simuleringa starta vart programmet satt opp for å bli simulert. 
Dette betyr at vi i første simuleringsfase, ikkje simulerer ein fullskala test, men at vi skalerer ned volumet i tankane 
for å få ned tida det tar å simulere ein full syklus av begge reaktorane. 
Som forventa dukk det opp nokre avvik når vi begynte å simulere. 
Nokre av desse var av mindre betydning og utbetra  undervegs, som for eksempel at våra forrigling av reaktortankane ikkje skal 
kunne gå i innpumpingstilstand samstundes, ikkje fungerte.

Våra fullskala simulering var veldig vellykka og vi fekk samla masse verdifull data av anlegget i simulert drift. 
Den visuelle representasjonen var oversiktleg og ein kunne lett sjå koden i drift og visuelt forstå kva koden våra gjorde.  
Vi fekk verifisert at våra kode som ligger til grunn verkar som tiltenkt, og vi vi vidare er klar for ein større simulering, 
der vi tar med dei funksjonane som enda ikkje er implementert, og til slutt ein fullskala test.




Her bør det stå litt av resultata av simulering? \newline 
Kanskje eksempel på litt kva vi såg osv.\newline


	% Kapittel 11 Diskusjon
	\chapter{Diskusjon}
\thispagestyle{fancy}
DISKUTERE
I slutten av april avslutta vi programeringsfasen våra, og starta rapportfasen.

KRAV OG MÅLOPPNÅING
DELMÅL

1. fsd 
2. Programmering
3. Verifisering og simuering.

Tillegsmål!
	\section{Vegen videre for anlegget}
\thispagestyle{fancy}

Sjølv om bacheloroppgåva vår avsluttast har vi alle fått ein ny fasinasjon av ein vekkgøymd sektor innan offentleg infrastruktur. 
Vi ynskjer derfor å diskuere litt rundt vegen vidare for anlegget og eventuelle oppgraderingar som kan gjerast
for å optimalisere prosessen.

\subsection{Ombygging}

Dersom den teoretiske bacheloroppgåva vår skal realiserast i praksis trenger anlegget fysiske oppgraderingar.
Reinseanlegget idag tilfredsstiller ikkje fleire relevante lover og forskrifter rundt industri og offentleg infrastruktur.
Mykje av denne delen er utanfor vårt fokusområde men reinseanlegget har foreksempel ingen handtering eller moglegheitar for naudstopp.
Vi anser det som viktig at koden og programmet ikkje settast i drift dersom anlegget ikkje gjennomgår ein grundig sikkerheitsanalyse.
Vi forventer då ei større ombygging og fleire aspekter av anlegget kan oppgraderast i eit slikt tilfelle. \newline \newline

\begin{figure}[htbp]
    \centering
    \includegraphics[width=1\textwidth]{Bilder/SandeGjennomgang.jpg}
    \caption{Bilete frå første gjennomgangen av anlegget}\label{fig:Bilete Gjennomgang}
\end{figure}

\begin{center}
    \textit{Foto: Håvar Dankel}
\end{center}

\newpage

\subsection{Anbefalingar sensorikk}

For å betre anlegget og styresystemet ytterlegare vil det være nyttig å få inn meir instrumentering. 
Auka sensorikk vil forbetre styresystemet ved å samle meir og nøyaktig data om tilstanden og ytelsen til systemet. 
Med meir data vil styresystemet bli meir automatisk og vil kunne tilpassast varierande forhold meir effektivt. 
Systemet vil og ha moglegheita til å oppdage avvik og agere tidlegare utan manuelle inngrep, noko som igjen aukar automasjonsevna.

Basert på vår nyerfarte kunnskap av annlegget og styresystemet presenterer vi ein
anbefaling på oppgradering av instrumentering som ville forbetre reinseanlegget. 
Sjølv om all sensorikk vil være verdifull, har vi vurdert kost-nytte og anbefaler kun sensorikk som vil gi tilstrekkeleg verdi.

\begin{itemize}
    \item \textbf{Strøymingsmålar} \newline
        Strøymingmålar i anlegget vil gi nøyaktige tall på mengder vatn som er i anlegget.
        Dette vil gi bedre kontroll på aktivering av høgbelastningsmodus og kontroll og rapportering av driftsdata.
        Fleire strøymningsmålarar er moglege men minimumsanbefaling er vatn inn og ut av anlegget.
    \item \textbf{Energimåling} \newline
        Måling av energi vil gi moglegeit å analyse energiforbruket for å redusere kostnadar og effektivisere prosessar.
        Komponentspesefikk energimåling kan også brukast til overvåking av utstyr for å oppdage slitasje og feil.
    \item \textbf{Reaktormålingar} \newline
        Oksygen, PH og temperatur er alle kritiske verdiar for å oppnå god biologisk reinsing i reaktorane. \newline
        Ved å ha kontroll på desse parameterane vil reaktoren kunne finjusterast for å effektivitere den biologiske reinseprosessen.
    \item \textbf{Tilbakemeldingar} \newline
        Anlegget har begrensa tilbakemeldingar på utstyr, spesielt ingen innan ventilstyring.
        Tilbakemelding er essensielt for prosesstyring, feiloppdagelse og sikkerheit.
        Utan tilbakemeldingar er reinseanlegget meir sårbar for feil som ikkje blir oppdaga før det er for seint 
        og slike situasjonar vil kunne resultere i nedetid, overlaup og andre utilsikta hendelser.
    \item \textbf{Forbetra nivåmåling} \newline
        Ei forbetring av dei akutelle nivåmålingane i anlegget vil gi meir nøyaktige mål på slammengde og slamnivå i reaktorane.
        Det vil då være mogleg å finne det eksakte skilljet mellom slam og reinsa vatn etter ein sedimenteringssekvens som igjen
        gir bedre oversikt over tilstanden til den biologiske reinseprosessen.
    \item \textbf{Oppløysning} \newline
        Oppløysning på analoge målingar er idag kunn 0-1000 bit på 4-20mA.(Sjå datablad appendix xXx) \newline
        Oppgradering av måleoppløysning vil gjere kvar eksisterande og nye analoge målingar
        meir nøyaktig som igjen gir moglegheit for betre styring og regulering.
\end{itemize}
\newpage


	\section{Vegen vidare for programmet}
\thispagestyle{fancy}

Programmeringsarbeidet hadde eit så stort omfang at vi ikkje klarte å ferdigstille prosjektet.
Etter å ha innsett at vi ikkje ville nå målet, måtte vi gjere nokre val angåande prioritering av ressursar. 
Vi valde å fokusere på å gjennomføre ei god samla simulering i staden for å arbeide vidare med ferdigstilling av funksjonar.

\subsection{Programmering}

Vegen vidare for programmet er ferdigstilling av funksjonane som ikkje er komplette, og 
implementering av funksjonar som er oppretta men ikkje tatt i bruk. 
Det gjennstår arbeid på feilhandtering og korleis anlegget skal reagere på spesifikke feilsituasjonar.

Her er punkter vi ønskjer å arbeide vidare med.

\begin{itemize}
    \item Styring av pumpehuset er forberedt men ikkje programmert. 
    \item Ferdigstilling av høgbelastningsmodus.
    \item Spesefikk feilhandtering
\end{itemize}

\subsection{Simulering}

Det gjennstår framleis noko simulering og testing før ein kan ansjå programmet ferdig verifisert.
Til dømes er funksjonsblokkene for slamuttapping og sivbedrotasjon testa separat, men ikkje simulert saman med
tilstandsmaskinene og resterande program.

Under testing av eit slikt program bør ein aktivt leite og prøve å identifisere feil og manglar for å gjere programmet best mogleg.
For blokkparameterer manglar eksempelvis nokre beskyttelsar mot ulovlege parameterverdiar.

Ved å investere meir tid i vidare simuleringsfase sikrar vi at programmet oppfyller forventningane til \gls{Sunnfjord Kommune} og at ein eventuell
praktisk innkøyringsfase går effektivt. \newline
Programmet er no sett opp i simuleringsmodus med simuleringsblokker og endra parameterverdiar, 
og det må tilbakestillast til original tilstand før det kan nyttast. 




\newpage
	\section{Måloppnåing}
\thispagestyle{fancy}

\subsection{Krav}
Når ein tar eit tilbakeblikk på kravspesifikasjonen i kapittel 5 sit vi igjen med ein god følelse.
Krava som vart godkjent ilag med oppdragsgivar var relativt opne og 
mykje av grunnen til dette er at oppgåva ikkje var utlyst til oss, men at vi etterlyste ei oppgåve ifrå Renasys.
Oppgåva hadde dermed ingen ferdigdefinerte krav eller mållinjer noko som gjorde at vi kunne definere oppgåvå slik vi ville.
Sjølv om krava var opne føler vi at vi har svart på dei punkta som var essensielle og levert det som var forventa.

Det er også eit faktum at vi ikkje kom i mål med alt arbeidet og at enkelte detaljer av både dokumentasjon, program og simulering
kunne blitt forbetra og utvida, men med den tida vi hadde føler vi at vi har levert etter krav.

\subsection{Mål}
Våre personlege mål under oppgåva dreia seg mykje om læring og å utfordre det nye, og under dette prosjektet har vi lært mykje nytt. 
Programmeringsdelen av oppgåva har bydd på opplæring i nye programmer og pugging og lesing av normer og standarar.
Alle delane av denne bacheloroppgåva har bydd på utfordringar og meistring og når vi ser tilbake på dei personlege måla våre er vi veldig fornøyde.

\subsection{Forprosjekt}
Ein viktig del av slutten på eit prosjekt er å sjå tilbake på starten.
Klarte vi å legge ein god plan og klarte vi å følge planen vi la?
Dette er sentrale spørsmål ein er nødt å ta stilling til dersom ein ynskjer å lære mest mogeleg.

Når vi samanliknar forprosjektet og bacheloroppgåva, ser vi ein god gjenspegling av kvarandre. 
I løpet av arbeidsprosessen og etterkvart som me fekk meir kunnskap, vart det naturleg å gjere nokre justeringar og tilpassingar. 
Likevel kjenner vi at forprosjektet la ein solid plan for oss, og at me har følgt denne planen i utviklinga av bacheloroppgåva


\newpage
	
	% Kapittel 12 Konklusjon
	\chapter{Konklusjon}
\thispagestyle{fancy}
Må skrivast i fortid, Konklusjon vs Avslutting?

Fra chatgpt
Oppsummer hovedfunnene: Gjennomgå kort de viktigste funnene og resultatene av studien din.

Relater til problemstillingen: Diskuter hvordan funnene svarer på problemstillingen eller forskningsspørsmålene 
som ble presentert i innledningen.

Diskuter implikasjoner: Vurder hvilke implikasjoner dine funn kan ha for fagfeltet ditt eller relevante praksisområder. 
Hvordan kan resultatene påvirke fremtidig forskning eller handling?

Styrker og svakheter: Reflekter over styrker og begrensninger ved studien din. 
Hvilke faktorer kan ha påvirket resultatene,
----

sokogskriv.no - tips
Pek på nye problemstillinger som du har kommet over gjennom arbeidet
Ta fatt i ubesvarte spørsmål fra ditt eget prosjekt, og pek på mulig oppfølging og nye, potensielle prosjekter
Konklusjonen følger av det som er sagt
Oppgaven «biter seg selv i halen»

https://www.superprof.no/blog/bachelor-konklusjon/

	\clearpage
	\renewcommand{\glossaryname}{Ordliste} % Denne er med for å kunne endre navn på ordlista
	\printglossaries %Printer ut alle ord som er brukt frå ordlista

	%Referanseliste (Berre for test, citep må flyttast inn i teksten)
	%\clearpage
	%\chapter{Referansar} % sjekk om det skal være slik, for å få den opp i innhaldslista.
	\clearpage
	\bibliography{Referansar} % Dette er ein referanse til Referansar.bib
	\addcontentsline{toc}{chapter}{Referansar}  % Adds the unnumbered chapter to the table of contents


	% Create a partial table of contents for appendices
	% Generate the list of appendices
	% Start of appendices, capture contents for appendices list
	%\begin{appendices}
	%	% This command initializes the recording of contents for appendices
	%	\startcontents[appendices]
	%	\phantomsection
	%	\addcontentsline{toc}{chapter}{Liste av Appendikser} % Her kan ein endre navn for appendix lista
	%	%\chapter{MB Blokk}
%Dette er ein appendiks for MB blokka
%\includepdf[pages=-]{MB Dokumentasjon.pdf}  

\chapter{IEC Blokker}

% MB Blokk
% Include only the first page
\includepdf[pages=1,scale=0.8, pagecommand={\section{Montor binary}\thispagestyle{empty}},fitpaper=true]{Appendix/IEC Blokker/MB Dokumentasjon.pdf}
% Include the rest of the pages, Starts from the second page to the last, Keep the same scale for all pages
\includepdf[pages=2-, scale=0.8, pagecommand={\thispagestyle{empty}}, fitpaper=true]{Appendix/IEC Blokker/MB Dokumentasjon.pdf}

% MA Blokk
\includepdf[pages=1, scale=0.8, pagecommand={\section{Monitor Analogue}\thispagestyle{empty}}, fitpaper=true ]{Appendix/IEC Blokker/MA Dokumentasjon.pdf}
\includepdf[pages=2-,scale=0.8, pagecommand={\thispagestyle{empty}},fitpaper=true]{Appendix/IEC Blokker/MA Dokumentasjon.pdf}

% SBE Blokk
\includepdf[pages=1, scale=0.8, pagecommand={\section{Switch Binary Eletrical}\thispagestyle{empty}}, fitpaper=true ]{Appendix/IEC Blokker/SBE Dokumentasjon.pdf}
\includepdf[pages=2-,scale=0.8, pagecommand={\thispagestyle{empty}},fitpaper=true]{Appendix/IEC Blokker/SBE Dokumentasjon.pdf}

% SBV Blokk
\includepdf[pages=1, scale=0.8, pagecommand={\section{Switch Binary Value}\thispagestyle{empty}}, fitpaper=true ]{Appendix/IEC Blokker/SBV Dokumentasjon.pdf}
\includepdf[pages=2-,scale=0.8, pagecommand={\thispagestyle{empty}},fitpaper=true]{Appendix/IEC Blokker/SBV Dokumentasjon.pdf}


% ------------ Nytt Kapittel
\chapter{Funksjons Blokker}
% Sekvens Blokker 
% FB Pause
\includepdf[pages=1, scale=0.8, pagecommand={\section{FB Pause Sekvens}\thispagestyle{empty}}, fitpaper=true ]{Appendix/Funksjons Blokker/fbPause.pdf}
\includepdf[pages=2-,scale=0.8, pagecommand={\thispagestyle{empty}},fitpaper=true]{Appendix/Funksjons Blokker/fbPause.pdf}
% FB Innpumping
\includepdf[pages=1, scale=0.8, pagecommand={\section{FB Innpumpings Sekvens}\thispagestyle{empty}}, fitpaper=true ]{Appendix/Funksjons Blokker/fbInnpumping.pdf}
\includepdf[pages=2-,scale=0.8, pagecommand={\thispagestyle{empty}},fitpaper=true]{Appendix/Funksjons Blokker/fbInnpumping.pdf}
% FB Reaksjon
\includepdf[pages=1, scale=0.8, pagecommand={\section{FB Reaksjon Sekvens}\thispagestyle{empty}}, fitpaper=true ]{Appendix/Funksjons Blokker/fbReaksjon Dokumentasjon.pdf}
\includepdf[pages=2-,scale=0.8, pagecommand={\thispagestyle{empty}},fitpaper=true]{Appendix/Funksjons Blokker/fbReaksjon Dokumentasjon.pdf}
% FB Sedimentering
\includepdf[pages=1, scale=0.8, pagecommand={\section{FB Sedimentering Sekvens}\thispagestyle{empty}}, fitpaper=true ]{Appendix/Funksjons Blokker/fbSedementering Dokumentasjon.pdf}
\includepdf[pages=2-,scale=0.8, pagecommand={\thispagestyle{empty}},fitpaper=true]{Appendix/Funksjons Blokker/fbSedementering Dokumentasjon.pdf}
% FB Uttapping
\includepdf[pages=1, scale=0.8, pagecommand={\section{FB Uttapping Sekvens}\thispagestyle{empty}}, fitpaper=true ]{Appendix/Funksjons Blokker/fbUttaping Dokumentasjon.pdf}
\includepdf[pages=2-,scale=0.8, pagecommand={\thispagestyle{empty}},fitpaper=true]{Appendix/Funksjons Blokker/fbUttaping Dokumentasjon.pdf}
% FB Slammuttak
\includepdf[pages=1, scale=0.8, pagecommand={\section{FB Slammuttak}\thispagestyle{empty}}, fitpaper=true ]{Appendix/Funksjons Blokker/fbSlamuttak Dokumentasjon.pdf}
\includepdf[pages=2-,scale=0.8, pagecommand={\thispagestyle{empty}},fitpaper=true]{Appendix/Funksjons Blokker/fbSlamuttak Dokumentasjon.pdf}
% 
% Analog Alarm
\includepdf[pages=1, scale=0.8, pagecommand={\section{FB Analog alarm}\thispagestyle{empty}}, fitpaper=true ]{Appendix/Funksjons Blokker/fbAnalogueAlarm Dokumentasjon.pdf}
\includepdf[pages=2-,scale=0.8, pagecommand={\thispagestyle{empty}},fitpaper=true]{Appendix/Funksjons Blokker/fbAnalogueAlarm Dokumentasjon.pdf}
% Digital Alarm
\includepdf[pages=1, scale=0.8, pagecommand={\section{FB Digital alarm}\thispagestyle{empty}}, fitpaper=true ]{Appendix/Funksjons Blokker/fbDigitalAlarm.pdf}
\includepdf[pages=2-,scale=0.8, pagecommand={\thispagestyle{empty}},fitpaper=true]{Appendix/Funksjons Blokker/fbDigitalAlarm.pdf}
% Calculations
\includepdf[pages=1, scale=0.8, pagecommand={\section{FB Kalkuleringer}\thispagestyle{empty}}, fitpaper=true ]{Appendix/Funksjons Blokker/fbCalculations.pdf}
\includepdf[pages=2-,scale=0.8, pagecommand={\thispagestyle{empty}},fitpaper=true]{Appendix/Funksjons Blokker/fbCalculations.pdf}
% Data Processing
\includepdf[pages=1, scale=0.8, pagecommand={\section{FB Data Prossesering}\thispagestyle{empty}}, fitpaper=true ]{Appendix/Funksjons Blokker/fbDataProcessing.pdf}
\includepdf[pages=2-,scale=0.8, pagecommand={\thispagestyle{empty}},fitpaper=true]{Appendix/Funksjons Blokker/fbDataProcessing.pdf}
% Høgbelastningsmodus
\includepdf[pages=1, scale=0.8, pagecommand={\section{FB High Load}\thispagestyle{empty}}, fitpaper=true ]{Appendix/Funksjons Blokker/fbHighLoad.pdf}
\includepdf[pages=2-,scale=0.8, pagecommand={\thispagestyle{empty}},fitpaper=true]{Appendix/Funksjons Blokker/fbHighLoad.pdf}
% Processed Water
\includepdf[pages=1, scale=0.8, pagecommand={\section{FB Processed Water}\thispagestyle{empty}}, fitpaper=true ]{Appendix/Funksjons Blokker/fbProcessedWater Dokumentasjon.pdf}
\includepdf[pages=2-,scale=0.8, pagecommand={\thispagestyle{empty}},fitpaper=true]{Appendix/Funksjons Blokker/fbProcessedWater Dokumentasjon.pdf}
% Sivbed Rotation
\includepdf[pages=1, scale=0.8, pagecommand={\section{FB Sivbed Rotation}\thispagestyle{empty}}, fitpaper=true ]{Appendix/Funksjons Blokker/fbSivbed Rotation Dokumentasjon.pdf}
\includepdf[pages=2-,scale=0.8, pagecommand={\thispagestyle{empty}},fitpaper=true]{Appendix/Funksjons Blokker/fbSivbed Rotation Dokumentasjon.pdf}
% Swap
\includepdf[pages=1, scale=0.8, pagecommand={\section{FB Swap}\thispagestyle{empty}}, fitpaper=true ]{Appendix/Funksjons Blokker/fbSwap Dokumentasjon.pdf}
\includepdf[pages=2-,scale=0.8, pagecommand={\thispagestyle{empty}},fitpaper=true]{Appendix/Funksjons Blokker/fbSwap Dokumentasjon.pdf}
% Time Meter
\includepdf[pages=1, scale=0.8, pagecommand={\section{FB Time Meter}\thispagestyle{empty}}, fitpaper=true ]{Appendix/Funksjons Blokker/fbTimeMeter Dokumentasjon.pdf}
\includepdf[pages=2-,scale=0.8, pagecommand={\thispagestyle{empty}},fitpaper=true]{Appendix/Funksjons Blokker/fbTimeMeter Dokumentasjon.pdf}
% Timer
\includepdf[pages=1, scale=0.8, pagecommand={\section{FB Timer}\thispagestyle{empty}}, fitpaper=true ]{Appendix/Funksjons Blokker/fbTimer.pdf}
\includepdf[pages=2-,scale=0.8, pagecommand={\thispagestyle{empty}},fitpaper=true]{Appendix/Funksjons Blokker/fbTimer.pdf}

% ----------- Nytt Kapittel
\chapter{Funksjoner}
% Volume Rektangel
\includepdf[pages=1, scale=0.8, pagecommand={\section{FC Volum rektangel}\thispagestyle{empty}}, fitpaper=true ]{Appendix/Funksjoner/Volume_Rectangle.pdf}
\includepdf[pages=2-,scale=0.8, pagecommand={\thispagestyle{empty}},fitpaper=true]{Appendix/Funksjoner/Volume_Rectangle.pdf}
% Volume Sylinder
\includepdf[pages=1, scale=0.8, pagecommand={\section{FC Volum Sylinder}\thispagestyle{empty}}, fitpaper=true ]{Appendix/Funksjoner/Volume_Cylinder.pdf}
\includepdf[pages=2-,scale=0.8, pagecommand={\thispagestyle{empty}},fitpaper=true]{Appendix/Funksjoner/Volume_Cylinder.pdf}

%Importering av IO-liste.csv
\label{sec:IOliste}
\includepdf[pages=1, scale=0.8, pagecommand={\chapter{IO liste}\thispagestyle{empty}}, fitpaper=true ]{Appendix/IO/io.pdf}
\includepdf[pages=2-,scale=0.8, pagecommand={\thispagestyle{empty}},fitpaper=true]{Appendix/IO/io.pdf}




% Litt enklere format
%\includepdf[pages=-, pagecommand={\thispagestyle{empty}}]{Appendix/IEC Blokker/MB Dokumentasjon.pdf}
%\section{MA Blokk}
%\includepdf[pages=-, pagecommand={\thispagestyle{empty}}]{Appendix/IEC Blokker/MA Dokumentasjon.pdf}
	%	% Stop capturing contents and print them
	%	\stopcontents[appendices]
	%\end{appendices}




	

\end{document} % Dokument slutt


