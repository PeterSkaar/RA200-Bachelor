\documentclass[11pt, a4paper]{report}
%-----------NORSK LaTeX--------------

% Denne pakken inkluderer skandinaviske bokstaver( æ, ø og å)
\usepackage[utf8]{inputenc}

% Denne pakken bruker norsk typesetting (innholdsliste osv.)
\usepackage[norsk]{babel}

%--------DOKUMENTOPPSETT------------

% Lar deg tilpasse layouten i prosjektet
\usepackage{fancyhdr}

% Definere egne farger 
\usepackage[usenames, dvipsnames]{color}

% Avstand på marger til sidene på arket
%\usepackage{geometry}

% Linker i dokumentet (figurer, seksjoner osv)
\usepackage[hidelinks]{hyperref}

% Kildeliste-setup
\usepackage[natbibapa]{apacite}

% Spacing mellom bokstaver
\usepackage{microtype}

% Laster inn forskjellige skrifttyper
\usepackage{uarial} % Arial
\usepackage[scaled]{helvet} % Helvetica
%\usepackage{tgbonum} % Usikker
%\usepackage{mathptmx} % Times New Roman
%\usepackage{lmodern} % Latin Modern
%\usepackage[charter]{mathdesign} % Charters
\usepackage[T1]{fontenc}

% Avsnitt mellom paragrafer
\usepackage{parskip}

% Linjeavstand i dokumentet
\usepackage{setspace}
\setlength{\parskip}{1em}

% Fet skrift på alle figurnr
\usepackage[labelfont=bf]{caption}
\usepackage{capt-of}
\usepackage{caption}

% ------------------------------------- KOMMENTERT UT FOR TEST
% Nummererer figurer og tabeller etter kapittel, delkapittel osv.
%\counterwithin{figure}{section}
%\counterwithin{figure}{subsection}
%\counterwithin{figure}{subsubsection}

%\counterwithin{table}{section}
%\counterwithin{table}{subsection}
%\counterwithin{table}{subsubsection}

%\renewcommand{\thefigure}{
%	\ifnum\value{subsection}=0{\thesection-\arabic{figure}}
%	\else\ifnum\value{subsubsection}=0{\thesubsection-\arabic{figure}}
%	\else{\thesubsubsection-\arabic{figure}}\fi}
%
%\renewcommand{\thetable}{
%	\ifnum\value{subsection}=0{\thesection-\arabic{table}}
%	\else\ifnum\value{subsubsection}=0{\thesubsection-\arabic{table}}
%	\else{\thesubsubsection-\arabic{table}}\fi}
%--------------------------------------------------------------------
\usepackage{chngcntr}
\counterwithin{figure}{chapter}
\counterwithin{table}{chapter}

% Definerer mellomrom mellom seksjons-, figur- og tabellnummer i innholdsliste til teksten begynner
\renewcommand{\numberline}[1]{#1 \,}

% Kan skrive i kolonner
\usepackage{multicol}


%----------ANDRE PAKKER--------------

% Pakker fra AMS. Kan skrive matteformler
\usepackage{amsfonts, amssymb, amsthm}
\usepackage{mathtools}

% For grafikk (lime inn bilder, pdf-dokumenter)
\usepackage{graphicx}
\usepackage{float}
\usepackage{pdfpages}
\usepackage{subcaption}

% Generere tekst
\usepackage{kantlipsum}

% Definere egne objekt
\usepackage{tikz}

% Lar deg lage lister på en mer oversiktlig måte
\usepackage[sharp]{easylist}

% Boks rundt tekst
\usepackage{fancyvrb}

% Test pakker
\usepackage{array}
\usepackage[margin=1in]{geometry} % Juster margene etter behov
\usepackage{titlesec}
\usepackage{tabularx}
\usepackage{xcolor}
\usepackage{lastpage}

%---------------------------------------------------

% Velger arial som skrifttype i dokumentet 
%\renewcommand{\rmdefault}{qpl} % 
%\renewcommand{\rmdefault}{phv} % Arial
\renewcommand*\familydefault{\sfdefault} % Helvetica(sans-serif)


% Definerer egne farger
\definecolor{mybla}{RGB}{25, 49, 90}
\definecolor{mylysbla}{RGB}{0, 136, 176}
\definecolor{renasysGreen}{RGB}{4, 255, 4}


% Velger norsk språk på innholdsfortegnelse, figurliste og tabelliste.
% Legger disse så til i innholdsfortegnelsen.
\addto\captionsnorsk{\renewcommand{\contentsname}{Innholdsliste}}
\addto\captionsnorsk{\renewcommand{\listfigurename}{Figurliste}}
\addto\captionsnorsk{\renewcommand{\listtablename}{Tabelliste}}
% Nynorsk!
\addto\captionsnynorsk{\renewcommand{\contentsname}{Innhaldsfortegnelse}} 
\addto\captionsnynorsk{\renewcommand{\listfigurename}{Figurliste}}
\addto\captionsnynorsk{\renewcommand{\listtablename}{Tabelliste}}


% ----- Justering av avstander mellom chapter / section / subsection / subsubsection

% Justeringa av chapter, 
% første verdi : Horisontalverdi frå venstre marg
% Andre verdi : avstanden frå toppen av sida til tittelen
% Tredje verdi : Avstand til neste tekst
\titleformat{\chapter}[hang]
  {\normalfont\bfseries\Large}{\thechapter}{1em}{}
\titlespacing*{\chapter}{0pt}{-50pt}{20pt}

% Juster avstanden for \section
\titleformat{\section}
  {\normalfont\Large\bfseries}{\thesection}{1em}{}
\titlespacing*{\section}{0pt}{10pt}{10pt}

% Juster avstanden for \subsection
\titleformat{\subsection}
  {\normalfont\large\bfseries}{\thesubsection}{1em}{}
\titlespacing*{\subsection}{0pt}{10pt}{10pt}

% Juster avstanden for \subsubsection om nødvendig
\titleformat{\subsubsection}
  {\normalfont\normalsize\bfseries}{\thesubsubsection}{1em}{}
\titlespacing*{\subsubsection}{0pt}{10pt}{10pt}

% -------------------------------------------------------------------------------

%---------------------------------------------------------------------
% Starter dokumentet

% Define the page style - Her lager Peter til OP mal
\fancypagestyle{fancy}{
  % Clear the header and footer
  \fancyhf{}
  % Set the right side of the footer to be the location of the page number
  \fancyfoot[R]{Side \thepage\ av~\pageref{LastPage}} % Her sett ein fra side til side
  \renewcommand{\headrulewidth}{0pt} % Remove header line
  \renewcommand{\footrulewidth}{0pt} % Remove footer line
}



\begin{document}

	% Turn on the style
	\pagestyle{fancy}

	%------------------FREMSIDE------------------------

	% Linker inn fremsiden fra adms (administrative sider) fremside-filen.
	% Førstesiden skal ha følgende marger.
\newgeometry{top=2.5cm,bottom=2.5cm,right=3cm,left=5cm}

% Ønsker ikke sidetall på denne siden.
\thispagestyle{empty}


% Legger inn banner
\begin{center}
	\begin{tikzpicture}[remember picture, overlay]
	\node [xshift=0cm, yshift=11.5cm] at (current page.center) 
	{\includegraphics[width=\paperwidth]{Bilder/frontpage.png}};
	\end{tikzpicture}
\end{center}



\vspace{6cm}

{\fontsize{36}{50}\selectfont \textbf{\textcolor{mybla}{Programmering PLC}}}


\vspace{1cm}
% Forfattarar
{\fontsize{26}{30}\selectfont \textbf{\textcolor{mylysbla}{Roar Bøyum}}}\\
{\fontsize{26}{30}\selectfont \textbf{\textcolor{mylysbla}{Vegard Aven Ullebø}}}\\
{\fontsize{26}{30}\selectfont \textbf{\textcolor{mylysbla}{Peter Søreide Skaar}}}\\ 


\vspace{1.3cm}


\setstretch{0.7}

{\fontsize{20}{20}\selectfont\textcolor{mylysbla}{Name of Masterprogram}} 

\vspace{-0.4cm}
{\fontsize{20}{20}\selectfont\textcolor{mylysbla}{Department/Institute/Program}} 

\vspace{-0.4cm}
{\fontsize{20}{20}\selectfont\textcolor{mylysbla}{Supervisor (in agreement with supervisor)}} 

\vspace{-0.4cm}
{\fontsize{20}{20}\selectfont\textcolor{mylysbla}{Submission Date}} 



% Endrer skriften for resten av dokumentet
{\fontfamily{qpl}\selectfont}


\vspace{1cm}

\setstretch{0.8}
{\fontsize{9.5}{20}\selectfont
	I confirm that the work is self-prepared and that references/source references to all sources used in the work are provided, cf.\ \href{https://lovdata.no/dokument/SF/forskrift/2016-12-21-1851#KAPITTEL_10}{Regulation relating to academic studies and examinations at the Western Norway University of Applied Sciences (HVL), §~10}.}
	

	%------------------------------------------------------

	\newpage

	% Nye marger
	\newgeometry{top=2.5cm,bottom=3cm,right=2.2cm,left=2.2cm}

	% Linjeavstand
	\setstretch{1.3}



	% Clear the header and footer
	%\fancyhead{}
	%\fancyfoot{}
	%\fancyfoot[R]{Side \thepage\ av~\pageref{LastPage}} % Right footer
	%\renewcommand{\headrulewidth}{0pt} % Remove header line
	%\renewcommand{\footrulewidth}{0pt} % Remove footer line

	% Set the right side of the footer to be location of the page number



	% Ønsker tallnummerering videre.
	%-------------------- 			FORORD			---------------------	
	\chapter*{Forord}
\thispagestyle{fancy}
Ønskjer å takke, Hansa Mango IPA for ei fattig trøst til ein fattig student.


% Ønsker ikke sidetall på denne siden.
\thispagestyle{empty}

	%--------------------Innhaldsliste / Figurliste / tabell liste---------------------	
	% Innholdsliste
	\tableofcontents % Genererer Innhaldsliste
	\thispagestyle{empty} % Fjerner sidetall
	\listoffigures % Genererer figur liste
	\thispagestyle{empty}
	%\quad
	%\listoftables
	%\thispagestyle{empty}
	%\cleardoublepage


	% Set counter setter sidetallet for innholdsliste-siden.
	\pagenumbering{arabic}
	\setcounter{page}{4}

	%------------------------------------------------------

	%------------------ Lim inn sider til documentet her ------------------------

	% Kapittel 1
	\chapter{Innleiing}
\thispagestyle{fancy}
Rapporten er skrevet for Sunnfjord Kommune via oppdragsgivar Renasys AS.
Oppgåva er fokusert rundt det noverande avløpsreinseanlegget på Sande «RA200». Reinseanlegget har hatt 
problem over lengre tid noko som har gjort at Sunnfjord kommune har sett på forskjellige moglegheiter for å 
forbetre anlegget, spesielt innan styresystemet.
Under arbeidet har bachelorgruppa henta informasjon og spesifikasjonar frå dei forskjellige aktørane for å 
danne eit bra bilete av arbeidet. Rapporten legger grunnlaget for bacheloroppgåva som skal skrivast om same 
tema.

\newpage
\section{Organisering av rapporten}
Rapporten er organisert etter standard HVL-mal.
Hensikta med kapittel ein er og gje lesaren ein betre forståelse for målet og problembeskrivinga. 
Vidare går vi igjennom krav og spesifikasjonar før vi legger fram løysingsalternativ og trekker ein konklusjon til løysning.

\section{Oppdragsgivar}

\subsection{Renasys AS}
Renasys AS er ein startup som arbeider med banebrytande teknologi innan mekanisk finpartikkelfiltrering av avløpsvatn. 
Renasys har gått offentleg med teknologien sin i løpet av 2023 og tilbyr no tenester til kommunar og interkommunale selskap. 
Renasys arbeider mot «Mission Zero» som innebærer null utslipp, null avfall og null energi.

\subsection{Sunnfjord kommune}
Etter kommunereformen i 2020 blei Sunnfjord kommune danna av tidlegare Gaular, Naustdal, Førde, og Jølster kommune. Sunnfjord kommune teknisk drift har ansvar for avfall, veg, vann og avløp i Sunnfjord kommune.
	\section{Problemstilling}
Det noverande reinseanlegget på Sande blei etablert i 2003. Anlegget har hatt problem over lengre tid og 
Sunnfjord kommune har undersøkt moglegheita for å forbetre anlegget. Reinseanlegget er teknisk utdatert og 
er avhengig av modernisering, spesielt innan styresystemet.
Styresystemet er over tjue år gammalt og består stort sett av utdaterte komponentar, 
kommunikasjonsprotokoller og programmeringslogikk. Dette gjer oppgradering av anlegget problematisk.
Dersom ein kritisk prosesskomponent skulle svikte vil det være vanskeleg å finne reservedelar. Eventuell 
nedetid på ein slik anlegg er i praksis ikkje mogleg ettersom avløpshandtering er kritisk for miljøet og 
samfunnets velferd.
Det firma som installerte anlegget tilbake i 2003 «Watercare As» har i mellomtida blitt avvikla noko som gjer 
kompetanse innan det eksisterande styresystemet vanskeleg få tak i.
Anlegget har også begrensa fjernstyring og overvaking noko som gjer lokalt tilsyn nødvendig.
Dokumentasjonen knytt til anlegget er mangelfull. Det føreligger begrensa dokument som beskriver drift og 
vedlikehald av anlegget. Det er behov for en grundig og omfattande dokumentasjonsprosess for å sikre at alle 
relevante aspekt av anleggets funksjonalitet og tekniske detaljer blir følgt og dokumentert.


\begin{figure}[htbp]
    \centering
    \includegraphics[width=0.5\textwidth]{Bilder/mango.jpg}
    \caption{Mango Ipa We Like}\label{fig:Mango-Logo}
\end{figure}
    

	% Kapittel 2
	
\chapter{Forrapport}
\thispagestyle{fancy}
%


\section{Forrapport}

    


	% Kapittel 3
	\section{Anleggets Virkemåte}
\thispagestyle{fancy}

\section{Aktiv slam - Sequencing Batch Reaktor(SBR)} 
SBR står for Sequence Batch Reactor. På anlegget er det nytta SBR-teknologi, en reinsemetode basert på aktiv slam der alle prosessar føregår i same reaktortank. 
Reaktor nyttar biologisk reinsing for å koagulere og fjerne ikkje sedimenterbare partiklar og stabilisere organisk materiale. Avløpsvatn tilførast reaktor i «batcher» for å bli reinsa og uttappa. 
Kvar avløps-batch går igjennom ein reaktorsekvens som består av fem delsekvensar:

\begin{itemize}
    \item Pause: \newline Reaktoren venter til det er behov for kapasitet.
    \item Innpumping: \newline Reaktoren mottar avløpsvatn frå mottakstanken. 
    \item Reaksjon: \newline Reaktoren luftast for å tilføre oksygen til mikroorganismane som igjen bryter ned organisk materiale, 
    og næringsstoffet som nitrogen og fosfor.
    \item Sedimentering: \newline 
    I sedimenteringsfasen skilast dei tyngre partiklane frå vatnet ved hjelp av gravitasjon. 
    Blåser og alle ventiler stengast i denne fasen for å oppnå rolege og stabile sedimenteringsforhold. Dette gir lave konsentrasjonar av suspendert stoff i avløpet.
    \item Uttapping: \newline Reinsa vatn drenerast mot resipient
\end{itemize}

\begin{figure}[htbp]
    \centering
    \includegraphics[width=1\textwidth]{Figurar/SBR-Prosessen.png}
    \caption{SBR-Prosessen}\label{fig:SBR-prossessen}
\end{figure}
    
	\newpage
	
	% Kapittel 4
	\chapter{Programering}
\thispagestyle{fancy}
Anlegget er programert med PLS. no worries.


	% Kapittel 5
	\chapter{JOKERNORD!}
\thispagestyle{fancy}
Du har nå gambla vekk alle pengane dine, vil du fortsette og bruke bæstemor sine penga istedenfor?
Flott, da kjører vi.

\begin{figure}[htbp]
    \centering
    \includegraphics[width=0.5\textwidth]{Bilder/JokerNord.jpg}
    \caption{Gjeld er for alle.}
    \label{fig:JokerNord}
\end{figure}
    

	% Kapittel 6
	\chapter{Testing}
\thispagestyle{fancy}
Alt er timet og tilrettelagt.


\begin{figure}[htbp]
    \centering
    \includegraphics[width=0.5\textwidth]{Bilder/mango.jpg}
    \caption{Mango Ipa We Like}
    \label{fig:Mango-Logo}
\end{figure}
    

	% Kapittel 7
	\chapter{Avslutting}
\thispagestyle{fancy}
Og så avslutta vi alt, og det var fint vær og konge.

\begin{figure}[htbp]
    \centering
    \includegraphics[width=0.5\textwidth]{Bilder/mango.jpg}
    \caption{Mango Ipa We Like}
    \label{fig:Mango-Logo}
\end{figure}
    
	% Førstesiden skal ha følgende marger.
\newgeometry{top=2.5cm,bottom=2.5cm,right=3cm,left=2cm}

\noindent
\begin{center}
  \textbf{Utført av:}
\end{center}

\vspace{1em} % Legg til vertikal avstand etter overskriften

\noindent
\begin{tabular}{>{\centering\arraybackslash}m{5cm} >{\centering\arraybackslash}m{5cm} >{\centering\arraybackslash}m{5cm}}
  \textbf{\textcolor{mybla}{Vegard Aven Ullebø}} & \textbf{Roar Boyum} & \textbf{Peter Søreide Skaar} \\
  596932 & 597238 & 597237 \\
  Høgskulen på Vestlandet & Høgskulen på Vestlandet & Høgskulen på Vestlandet \\
  Automatisering med robotikk & Automatisering med robotikk & Automatisering med robotikk \\
  Førde & Førde & Førde \\
  \href{mailto:v.ulleboe@gmail.com}{\textbf{\textcolor{mybla}{v.ulleboe@gmail.com}}} & \href{mailto:roar.boyum@gmail.com}{\textbf{\textcolor{mybla}{roar.boyum@gmail.com}}} & \href{mailto:peter.skaar@gmail.com}{\textbf{\textcolor{mybla}{peter.skaar@gmail.com}}} \\
\end{tabular}


\vspace{2em} % Legg til vertikal avstand før datoen

\noindent
\begin{center}
  \textbf{05.02.2024}
\end{center}
	
% Endrer skriften for resten av dokumentet
{\fontfamily{qpl}\selectfont}

\vspace{1cm}

\setstretch{0.8}
section{Testing}
{\fontsize{9.5}{20}\selectfont
	I confirm that the work is self-prepared and that references/source references to all sources used in the work are provided, cf.\ \href{https://lovdata.no/dokument/SF/forskrift/2016-12-21-1851#KAPITTEL_10}{Regulation relating to academic studies and examinations at the Western Norway University of Applied Sciences (HVL), §~10}.}
	

  \section*{The product}
  \textbf{CODESYS – a strong brand}\\
  \vspace{0.5cm}
  CODESYS is the most widely used manufacturer independent IEC 61131-3 Development System on the market. And not without reason: As an independent software manufacturer we maintain a close dialogue with our customers but are also free to pursue new concepts and implement our ideas. With a strong market intuition and profound technological knowledge, we have turned CODESYS into a strong brand. And that is what our customers around the world have been benefiting from for almost 30 years now.

  We cover the complete software side of automation: CODESYS offers the complete functionality of a modern IEC 61131-3 development tool; including an integrated visualization with different clients, integrated connection to all standard fieldbus systems, motion functionality, safety solutions, communication interfaces, and a solution for Industry 4.0 that allows for a convenient remote management of control landscapes.
  
  The CODESYS ST Editor -the CANopen® stack in IEC -our concept for object-oriented programming -the integrated CODESYS Safety solution  -every CODESYS product is developed following a straightforward principle: We aim for the best.
	%% Definer sidemargene
\newgeometry{top=2.5cm,bottom=2.5cm,right=3cm,left=5cm}

% Definer seksjonstitler
\titleformat{\section}
  {\normalfont\Large\bfseries}{\thesection}{1em}{}
\titleformat{\subsection}
  {\normalfont\large\bfseries}{\thesubsection}{1em}{}

% Definer header og footer
\fancyhf{}
\fancyhead[L]{\includegraphics[height=2cm]{mango.jpg}} % Bytt ut 'logo.png' med filnavnet til din logo
\fancyhead[R]{\thepage\~av~\pageref{LastPage}}
\fancyfoot[L]{SPEC}
\fancyfoot[C]{CONFIDENTIAL}
\fancyfoot[R]{Peter Søreide Skaar \\ Student engineer/Operator \\ p.skaar@gmail.com}
\renewcommand{\headrulewidth}{0pt} % Ingen linje i header
\renewcommand{\footrulewidth}{0pt} % Ingen linje i footer
\pagestyle{fancy}


% Tittelområde
\begin{center}
  \Large \textbf{Oppdatering PLS-program} \\
  \large RA100 Førde \\
  Enkel funksjonsbeskrivelse for sampling \\
\end{center}

% Systeminformasjon-seksjon
\section{systeminformasjon}
Volumetrisk prøvetaking for \textit{etc.}

% Systembeskrivelse-seksjon
\section{Systembeskrivelse}
For å kunne starte prøvetaking remote er \textit{etc.}

% Tabell
\begin{tabularx}{\textwidth}{|X|X|X|}
  \hline
  Start: & Pulser: & Sample: \\
  \hline
  \textit{etc.} & \textit{etc.} & \textit{etc.} \\
  \hline
\end{tabularx}

% Resten av dokumentet følger her...






\end{document}