\documentclass[11pt, a4paper]{report}
%-----------NORSK LaTeX--------------

% Denne pakken inkluderer skandinaviske bokstaver( æ, ø og å)
\usepackage[utf8]{inputenc}

% Denne pakken bruker norsk typesetting (innholdsliste osv.)
\usepackage[norsk]{babel}

%--------DOKUMENTOPPSETT------------

% Lar deg tilpasse layouten i prosjektet
\usepackage{fancyhdr}

% Definere egne farger 
\usepackage[usenames, dvipsnames]{color}

% Avstand på marger til sidene på arket
%\usepackage{geometry}

% Linker i dokumentet (figurer, seksjoner osv)
\usepackage[hidelinks]{hyperref}

% Kildeliste-setup
%\usepackage[natbibapa]{apacite} % For apalike eller apacite
\usepackage{natbib} % For Chicago, Vancover, Harvard Style, Nature Style osv

% Spacing mellom bokstaver
\usepackage{microtype}

% Laster inn forskjellige skrifttyper
\usepackage{uarial} % Arial
\usepackage[scaled]{helvet} % Helvetica
%\usepackage{tgbonum} % Usikker
%\usepackage{mathptmx} % Times New Roman
%\usepackage{lmodern} % Latin Modern
%\usepackage[charter]{mathdesign} % Charters
\usepackage[T1]{fontenc}

% Avsnitt mellom paragrafer
\usepackage{parskip}

% Linjeavstand i dokumentet
\usepackage{setspace}
\setlength{\parskip}{1em}

% Fet skrift på alle figurnr
\usepackage[labelfont=bf]{caption}
\usepackage{capt-of}
\usepackage{caption}

\usepackage{chngcntr}
\counterwithin{figure}{chapter}
\counterwithin{table}{chapter}

% Definerer mellomrom mellom seksjons-, figur- og tabellnummer i innholdsliste til teksten begynner
\renewcommand{\numberline}[1]{#1 \,}

% Kan skrive i kolonner
\usepackage{multicol}

% For appendix
\usepackage{titletoc}
\usepackage{appendix}

%----------ANDRE PAKKER--------------

% Pakker fra AMS. Kan skrive matteformler
\usepackage{amsfonts, amssymb, amsthm}
\usepackage{mathtools}

% For grafikk (lime inn bilder, pdf-dokumenter)
\usepackage{graphicx}
\usepackage{float}
\usepackage{pdfpages}
\usepackage{subcaption}

% For inportering av csv-filer som tabeller. ( Ikkje i bruk)
%usepackage{csvsimple}

% Pakke for citations
%\usepackage{biblatex}
% Pakke for at url linker skal fungere i bl.a. referansar
\usepackage{url}

% Generere tekst
\usepackage{kantlipsum}

% Definere egne objekt
\usepackage{tikz}

% Lar deg lage lister på en mer oversiktlig måte
\usepackage[sharp]{easylist}

% Boks rundt tekst
\usepackage{fancyvrb}

% Test pakker
\usepackage{array}
\usepackage[margin=1in]{geometry} % Juster margene etter behov
\usepackage{titlesec}
\usepackage{tabularx}
\usepackage[table]{xcolor}
\usepackage{lastpage}

% pakke for ordliste
\usepackage[automake]{glossaries-extra}

%---------------------------------------------------

% Velger arial som skrifttype i dokumentet 
%\renewcommand{\rmdefault}{qpl} % 
%\renewcommand{\rmdefault}{phv} % Arial
\renewcommand*\familydefault{\sfdefault} % Helvetica(sans-serif)


% Definerer egne farger
\definecolor{mybla}{RGB}{25, 49, 90}
\definecolor{mylysbla}{RGB}{0, 136, 176}
\definecolor{renasysGreen}{RGB}{4, 255, 4}
\definecolor{purewhite}{RGB}{255,255,255}
\definecolor{lightgray}{gray}{0.8}
\definecolor{myblack}{gray}{0.1}

% Velger norsk språk på innholdsfortegnelse, figurliste, tabelliste og Referanseliste
% Legger disse så til i innholdsfortegnelsen.
\addto\captionsnorsk{\renewcommand{\contentsname}{Innhaldsliste}}
\addto\captionsnorsk{\renewcommand{\listfigurename}{Figurliste}}
\addto\captionsnorsk{\renewcommand{\listtablename}{Tabelliste}}
\addto\captionsnorsk{\renewcommand{\bibname}{Referansar}}
% Nynorsk!
\addto\captionsnynorsk{\renewcommand{\contentsname}{Innhaldsliste}} 
\addto\captionsnynorsk{\renewcommand{\listfigurename}{Figurliste}}
\addto\captionsnynorsk{\renewcommand{\listtablename}{Tabelliste}}
\addto\captionsnynorsk{\renewcommand{\bibname}{Referansar}}


% ----- Justering av avstander mellom chapter / section / subsection / subsubsection

% Justeringa av chapter, 
% første verdi : Horisontalverdi frå venstre marg
% Andre verdi : avstanden frå toppen av sida til tittelen
% Tredje verdi : Avstand til neste tekst
\titleformat{\chapter}[hang]
  {\normalfont\bfseries\LARGE}{\thechapter}{1em}{}
\titlespacing*{\chapter}{0pt}{-50pt}{20pt}

% Juster avstanden for \section
\titleformat{\section}
  {\normalfont\Large\bfseries}{\thesection}{1em}{}
\titlespacing*{\section}{0pt}{10pt}{10pt}

% Juster avstanden for \subsection
\titleformat{\subsection}
  {\normalfont\large\bfseries}{\thesubsection}{1em}{}
\titlespacing*{\subsection}{0pt}{10pt}{10pt}

% Juster avstanden for \subsubsection om nødvendig
\titleformat{\subsubsection}
  {\normalfont\normalsize\bfseries}{\thesubsubsection}{1em}{}
\titlespacing*{\subsubsection}{0pt}{10pt}{10pt}

% Define the page style - Her lager Peter til OP mal
\fancypagestyle{fancy}{
  % Clear the header and footer
  \fancyhf{}
  % Set the right side of the footer to be the location of the page number
  \fancyfoot[R]{Side \thepage\ av~\pageref{LastPage}} % Her sett ein fra side til side
  \renewcommand{\headrulewidth}{0pt} % Remove header line
  \renewcommand{\footrulewidth}{0pt} % Remove footer line
}

% ------ Justering av avstander i dokumentet for itemize ---------------------

\let\olditemize\itemize
\renewcommand{\itemize}{
  \olditemize
  \setlength{\itemsep}{-10pt} 
}

% -------------------------------------------------------------------


% ------------------------- KODE FOR ORDLISTE ---------------------

% Kode for å lage ei meir kompakt ordliste
\newglossarystyle{compact}{%
  % base this style on the list style:
  \setglossarystyle{list}%
  % adjust the space between entries:
  \renewenvironment{theglossary}{%
    \begin{description}%
    \setlength{\itemsep}{-10pt}% adjust this value as needed
    \setlength{\parskip}{0pt}%
  }%
  {%
    \end{description}%
  }%
}

\setglossarystyle{compact}

% Kode for å lage stil for ordlista
\newglossarystyle{tablestyle}{
  % Start the glossary as a tabular environment
  \renewenvironment{theglossary}{
	\setlength{\LTleft}{0pt} % Align the table with the left margin
    \begin{longtable}{lp{\glsdescwidth}}
    }{
    \end{longtable}
  }
  \renewcommand*{\glossaryheader}{
    \bfseries Term & \bfseries Description \\
    \endfirsthead
    \bfseries Term & \bfseries Description \\
    \endhead
  }
  % No heading between groups
  \renewcommand*{\glsgroupheading}[1]{}
  % No space between groups
  \renewcommand*{\glsgroupskip}{}
  % Define how each entry should appear:
  \renewcommand*{\glossentry}[2]{
    \glstarget{##1}{\glossentryname{##1}} &
    \glossentrydesc{##1} \\
  }
}

% Setter Stil for Ordlista
\setglossarystyle{tablestyle}


\makeglossaries
\loadglsentries{Ordliste.tex}

% Stil for referansar
\bibliographystyle{unsrt}

% Endre namn på overskrifta til referanselista


% ---------------------------------- END kode for ordliste -----------------------------

% ------------------------------------ Dokumentet starter her ----------------------------

\begin{document} % Dokument start

	% Turn on the style
	\pagestyle{fancy}

	%------------------FREMSIDE------------------------

	% Linker inn fremsiden fra adms (administrative sider) fremside-filen.
	% Førstesiden skal ha følgende marger.
\newgeometry{top=2.5cm,bottom=2.5cm,right=3cm,left=5cm}

% Ønsker ikke sidetall på denne siden.
\thispagestyle{empty}


% Legger inn banner
\begin{center}
	\begin{tikzpicture}[remember picture, overlay]
	\node [xshift=0cm, yshift=11.5cm] at (current page.center) 
	{\includegraphics[width=\paperwidth]{Bilder/frontpage.png}};
	\end{tikzpicture}
\end{center}



\vspace{6cm}

{\fontsize{36}{50}\selectfont \textbf{\textcolor{mybla}{Programmering PLC}}}


\vspace{1cm}
% Forfattarar
{\fontsize{26}{30}\selectfont \textbf{\textcolor{mylysbla}{Roar Bøyum}}}\\
{\fontsize{26}{30}\selectfont \textbf{\textcolor{mylysbla}{Vegard Aven Ullebø}}}\\
{\fontsize{26}{30}\selectfont \textbf{\textcolor{mylysbla}{Peter Søreide Skaar}}}\\ 


\vspace{1.3cm}


\setstretch{0.7}

{\fontsize{20}{20}\selectfont\textcolor{mylysbla}{Name of Masterprogram}} 

\vspace{-0.4cm}
{\fontsize{20}{20}\selectfont\textcolor{mylysbla}{Department/Institute/Program}} 

\vspace{-0.4cm}
{\fontsize{20}{20}\selectfont\textcolor{mylysbla}{Supervisor (in agreement with supervisor)}} 

\vspace{-0.4cm}
{\fontsize{20}{20}\selectfont\textcolor{mylysbla}{Submission Date}} 



% Endrer skriften for resten av dokumentet
{\fontfamily{qpl}\selectfont}


\vspace{1cm}

\setstretch{0.8}
{\fontsize{9.5}{20}\selectfont
	I confirm that the work is self-prepared and that references/source references to all sources used in the work are provided, cf.\ \href{https://lovdata.no/dokument/SF/forskrift/2016-12-21-1851#KAPITTEL_10}{Regulation relating to academic studies and examinations at the Western Norway University of Applied Sciences (HVL), §~10}.}
	

	%------------------------------------------------------

	\newpage

	% Nye marger
	\newgeometry{top=2.5cm,bottom=3cm,right=2.2cm,left=2.2cm}

	% Linjeavstand
	\setstretch{1.3}

	% Set the right side of the footer to be location of the page number

	% Ønsker tallnummerering videre.
	%-------------------- 			FORORD			---------------------	
	\chapter*{Forord}
\thispagestyle{fancy}
Ønskjer å takke, Hansa Mango IPA for ei fattig trøst til ein fattig student.


% Ønsker ikke sidetall på denne siden.
\thispagestyle{empty}

	%--------------------Innhaldsliste / Figurliste / tabell liste---------------------	
	% Innholdsliste
	\tableofcontents % Genererer Innhaldsliste
	\thispagestyle{empty} % Fjerner sidetall
	\listoffigures % Genererer figur liste
	\thispagestyle{empty}
	%\quad
	\listoftables % Genererer ei liste med tabeller
	%\thispagestyle{empty}
	%\cleardoublepage
	\renewcommand{\glossaryname}{Ordliste} % Denne er med for å kunne endre navn på ordlista
	\printglossaries %Printer ut alle ord som er brukt frå ordlista
	

	% Set counter setter sidetallet for innholdsliste-siden.
	\pagenumbering{arabic}
	\setcounter{page}{5}

	%------------------------------------------------------

	%------------------ Lim inn sider til documentet her ------------------------



	% Kapittel 1 Innleiing
	
\thispagestyle{fancy}

\section{Om oss}

Vi er tre studentar som studerar Automasjon med robotikk ved HVL campus Førde.
Vi har alle fagbrev som elektrikar og fant lett tonen i starten på studiet.
Igjennom tre år har vi brukt vår breie kompetanse innen industri, programmering og elektronikk
til å danne eit godt team.

Fyll inn?



	\section{Oppdragsgivar}
\textbf{Renasys AS i samarbeid med Sunnfjord kommune}

Renays\citep{Renasys} er ein startup som arbeider med banebrytande teknologi innan mekanisk finpartillekfiltering av avlaupsvatn.
Renasys har 14 tilsette fordelt på kontor på Øyrane i Førde og Sandnes i Rogaland. 
Renasys har lenge arbeida konfedensielt men har gått offentleg med teknologien sin iløpet av 2023. 
Renasys tilbyr reinsetjenester til kommunar og interkommunale selskap innan avlaup og maritim sektor.

Renasys og Sunnfjord kommune\citep{SunnfjordKommune} arbeider ilag mot Mission Zero som innebærer 
null utslepp, null avfall, null energi og ein generell modernisering av avlaupssektoren i Norge.
Sunnnfjord kommune er ansvarleg for vann, veg og avlaup i sitt område og har bedt 
Renasys om å undersøke moglegheiter for forbetringar av reinseanlegget på Sande i Sunnfjord.
\newline
\newline

\begin{figure}[htbp]
    \centering
    \begin{subfigure}[b]{0.3\textwidth}
        \centering
        \includegraphics[width=1\textwidth]{Bilder/renasys.png}
        \caption{Logo Renasys}\label{fig:subfig1}
    \end{subfigure}
    \hfill
    \begin{subfigure}[b]{0.3\textwidth}
        \centering
        \includegraphics[width=0.7\textwidth]{Bilder/SK.png}
        \caption{Logo Sunnfjord kommune}\label{fig:subfig2}
    \end{subfigure}
    \caption{Logo oppdragsgivar}\label{fig:Illustrasjon-Diffuser}
\end{figure}

	% Kapittel 2 Analyse av problemet
	\chapter{Analyse av problemet}
\thispagestyle{fancy}
Her kan vi starte underlaget for kappittelet
	\section{Problemstilling}
Reinseanlegget er teknisk utdatert og er avhengig av modernisering. Stryringssystemet er over tjue år gammalt
og består hovudsakeleg av eldre og utgåtte komponentar. Med gamle komponentar aukar risikoen for svikt, 
og reservedelar som passar kan være vanskeleg å finne.

WaterCare, som leverte styresystemet er i seinare tid også blitt avvikla noko som gjer at kompetansen 
og moglegheiten for å gjere endringar i styresystemet er vankeleg. 
Mykje grunna denne utfordrina har ikkje anlegget følgt den teknologiske framdrifta 
og små problem har samla seg opp til å bli større utfordringar.

Samtidig med desse faktorane er dokumentasjonen til reiseanlegget dårleg noko som gjer at enkle arbeidsoppgåver blir lange og tunge.
I eit værste tilfelle vil styringseinheten på anlegget svikte og med det som er nevnt tidlegare vil det være krevjande
å få anlegget i drift igjen. Dette er eit kritisk problem innan avlaupshandtering og kan ikkje sjåast vekk ifrå.
\newline

\begin{figure}[htbp]
    \centering
    \includegraphics[width=0.8\textwidth]{Bilder/BeijerSkjerm.JPG}
    \caption{Beijer HMI}\label{fig:HMI}
\end{figure}
	
\section{Løysningsforslag}
Fellsfaktor for løysningsalternativ til reinseanlegget vil vere å modernisere 
styresystemet med ein ny moderne styringseining (PLS). 
Dette vil være med å løyse mykje av utfordringane som anlegget har idag. 
Det er fleire måtar og utføre dette på men vi velger å ta med dei som er mest relevant.
\newpage


\subsection{Løysningsalternativ: Bytte PLS}
Den enklaste løysninga er å oppgradere eksisterande styresystem ein til ein. 
Dette vil seie å skifte ut den eksisterande PLS'en med ein moderne PLS frå same levrandør som køyrer lik software.
Dette vil minke faren for kritisk komponentsvikt men sidan PLS-programmet er det same vil det ikkje gjere det enklare å foreta
software endringar.

\subsection{Løysningsalternativ: Bytte PLS, oversette kode}
Eit steg vidare frå løysningsalternativ bytte PLS er å også oversette gammal software over til eit programmeringsspråk
som er meir egna og der kompetansen er større. Ved denne løysningen ville ein oversatte logikk for logikk, litt som å
oversette mellom engelsk og norsk.\newline
Problemet med dette alternativet er at ein risikerer å ta med seg dårleg kode over på det nye språket og
det vil være problematisk å gjere software endringar ettersom heilheita i programmet ikkje blir tolka.

\subsection{Løysningsalternativ: Styresystem A til Å}
Ein tredje meir komplett løysning bygger på dei to førre, men forholder seg rundt problematikken med endigar i software.
Dette løysningsalternativet forholder seg ikkje til PLS eller den gamle koden.\newline
Ved å sette seg inn i anleggets teknologisk verkemåte vil ein kunne få eit bilete av korleis anlegget skal fungere teoretisk.
På denne måten vil ein kunne oprette ein ny funksjonsbeskrivelse til anlegget og bygge ein software utifrå denne funksjonsbeskrivelsen.
Dette vil som å starte på ny, men det vil gi sikkerheit på at anlegget driftast optimalt og at software følger
dagens standard. Dette tilsvarer ein prosess av eit nytt anlegg.

\subsection{Løysningsalternativ: Nytt anlegg, moderne teknologi}
I tillegg til ein full gjennomgang av styresystemet, 
kan ein undersøke nye løysningar for å optimalisere, forbetre og utvide heile reinseanlegget.
Ilag med Renasys arbeider Sunnfjord kommune med å undersøke moglegheiten og kostnadane for eit slikt alternativ,
men første steg i ein slik prosess er uansett eit løysningsalternativ som forholder seg til problema rundt styresystemet.

	\newpage
\section{Val og Drøfting}
	
	% Kapittel 3 Krav og mål
	\chapter{Krav og mål}
\thispagestyle{fancy}
Her kan vi starte underlaget for kappittelet

	% Kapittel4 - Anleggets Verkemåte
	\chapter{Anleggets verkemåte}
\thispagestyle{fancy}

Første steg i løysninga var å sette seg inn i anleggets verkemåte.\newline
For å setje oss in i korleis Sande reinseanlegg fungerar var vi nødt til å forstå
korleis eit generelt avlaupsreinseanlegg er bygd opp.



	\section{Generel verkemåte}

Eit avlaupsreinseanlegg er bygd opp av 3 hovuddelar: Primær, sekunder og tertiærreinsing,
samt ein del for behandling av slam.
Alle desse delane kan løysast på forskjellige måtar, men hovudoppgåvene er dei same
i alle reinseanlegg.

Primærreinsing handlar om å skille organisk og uorganisk materiale.
I eit avlaupsreinseanlegg tilsvarer dette å skille avlaupsvatnet, 
som ein vil behandle frå, sand, Q-tips, våtserviettar
og anna uønska material som ein ikkje ynskjer vidare i prosessen. \newline
Primærreinsing er eit viktig steg for å beveare pumper og anna prosessutstyr.

Sekunderreising handlar om å fjerne mest mulig suspanderte stoffer og organisk materiale.
Det tilsvarer å skilje det meste av organisk materialet frå vatnet.
Sekunderreising er i kvart anlegg avhengig av kva `reinseprinsipp' som er brukt. Dette tilsvarer
kva teknolgisk metode som nyttast for å utføre dette steget.
Sekunderreising refererast oftast til biologiskreinsing.

Tertiærreising handlar om å fjerne resterande forureiningar i vatnet.
Dette steget varierer veldig frå anlegg til anlegg og eg er
avhengig av kva krav reiseanlegget har som krav på sitt utslippsvatn.

Slamebehandling handler om å fjerne og behandle det oppbygde organiske materialet (Slam)
som skillast ut i sekundær og tertiærreinsing 

\begin{figure}[htbp]
    \centering
    \includegraphics[width=1\textwidth]{Figurar/Generellverkemåte.png}
    \caption{Generell verkemåte for eit avlaupsreinseanlegg}\label{fig:HMI}
\end{figure}

Når vi hadde satt oss inn i den generelle verkemåten til eit avlaupsreinseanlegg
valgte vi å fokusere mot Sande.
Dette valgte vi å gjere i tre hovuddelar. Kva teknologisk reinseprinsipp er brukt, korleis er Sande reiseanlegg
kopla opp mot teknologien teoretisk og korleis fungere reinseanlegget praktisk tjue år etter innstalasjon.
	\input{Tekst/Kapittel4 - Anleggets Verkemåte/4.2 - Teknologisk Virkemåte.tex}
	\newpage
\section{Teoretisk Virkemåte}
\thispagestyle{fancy}


% Sande reinseanlegg består av primærreinsing via ein grovrist. Frå grovrista
% rennet vatnet med sjølvfall til ein mottakstank som fungerar som utjamningstank og samlar varierande tilstrøymingar 
% for å gi resten av anlegget homogene forhold. 
% Vidare blir vatnet pumpa frå mottakstanken opp til ein av reaktorane.
% Innpumping skjer til den reaktoren som er i riktig fase og er klar for ny batch.
% Sekundær og tærtierreinsing skjer i desse reaktorane som anvender SBR-teknologi.

% Dersom ingen av reaktorane er i riktig fase vil avlaupsvatnet lagrast i mottakstanken
% heilt til ein av reaktorane har avslutta sin syklus.
% Avlaupsvatnet vil opphalde seg i eller på veg mot ein av desse fire hovuddelane medan det er i anlegget.

Sande reinseanlegg består av primærreinsing via grovrist, ein mottakstank 
som samlar varierande tilstrøymingar for å gi resten av anlegget homogene forhold og
sekundær og tærtierreinsing ved to reaktorar som anvender SBR-teknologi.

Avlaupsvatnet vil opphalde seg i eller på veg mot ein av desse fire hovuddelane medan det er i anlegget.
Ferdig behandle avlaupsvatn blir drenert ut til resepienten Gaula. 
Gaula er ei elv som renn ifrå Hestad, forbi Sande og munner ut i Dalsfjorden.

\begin{figure}[htbp]
    \centering
    \includegraphics[width=1\textwidth]{Figurar/Sande verkemåte.png}
    \caption{RA200 flytskjema}\label{fig:HMI}
\end{figure}

\subsection{Grovrist}
Innløpet på anlegget renn først igjennom grovrista. Grovrista på reinseanlegget
er ein (Huber rotomat R09). Grovrista fungerer som ein liten tank og ein nivågivar starters
skruen ved innkommande avlaupsvatn. Skruen tek med unøska materiale og fjernar det til eigen avfallshandtering.
Dersom grovrist feile vil vatnet renne vidare via overløpsrøyr.

\newpage
\subsection{Mottakstank}
Frå grovrista renner vatnet med sjølvfall mot mottakstanken som ligger som lavaste punkt på anlegget.
Mottakstanken er 120 $m^3$ og er ein felles lagringsplass for vatnet før det går vidare mot reaktorane.
Mottakstanken har fire sensorar som heng ifrå taket.

\begin{itemize}
    \item Trykkgivar for nivå (PP00-LT01)
    \item Trykkgivar for overløp (PP00-LT02)
    \item Flottør-vippe lav (PP00-LS02)
    \item Flottør-vippe høg (PP00-LS01)  
\end{itemize}

Nivået i mottakstanken blir primært målt med trykkgivar LT01. For at vatnet skal pumpast vidare mot ein
reaktor i riktig sekvens må trykkgivar indikere at nivået er høgt nok. LS02 fungerar som backup.
I toppen av mottakstanken er det ei overløpskasse som drenerer mot resepient, her vil det
ved normale omstendigheter ikkje renne anna ein reinsavatn. Trykkgivar for overløp måler
dersom ureinsavatn renner i resepientrøyret.

%% Må endre på tegning fordi det eine navnet er feil. Ikkje reject frå sivbed.

\begin{figure}[htbp]
    \centering
    \includegraphics[width=1\textwidth]{Figurar/Mottakstank.png}
    \caption{Illustrasjon mottakstank}\label{fig:HMI}
\end{figure}

\newpage
\subsection{Reaktor}


	\newpage
\section{Praktisk Virkemåte}
\thispagestyle{fancy}

Sjølv om Sande reiseanlegg anvender SBR-teknologi så er det enkle spesefikke
punkter der dette reiseanlegget avviker ifrå normen. 
Sande reinseanlegg bruker eksempelvis ein spesiel form for slambehandling sett i norsk perspektiv.

Vanleg slambehandling i Noreg er oppsamling av slammet i tanker som handterast og tømmast av lokale etater.
På Sande blir ikkje slammet lagra, men jamnlig spreid ut over eit designert område. På dette området er
det planta siv som skal ta opp slammet og det resterande vatnet blir naturleg filtrert og drenert.
Desse områda kallast sivbed.

Sivbeda er konstruert med fleire dreneringslag som gjer at resterande vatn skiljast ut i designerte soner.
Her kan desse handterast vidare etter ønska behov. 
På Sande reiseanlegg er djupaste dreneringssona klassifisert som rein reject og blir sendt ut til resepient.
Den øvre dreneringssona er forsatt klassifisert som skitten og blir returnert til mottakstanken.

Grunna denne slambehandlingsmetoden er det på Sande reiseanlegg heller slamfjerning frå reaktor
i reaksjonsfasen. Dette blir gjort for å ha mindre konsentrert slam.

\begin{figure}[htbp]
    \centering
    \includegraphics[width=1\textwidth]{Figurar/Sivbed.png}
    \caption{Illustrasjon sivbed}\label{fig:HMI}
\end{figure}


	\newpage
\section{Dokumentasjon}
\thispagestyle{fancy}

Som tidlegare spesifisert i kapittel 3 var ein stor del av oppgåva vår og dokumentere
anlegget og verkemåten til anlegget. Vi bestemte oss for å gjere ei dokumentasjonsfornying,
altså at vi henta det som var av tidlegare dokumentasjon og kombinerte det med vår nyerfarte kunnskap.

Vi oppretta ein funksjonsbeskrivelse som bygger vidare på driftsinstruksen som var levert av 
Watercare i 2003. Dokumentet er tiltenkt ein slags bruksanvisning på heile anlegget,
der sikkerheit, prosess, verkemåte og programmering er sentrale tema. \newline
Funksjonsbeskrivelsen skal være forståeleg for alle parter som har interesse i reinseanlegget
og både driftspersonell og programmerere skal kunne få eit godt innblikk i anlegget.

Funksjonsbeskrivelsen inneheld alt om reinseanlegget på Sande men er delt opp i kapittel for å ikkje
overvelde lesar med mykje unødvendig informasjon. Desto lenger ned i dokumentet ein kjem jo meir avansert blir infomrasjonen
og programmering av anlegget ligger under udjupa teknisk beskrivelse.

Dokumentet er delt opp i desse kapittela

\begin{enumerate}
    \item Introduksjon
    \item Verkemåte
    \item Teknisk beskrivelse
    \item Drift og vedlikehald
    \item Feilsøking
    \item Utdjupa teknisk beskrivelse
    \item Teknisk underlag
\end{enumerate}

Alt som er nemnt her i kapittel 4 står meir detaljert under kapittel Teknisk beskrivelse i funksjonsbeksrivelsen, og
er sentralt for å best forstå korleis anlegget fungerer.

Funksjonsbeskrivelsen i sin heilhet ligger som vedlegg (SETT INN VEDLEGG)

	% Kapittel 5 - Tilrettelegging for programmering
	\chapter{Tillrettelegging for programmering}
\thispagestyle{fancy}
\label{sec:5} 

Som tidlegare beskrive under krav og mål, ønska vi å bruke ein løysning som ikkje har løypande lisenskostnadar for vår sluttkunde. 
Sunnfjord kommune var veldig interessert i å ikkje låse seg til ein fast leverandør, men heller ha moglegheita til å ha valet mellom fleire 
leverandørar innan levering av PLS. Dette gjorde at vi såg vekk ifrå Siemens TIA-portal som vi hadde lært
igjennom PLS faget, og begynte å sjå i andre endar.
Programmet ønska vi å skrive i Structured Text (ST) men fikk også anbefalt typar av grafiske diagram baserte språk for å lettare
vise sammenheng i programmet.

Det var også viktig at alle på gruppa kunne programmere samtidig og at ein felles programmeringsstandard skulle nyttast.
Det var viktig at parralelt arbeid ikkje skulle by på synkroniseringsproblemer og at det var ei god løsyning for dette.

	\section{Codesys}
\thispagestyle{fancy}
Som tidlegare beskrive under krav og mål, ønska vi å bruke ein løysning som ikkje har løypande lisenskostnadar for våra sluttkunde. 
Kommunen var veldig interessert i å ikkje låse seg til ein fast leverandør, men heller ha moglegheita til å ha valet mellom fleire leverandørar innan levering av PLS, så valet falt på programeringsplattforma Codesys\citep{Codesys} laga av Codesys Group. 
Codesys tilbyr ein open kildekode løysning for prosjekta, og har ingen lisenskostnadar for sluttkunden. 
I tillegg så kan prosjektfilane brukast på fleire typar PLS einingar. 
Dette gir våra sluttkunde fleksibilitet i korleis dei ønska å implementere våra løysningsforslag til deira anlegg.

Codesys støttar programeringsspråkstandaren satt av IEC 61131 som blant anna Structured Text (ST), Sequential Function Chart (SFC) og Ladder Diagram (LD). I våra program er all logikk skrevet i strukturert tekst (ST), og bygd opp av ein blokk-basert programering med Continuous Function Chart (CFC). 
CFC er en grafisk programmeringsspråk som utvidar dei standardiserte språkene i standarden og gir koden ein god lesbarhet.

Codesys har nyleg fått støtte for integrering av Github i programmvaren, som gjer det mykje enklare for oss å halde versjonskontroll og enkelt for fleire av gruppemedlemmene å kode saman på same prosjekt. 
Denne funksjonaliteten har vi nytta oss av flittig, og har hjelpt oss med å kunne ha moglegheita til å jobbe individuelt med prosjektet. 
\newpage
	\section{IEC}
\thispagestyle{fancy}
\label{sec:5.2}


\gls{IEC} \citep{IEC} er ein internasjonal, ikkje statleg orginasjon som utviklar og publiserer tekniske standardar innan elektrofag. 
Norge er representert i IEC ved Norsk Elektrotekniske Komité (NEK) \citep{IEC-SNL}. 
IEC har standarar som dekker programmering av PLS som går heilt tilbake til 1993\citep{Wiki-93}. 
Den nåverande standaren som omfavner PLS er IEC 61131\citep{IEC-61131}. Dette er ein standar spesielt designet for programmerbare kontrollera, og er delt opp i 10 delar, der del 3 tar for seg programmeringsspråk. 

Våra program er tiltenkt programmert i hovudsak etter IEC 61131-3 og IEC \gls{PAS} 63131\citep{IEC-63131}. 
Der IEC PAS 63131 er ein standard utarbeida av IEC som gir oss grunnlag for å lage \gls{SCD} samt å bruke forhandsdefinerte funksjonstemplater for funksjonsblokker. 
IEC PAS 63131 er laga med formål at leverandørindustrien og oljeselskap skal ha et felles rammeverk for bruk på norsk sokkel, og er utarbeida etter NORSOK I-005:2013.
Med å bruke desse standardane så gir det oss eit robust og fleksibelt rammeverk for å programmere anlegget. 
Ved bruk av dei forhandsdefinerte funksjonsblokkane har vi moglegheit til å enkelt knytte i hop fleire delar av programmet våra, og ha fleksibilitet ved å enkelt kunne endre og legge til funksjonar i programmet. 
Dette er noko vi har heile vegen har fokusert på, da våra sluttkunde har ytra eit ønske om å ein del tilleggsfunksjonar i programmet, utover det som er der i dag. 

Vi bestemte oss for å fokusere på fire funksjonsblokker (MB, MA, SBV og SBE, sjå appendiks \ref{sec:IEC-Blokker})  
i frå IEC 63131 for å få dekka behovet for dei komponentane vi hadde, og for å kunne programmere anlegget i sin heilheit.
Sjølv om blokkene kunne verke overkvalifiserte valte vi likevel å ha dei med, da nokre av tilleggsfunksjonane eventuelt kunne nyttast seinare
og at det gav oss ein klar retning å arbeide mot.
\newpage
	
	\section{SCD}
\thispagestyle{fancy}

Etter arbeidet med å få hovudblokkene på plass var det naturleg å sette seg ned å planlegge korleis vi ønska å knytte desse blokkene opp i mot anlegget. 
Vi valte da å starte med å lage eit System Control Diagram (SCD) over anlegget. 
Eit SCD er ein grafisk representasjon av anlegget, som visar komponentane i anlegget og deira funksjon, forbindelsar og struktur. 
IEC PAS 63131 standaren gir oss retningslinjene for utforming av diagrammet.


\begin{figure}[htbp]
    \centering
    \includegraphics[width=0.35\textwidth]{Bilder/Visio_eksempel.png}
    \caption{Utsnitt frå SCD}\label{fig:SCD eksempel}    
\end{figure}


Første utkast av våra SCD innhald koplingane frå komponentar til blokker og korleis forbindelsane imellom desse skulle fungere. 
Korleis dei forskjellige sekvensane skulle interagere med våra blokker og komponentar var litt uklart for oss på detta stadiet, så vi valte å gå vidare for å starte å lage noko av programmet, slik at vi lettare kunne lage oss eit bilete av korleis vi ville løyse oppgåva.

\newpage
	\input{Tekst/Kapittel5 - Tillrettelegging for programmering/5.4 - Tiltenkt tilstandsmaskin.tex}

	% Kapittel 6 - Programmering
	\chapter{Programmering}
\thispagestyle{fancy}
I dette kapittelet tek me før oss programmeringa frå start til slutt.
Vi starta programmeringa med eit heilt tomt prosjekt i \gls{Codesys}.Då dette var eit av krava våre var at me skulle
holde oss vekke frå dyre lisensar og bruke open kjeldekode. Vi har nytt oss av nokon av dei innebygde bilboteka til 
\gls{Codesys}, men desse er ikkje nokon lisensar på.


%% CODESYS is a software platform for industrial automation technology. 
%%The core of the platform is the IEC-61131-3 programming tool "CODESYS Development System".
	\section{Programmering av blokker}
\thispagestyle{fancy}

%Skriv OM:
%Programmering av IEC Blokker (MB,MA,SBE,SBV) 
%+ fb blokker (fbTimer, fbAnalougeAlarm, fbCAC)
%- Til appendiks, Heile kode for t.d. fbTimer, fbCAC osv

\subsection{Monitor Analogue}

\gls{MA} funksjonsblokka er brukt for skalering, visning, overvåking og alarmhandtering av analoge inngangsvariablar i ein prosess.
Funksjonsblokka inneheld supression og blokking funksjonalitet.

VI bruker \gls{MA} funksjonsblokka er brukt i programmet for å overvåke analoge trykknivågivarar samt å skalere og vise desse som ein fyllingsgrad i prosent.

\begin{figure}[htbp]
    \centering
    \begin{subfigure}[b]{0.45\textwidth}
        \centering
        \includegraphics[width=1\textwidth]{Bilder/MABlokkIEC.png}
        \caption{IEC}\label{fig:Monitor Analogue blokk IEC}
    \end{subfigure}
    \hfill
    \begin{subfigure}[b]{0.45\textwidth}
        \centering
        \includegraphics[width=0.7\textwidth]{Bilder/MABlokkIProgrammet.png}
        \caption{Bruk i programmet}\label{fig:Monitor Analogue blokk i programmet}
    \end{subfigure}
    \caption{Monitor Analogue}\label{fig:Monitor Analogue}
\end{figure}


% Skrive litt om blokka sin funksjonalitet
% Korleis me brukte blokkene i programmet.

\begin{figure}[htbp]
    \centering
    \begin{subfigure}[b]{0.45\textwidth}
        \centering
        \includegraphics[width=1\textwidth]{Bilder/4_20mA_Scaling.png}
        \caption{Skalering av mA mot prosent}\label{fig:Skalering av mA mot prosent}
    \end{subfigure}
    \hfill
    \begin{subfigure}[b]{0.45\textwidth}
        \centering
        \includegraphics[width=0.95\textwidth]{Bilder/27327_prosent_Scaling.png}
        \caption{Skalering av prosent til verdi}\label{fig:Skalering av prosent til verdi}
    \end{subfigure}
    \caption{Dei forskjellige skaleringane av inngangssignal}\label{fig:Skalering av prosent til verdi}
\end{figure}
	\section{Tilstandsmaskin}
\thispagestyle{fancy}

Etter at IEC blokkene var ferdige var det naturlege neste steget å begynne på tilstandsmaskina som skulle styre sjølve SBR-prosessen
og styringa av anlegget. Vi hadde allereie laga oss eit bilete, men kunne no begynne å bruke kunnskapen frå verkemåten til anlegget
til å grovt fylle inn dei hendelsane og aksjonane som foregikk mellom tilstandsbytter. 
Tilstandsmaskina er byggd opp av dei fem reaktortilstandane som forekommer i eit SBR-anlegg.

Dette er ein simpel model av korleis tilstandsmaskina er programmert men gir eit godt innblikk i systemet.

\begin{figure}[htbp]
    \centering
    \includegraphics[width=1\textwidth]{Figurar/Simpel tilstandsmaskin.png}
    \caption{Simpel model av tilstandsmaskin}\label{fig:reaktorsoner}
\end{figure}

Utifrå det vi allereie hadde lært om anlegget visste vi kva inngangssignalar og logikk som ville gi
`Mottakstank nivå OK' og la tilstandsmaskina avansere ifrå Pause til Innpumping. Denne jobben utførte vi ved å lage ei
funksjonsblokk for kvar tilstand der aktuelle inngang, utgang og logikk skulle samlast.

\newpage

Tilstandsmaskina blei oppdretta som eigen funksjonsblokk noko som gav oss muligheten å bruke den for begge reaktorane.
Tilstandsmaskina er laga med fem inngangar og seks utgangar, og baserer seg på state/case logikk.

Tilstandsmaskina sender ut høg på den respektive utgangen som samsvarer med reaktortilstanden den er i. Dersom tilstandsmaskina får tilbake
OK signal på den respektive tilstandsinngangen avanserer tilstandsmaskina.
Det er også mogleg å hente ut kva tilstand ved hjelp av ein integer verdi (1-5)

\begin{figure}[htbp]
    \centering
    \includegraphics[width=0.4\textwidth]{Bilder/Tilstandsmaskin.png}
    \caption{Tilstandsmaskin implimentert i programmet}\label{fig:reaktorsoner}
\end{figure}



	\section{Oppbygging av programmet}
\thispagestyle{fancy}

- Korleis me brukte IEC Blokker, fbTimer osv i programmet
	\section{Tilstandslogikk}
\thispagestyle{fancy}

- Bilete fbInnpumping
- Forklaring av alle tilstandsblokkene, med bilder og forklaring av logikk


	\section{Utfordringer}
\thispagestyle{fancy}

Vi opplevde noko problematikk rundt det å ha fleire blokker, som styrer samme komponentar da vi skrivar til ein felles global variabel.
Denne variabelen blir då satt true og false i frå fleire plassar i programmet, noko som gjer at variablenen sinn tilstand vil vere tilfeldig, basert på korleis Codesys sin kompilator leser koden.

Dette kan vi løyse ved å bruke ein egen global variabel for kvar blokk, og så skrive til ein funksjonsblokk som styrar den endelege globale variabelen.
Det er fleire plassar i programmet vi møter denne utfordringa, som blant anna med pumpestyring, da kvar reaktor kan styre same pumpe.
Same løysning vil gjelde i denne situasjonen, der vi må lage ei blokk som tar inngangar frå begge pumpene og setter utgangen til riktig tilstand.  

Dette er eit klassisk eksempel der vi har koda noko vi trur fungera optimalt, men under testing så finner vi ut at det ikkje fungera som vi har tenkt.
Løysninga setter nokre føringar for variabelhandtering videre i programmet, og vi står over eit val der valet våra gjer koden noko meir avansert, men vi oppretthelder funksjonaliteten i programmet slik vi opphavleg hadde tenkt.


	% Kapittel 7 Dokumentasjon
	\chapter{Tillrettelegging for programmering}
\thispagestyle{fancy}
\label{sec:5} 

Som tidlegare beskrevet under krav og mål, ønskte vi å bruke ei løysning som ikkje hadde høge lisenskostnadar for vår sluttkunde. 
Sunnfjord kommune var interessert i å ikkje låse seg til ein fast leverandør, men heller ha moglegheita til å ha valet mellom fleire 
leverandørar innan PLS. Dette ilag med moglegheiten til å breie vår eigen kompetanse 
gjorde at vi såg vekk ifrå Siemens TIA-portal, som vi hadde lært igjennom PLS emnet, 
og begynte å sjå i andre endar.
Programmet ønska vi å skrive i Structured Text (ST) men undersøkte også 
typar av grafiske-diagram baserte språk for å lettare vise samanhengar i programmet.

Det var også viktig at alle på gruppa kunne programmere samtidig og at ein felles programmeringsstandard skulle nyttast.
Det var viktig at parallelt arbeid ikkje skulle by på synkroniserings problem og at det fantest ei god løysning for dette.


	\section{Alarmliste}
\thispagestyle{fancy}

Alarmliste, cause og effect
elefanten
\citep{website}
	\input{Tekst/Kapittel7 - Dokumentasjon/7.2 - Interlock.tex}	
	\input{Tekst/Kapittel7 - Dokumentasjon/7.3 - IO-liste.tex}
	\includepdf[pages=1, scale=0.75, pagecommand={\section{Objektliste}\thispagestyle{empty}}]{Appendix/Objektliste_midlertidig.pdf}

%\section{Objektliste}
%\thispagestyle{fancy}

%\includepdf[pages=2-,scale=0.8, pagecommand={\thispagestyle{empty}},fitpaper=true]{Appendix/Objektliste_midlertidig.pdf}

	\section{P-ID}
\thispagestyle{fancy}

Kan kanskje mergest i lag med scd-diagram at det er egen section med kanskje under sections
	\section{Vedlikeholdsmanual}
\thispagestyle{fancy}

Alarmliste, cause \& effect
	\input{Tekst/Kapittel7 - Dokumentasjon/7.7 - SCD-Diagram.tex}
	
	

	% Kapittel 8 Simulering og verifisering
	\chapter{Simulering og verifisering}
\thispagestyle{fancy}

Forklare prossessen med simulering av anlegget. Korleis vi har laga simuleringsverktøyet, simuleringsprossessen i Codesys, osv
	\section{Kontinuerlig simulering}
\thispagestyle{fancy}

Skriv om den kontinuerlege simulasjonen av funksjonsblokkene vi har laga igjennom heile bacheloroppgåva

	
	% Kapittel 9 Diskusjon
	\chapter{Diskusjon}
\thispagestyle{fancy}
Her kan vi starte underlaget for kappittelet
	\section{Vegen videre for anlegget og programmet}
\thispagestyle{fancy}

Sida oppgåva er ein teoretisk løysningsforslag på eit nytt PLS program i frå reinseanlegget på Sande, er det nokre praktiske detaljar som dukkar opp.

\begin{itemize}
    \item Manglar å mappe fysiske inn og utgangar
    \item Gammalt program nyttar ASiMASTER-bus  
    \item Utviding av det eksisterande anlegget med ny teknologi frå Renasys
    \item Manglande gjennomstrøymingsmåler.
\end{itemize}

Da vi forventar at anlegget går igjennom ein større ombygging,  da med tanke på Renasys og Kommunen sin dialog om utviding av det eksisterande anlegget med ny Renasys teknologi, er dette noko som bør vurderast opp mot programmet og om det blir behov for å endre eller legge til noko. 
Dette er eit naturleg punkt å vurdere tilleggsmåla, om dei ønsker å implementere noko av desse.
Det gamle programmet brukar i dag ASiMASTER-bus for tilkopling av ein del av komponentane mot PLS. 
Dette er noko eigar av anlegget må ta stilling til om ein ønsker å fortsette med eller gjere om. Alle dei fleste ventilar er styrt over bus, og om ein ønskjer å bruke ventilar med tilbakemelding på opne og lukke signal, må det nokre ombyggingar til sidan det berre er laga til for utgangar frå PLS ved ventilane.
Styreskapet der den gamle PLS står i, har då noko redusert plass om ein byrjar å utvide anlegget med fleire modular. 
Dette er moment som må takast med videre om ein ønsker å oppgradera anlegget. 



	\section{Vegen vidare for programmet}
\thispagestyle{fancy}


Sida oppgåva er ein teoretisk løysningsforslag på eit nytt PLS program (SKRIVAST OM?), så er det nokre praktiske moment som oppgåva ikkje reflektera.
Programmet 


\begin{itemize}
    \item Manglar å mappe fysiske inn og utgangar
    \item Gammalt program nyttar ASIMASTER-bus
    \item 
    \item   
\end{itemize}

	% I henhold til forrapporten skal vi og ha elektriske teikningar, brukarrettleiing og vedlikehaldsmanual
	
	% Kapittel 10 Konklusjon
	\chapter{Konklusjon}
\thispagestyle{fancy}
Her kan vi starte underlaget for kappittelet
	\chapter{Avslutting}
\thispagestyle{fancy}
Og så avslutta vi alt, og det var fint vær og konge.

\begin{figure}[htbp]
    \centering
    \includegraphics[width=0.5\textwidth]{Bilder/mango.jpg}
    \caption{Mango Ipa We Like}
    \label{fig:Mango-Logo}
\end{figure}
    
	\section{Konklusjon}
\thispagestyle{fancy}
Vi ønsker å gjere godt eit arbeid. Dette medfører at vi ønsker å ta for oss ein mindre del av prosessen for å løyse oppgåva på best mogleg måte.
Alternativet er å ta ein større del t.d. programmering i tillegg til installasjon.
Men med avgrensa tid kan dette medføre at arbeidet ikkje når potensialet vi ønsker.

Oppgåva vil bli løyst kunn teoretisk sjølv om anlegget er fysisk. 
Ved ei teoretisk oppgåva kan vi legge vekk noko av fokuset på sikkerhetsmomenta ved ein ny installasjon,
og heller bruke meir tid på sikker og robust programmering ilag med ein komplett og korrekt dokumentasjonspakke.

Løysningsalternativ to er det alternativet som blir best for oss. 
Anlegget har manglande dokumentasjon, 
og mykje av arbeidet vil være å bygge ein god funksjonsbeskrivelse for å gjere vidare programmeringsarbeid med reinseanlegget enklare.
	\section{Exit Points}
\thispagestyle{fancy}
(Denne kan kanskje flyttes til kap3)
Vi har definert nokon praktiske punkt i oppgåva der vi har moglegheit 
for å naturleg å avslutte arbeidet om ein ser at vi ikkje får nok tid,
eller at vi har tid til overs. 
Det originale stop punktet vårast er definert etter kravspesifikasjonen. Naturleg alternativ stopp punkt.

\begin{itemize}
    \item Etter programmering og før simulering og verifikasjon
    \item Før programmering.
\end{itemize}

Ved ekstra tid, har vi disse tilleggsoppgåver frå arbeidsgivar 
som kan implementerast i den nye styringssystemet.

Undersøke forbetringspotensiale av anlegget:
\begin{itemize}
    \item Temperatursensor
    \item Nivåsensor
    \item Trykksensor (reintvann inn)
    \item Ventiltilbakemeldingar
    \item Oksygenmåling
    \item Mengde måling «overflow»
    \item Frekvensstyring på hovudpumper
    \item Integrere MJK prøvetakar
    \item Energimåling
\end{itemize}

	% Create a partial table of contents for appendices
	% Generate the list of appendices
	% Start of appendices, capture contents for appendices list
	%\begin{appendices}
		% This command initializes the recording of contents for appendices
		%\startcontents[appendices]
		%\phantomsection
		%\addcontentsline{toc}{chapter}{Liste av Appendikser} % Her kan ein endre navn for appendix lista
		%%\chapter{MB Blokk}
%Dette er ein appendiks for MB blokka
%\includepdf[pages=-]{MB Dokumentasjon.pdf}  

\chapter{IEC Blokker}

% MB Blokk
% Include only the first page
\includepdf[pages=1,scale=0.8, pagecommand={\section{Montor binary}\thispagestyle{empty}},fitpaper=true]{Appendix/IEC Blokker/MB Dokumentasjon.pdf}
% Include the rest of the pages, Starts from the second page to the last, Keep the same scale for all pages
\includepdf[pages=2-, scale=0.8, pagecommand={\thispagestyle{empty}}, fitpaper=true]{Appendix/IEC Blokker/MB Dokumentasjon.pdf}

% MA Blokk
\includepdf[pages=1, scale=0.8, pagecommand={\section{Monitor Analogue}\thispagestyle{empty}}, fitpaper=true ]{Appendix/IEC Blokker/MA Dokumentasjon.pdf}
\includepdf[pages=2-,scale=0.8, pagecommand={\thispagestyle{empty}},fitpaper=true]{Appendix/IEC Blokker/MA Dokumentasjon.pdf}

% SBE Blokk
\includepdf[pages=1, scale=0.8, pagecommand={\section{Switch Binary Eletrical}\thispagestyle{empty}}, fitpaper=true ]{Appendix/IEC Blokker/SBE Dokumentasjon.pdf}
\includepdf[pages=2-,scale=0.8, pagecommand={\thispagestyle{empty}},fitpaper=true]{Appendix/IEC Blokker/SBE Dokumentasjon.pdf}

% SBV Blokk
\includepdf[pages=1, scale=0.8, pagecommand={\section{Switch Binary Value}\thispagestyle{empty}}, fitpaper=true ]{Appendix/IEC Blokker/SBV Dokumentasjon.pdf}
\includepdf[pages=2-,scale=0.8, pagecommand={\thispagestyle{empty}},fitpaper=true]{Appendix/IEC Blokker/SBV Dokumentasjon.pdf}


% ------------ Nytt Kapittel
\chapter{Funksjons Blokker}
% Sekvens Blokker 
% FB Pause
\includepdf[pages=1, scale=0.8, pagecommand={\section{FB Pause Sekvens}\thispagestyle{empty}}, fitpaper=true ]{Appendix/Funksjons Blokker/fbPause.pdf}
\includepdf[pages=2-,scale=0.8, pagecommand={\thispagestyle{empty}},fitpaper=true]{Appendix/Funksjons Blokker/fbPause.pdf}
% FB Innpumping
\includepdf[pages=1, scale=0.8, pagecommand={\section{FB Innpumpings Sekvens}\thispagestyle{empty}}, fitpaper=true ]{Appendix/Funksjons Blokker/fbInnpumping.pdf}
\includepdf[pages=2-,scale=0.8, pagecommand={\thispagestyle{empty}},fitpaper=true]{Appendix/Funksjons Blokker/fbInnpumping.pdf}
% FB Reaksjon
\includepdf[pages=1, scale=0.8, pagecommand={\section{FB Reaksjon Sekvens}\thispagestyle{empty}}, fitpaper=true ]{Appendix/Funksjons Blokker/fbReaksjon Dokumentasjon.pdf}
\includepdf[pages=2-,scale=0.8, pagecommand={\thispagestyle{empty}},fitpaper=true]{Appendix/Funksjons Blokker/fbReaksjon Dokumentasjon.pdf}
% FB Sedimentering
\includepdf[pages=1, scale=0.8, pagecommand={\section{FB Sedimentering Sekvens}\thispagestyle{empty}}, fitpaper=true ]{Appendix/Funksjons Blokker/fbSedementering Dokumentasjon.pdf}
\includepdf[pages=2-,scale=0.8, pagecommand={\thispagestyle{empty}},fitpaper=true]{Appendix/Funksjons Blokker/fbSedementering Dokumentasjon.pdf}
% FB Uttapping
\includepdf[pages=1, scale=0.8, pagecommand={\section{FB Uttapping Sekvens}\thispagestyle{empty}}, fitpaper=true ]{Appendix/Funksjons Blokker/fbUttaping Dokumentasjon.pdf}
\includepdf[pages=2-,scale=0.8, pagecommand={\thispagestyle{empty}},fitpaper=true]{Appendix/Funksjons Blokker/fbUttaping Dokumentasjon.pdf}
% FB Slammuttak
\includepdf[pages=1, scale=0.8, pagecommand={\section{FB Slammuttak}\thispagestyle{empty}}, fitpaper=true ]{Appendix/Funksjons Blokker/fbSlamuttak Dokumentasjon.pdf}
\includepdf[pages=2-,scale=0.8, pagecommand={\thispagestyle{empty}},fitpaper=true]{Appendix/Funksjons Blokker/fbSlamuttak Dokumentasjon.pdf}
% 
% Analog Alarm
\includepdf[pages=1, scale=0.8, pagecommand={\section{FB Analog alarm}\thispagestyle{empty}}, fitpaper=true ]{Appendix/Funksjons Blokker/fbAnalogueAlarm Dokumentasjon.pdf}
\includepdf[pages=2-,scale=0.8, pagecommand={\thispagestyle{empty}},fitpaper=true]{Appendix/Funksjons Blokker/fbAnalogueAlarm Dokumentasjon.pdf}
% Digital Alarm
\includepdf[pages=1, scale=0.8, pagecommand={\section{FB Digital alarm}\thispagestyle{empty}}, fitpaper=true ]{Appendix/Funksjons Blokker/fbDigitalAlarm.pdf}
\includepdf[pages=2-,scale=0.8, pagecommand={\thispagestyle{empty}},fitpaper=true]{Appendix/Funksjons Blokker/fbDigitalAlarm.pdf}
% Calculations
\includepdf[pages=1, scale=0.8, pagecommand={\section{FB Kalkuleringer}\thispagestyle{empty}}, fitpaper=true ]{Appendix/Funksjons Blokker/fbCalculations.pdf}
\includepdf[pages=2-,scale=0.8, pagecommand={\thispagestyle{empty}},fitpaper=true]{Appendix/Funksjons Blokker/fbCalculations.pdf}
% Data Processing
\includepdf[pages=1, scale=0.8, pagecommand={\section{FB Data Prossesering}\thispagestyle{empty}}, fitpaper=true ]{Appendix/Funksjons Blokker/fbDataProcessing.pdf}
\includepdf[pages=2-,scale=0.8, pagecommand={\thispagestyle{empty}},fitpaper=true]{Appendix/Funksjons Blokker/fbDataProcessing.pdf}
% Høgbelastningsmodus
\includepdf[pages=1, scale=0.8, pagecommand={\section{FB High Load}\thispagestyle{empty}}, fitpaper=true ]{Appendix/Funksjons Blokker/fbHighLoad.pdf}
\includepdf[pages=2-,scale=0.8, pagecommand={\thispagestyle{empty}},fitpaper=true]{Appendix/Funksjons Blokker/fbHighLoad.pdf}
% Processed Water
\includepdf[pages=1, scale=0.8, pagecommand={\section{FB Processed Water}\thispagestyle{empty}}, fitpaper=true ]{Appendix/Funksjons Blokker/fbProcessedWater Dokumentasjon.pdf}
\includepdf[pages=2-,scale=0.8, pagecommand={\thispagestyle{empty}},fitpaper=true]{Appendix/Funksjons Blokker/fbProcessedWater Dokumentasjon.pdf}
% Sivbed Rotation
\includepdf[pages=1, scale=0.8, pagecommand={\section{FB Sivbed Rotation}\thispagestyle{empty}}, fitpaper=true ]{Appendix/Funksjons Blokker/fbSivbed Rotation Dokumentasjon.pdf}
\includepdf[pages=2-,scale=0.8, pagecommand={\thispagestyle{empty}},fitpaper=true]{Appendix/Funksjons Blokker/fbSivbed Rotation Dokumentasjon.pdf}
% Swap
\includepdf[pages=1, scale=0.8, pagecommand={\section{FB Swap}\thispagestyle{empty}}, fitpaper=true ]{Appendix/Funksjons Blokker/fbSwap Dokumentasjon.pdf}
\includepdf[pages=2-,scale=0.8, pagecommand={\thispagestyle{empty}},fitpaper=true]{Appendix/Funksjons Blokker/fbSwap Dokumentasjon.pdf}
% Time Meter
\includepdf[pages=1, scale=0.8, pagecommand={\section{FB Time Meter}\thispagestyle{empty}}, fitpaper=true ]{Appendix/Funksjons Blokker/fbTimeMeter Dokumentasjon.pdf}
\includepdf[pages=2-,scale=0.8, pagecommand={\thispagestyle{empty}},fitpaper=true]{Appendix/Funksjons Blokker/fbTimeMeter Dokumentasjon.pdf}
% Timer
\includepdf[pages=1, scale=0.8, pagecommand={\section{FB Timer}\thispagestyle{empty}}, fitpaper=true ]{Appendix/Funksjons Blokker/fbTimer.pdf}
\includepdf[pages=2-,scale=0.8, pagecommand={\thispagestyle{empty}},fitpaper=true]{Appendix/Funksjons Blokker/fbTimer.pdf}

% ----------- Nytt Kapittel
\chapter{Funksjoner}
% Volume Rektangel
\includepdf[pages=1, scale=0.8, pagecommand={\section{FC Volum rektangel}\thispagestyle{empty}}, fitpaper=true ]{Appendix/Funksjoner/Volume_Rectangle.pdf}
\includepdf[pages=2-,scale=0.8, pagecommand={\thispagestyle{empty}},fitpaper=true]{Appendix/Funksjoner/Volume_Rectangle.pdf}
% Volume Sylinder
\includepdf[pages=1, scale=0.8, pagecommand={\section{FC Volum Sylinder}\thispagestyle{empty}}, fitpaper=true ]{Appendix/Funksjoner/Volume_Cylinder.pdf}
\includepdf[pages=2-,scale=0.8, pagecommand={\thispagestyle{empty}},fitpaper=true]{Appendix/Funksjoner/Volume_Cylinder.pdf}

%Importering av IO-liste.csv
\label{sec:IOliste}
\includepdf[pages=1, scale=0.8, pagecommand={\chapter{IO liste}\thispagestyle{empty}}, fitpaper=true ]{Appendix/IO/io.pdf}
\includepdf[pages=2-,scale=0.8, pagecommand={\thispagestyle{empty}},fitpaper=true]{Appendix/IO/io.pdf}




% Litt enklere format
%\includepdf[pages=-, pagecommand={\thispagestyle{empty}}]{Appendix/IEC Blokker/MB Dokumentasjon.pdf}
%\section{MA Blokk}
%\includepdf[pages=-, pagecommand={\thispagestyle{empty}}]{Appendix/IEC Blokker/MA Dokumentasjon.pdf}
		% Stop capturing contents and print them
		%\stopcontents[appendices]
	%\end{appendices}


	%Referanseliste (Berre for test, citep må flyttast inn i teksten)
	\clearpage
	\bibliography{Referansar} % Dette er ein referanse til Referansar.bib

	

\end{document} % Dokument slutt


