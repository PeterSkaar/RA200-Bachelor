\section{Kontinuerlig simulering}
\thispagestyle{fancy}

Undervegs i programmeringa har vi gjort kontinuerlege testar og simulering av blokkene vi har laga og
dette har vært ein sentral del av arbeidsmetodane vi har brukt i prosjektet. 

Kontinuerleg simulasjon er små øvingar og tester som ein utfører medan ein arbeider.
Det er ikkje ein testfase med klare start og stopp, men små kontinuerlege testar og sjekkar som gir tilbakemelding
om arbeidet ein har utført følger spesifikasjonane.

I dette prosjektet brukte vi mykje kontinuerleg simulasjon i programmering av IEC blokkene.
Desse blokkene hadde mange funkjonaliteter som ikkje nødvendigvis var avhengig av kvarandre og
det medførte at vi kunne gjere små enkle testar på dei forskjellege områda uten at heile blokka var ferdig. \newline
Som eit eksempel skal RX resete ein utgang Y. Dette blir implimentert og deretter testa ved hjep av simulasjon.
Kva som til slutt skal sette utgang Y høg er ikkje relevant i dette tilfellet, men vi har implimentert 
og testa at ein reset vil fungere.

\newpage