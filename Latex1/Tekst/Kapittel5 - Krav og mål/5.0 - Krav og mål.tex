\chapter{Krav og mål}
\thispagestyle{fancy}
\gls{Renasys} hadde som ynskje at vi etablerte kravspesifikasjonen til oppgåva 
i samråd med \gls{Sunnfjord Kommune}. \newline
Vi hadde eit møte med kommunen der dei kom med nokon konkrete ynskjer til oppgåva
og løysningsforslaget vi hadde presentert.

\begin{enumerate}
    \item Unngå høge lisenskostandar i planlegginga rundt ny PLS og val av programmeringsverktøy.
    \item Oprette ein funksjonsbeskrivelse som forklarer korleis anlegget virker.
    \item Undersøke og planlegge forbetringar av anlegget ved hjelp av ny sensorikk.
\end{enumerate}

%\begin{enumerate}
    %\item Overgang frå system med høge lisenskostandar og utskifting av PLS
    %\item Forbetring og utvikling av ny dokumentasjon, som inkludera ein ny funksjonsbeskrivelse 
    %\item Undersø
%\end{enumerate}

\section{Krav}
Vi tok utgangspunkt i ynskja frå \gls{Sunnfjord Kommune} for å forme kravspesifikasjon vår.
Deretter utarbeida vi ei liste med krav som naturleg delte seg i tre
hovuddelar: dokumentasjon, programmering og simulering med verifisering. 
Denne oppdelinga vart framlagt for oppdragsgivarane og godkjent. 

\begin{enumerate}
    \item Dokumentasjon:
    \begin{itemize}
        \item Dokumentere anlegget og opprette ein funksjonsbeskrivelse.
        \item Dokumentere det nye styresystemet.
    \end{itemize}
    \item Programmering, vi skal:
    \begin{itemize}
        \item Programmere etter den nyoppretta funksjonsbeskrivelsen.
        \item Anvende open kjeldekode og ikkje låse anlegget til ein spesefikk leverandør.
    \end{itemize}
    \item Simulering med verifisering
    \begin{itemize}
        \item Simulere og teste det nye programmet
    \end{itemize}
\end{enumerate}


%, fjernet "som vart gjort i samråd med rettleiar"

%\begin{enumerate}
    %\item Dokumentasjon med detaljert funksjonsbeskrivelse som inneheld:
        %\begin{itemize} 
        %\item Beskriving av anleggets verkemåte
        %\item \gls{blokkdiagram}
        %\item \gls{forrigling}, \gls{alarm}- og \gls{IO}-lister
        %\item Oversikt over objekta
        %\item \gls{PID}
        %\end{itemize}
    %\item Programmering, vi skal:
        %\begin{itemize}
        %\item Bruke den ny funksjonsbeskrivelsen som grunnlag for programmet.
        %\item Anvende open kjeldekode og ikkje låse oss til spesifikke leverandørar.
        %\end{itemize}
    %\item Simulering med verifisering
        %\begin{itemize}
        %\item Utvikle eit simuleringsverktøy for å teste og verifisere programmet
        %\end{itemize}
%\end{enumerate}

%\newpage

\section{Mål}
Utan om kravspesifikasjonane som var avklart med arbeidsgivar satt vi oss
også nokre personlege mål for oppgåva.

\begin{itemize}
    \item Breie vår kompetanse og bruke eit nytt programmeringsprogram.
    \item Programmere eigne funksjonsblokker som anvender relevante industri standarar.
    \item Levere eit program som er lett å annvende og enkelt å vedlikehalde.
    \item Implementere tilstandsmaskin som styringsform.
\end{itemize}

%\section{Ny funksjonalitet og sensorar}
%Følgande funksjonar og sensorar er ikkje nødvendig for programmets grunnleggande verkemåte, men 
%er ønska av \gls{Sunnfjord Kommune} om tiden strekker til.  
%\begin{itemize}
    %\item Temperatur, nivå og trykksensorar(reintvan inn).
    %\item Ventil tilbakemeldingar og oksygenmåling
    %\item Mengdemåling for overlaup og frekvensstyring av pumper
    %\item Integrasjon av \gls{MJK} prøvetakar og energimåling
%\end{itemize}

