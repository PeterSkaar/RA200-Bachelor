\chapter{Krav og mål}
\thispagestyle{fancy}
\gls{Renasys} hadde som krav at vi etablerte ein kravspesifikasjon i samråd med \gls{Sunnfjord Kommune}.
Vi hadde eit møte med kommunen der dei hadde nokon konkrete ønsker til oppgåva.

\begin{enumerate}
    \item Overgang frå system med høge lisenskostandar og utskifting av pls
    \item Forbetring og utvikling av ny dokumentasjon, som inkludera ei ny funksjonsbeskriving 
    \item Klargjering av styresystemet for integrering av ny funksjonalitet og komponentar
\end{enumerate}

\section{Krav}
Vi tok utgangspunkt i ønska frå \gls{Sunnfjord Kommune} for å forme kravspesifikasjon vår.
Deretter utarbeida vi ei liste med krav som naturleg delte seg i tre
hovuddelar: dokumentasjon, programmering og simulering med verifisering. Denne oppdelinga vart framlagt for oppdragsgivarane våra og godkjent av dei. 


%, fjernet "som vart gjort i samråd med rettleiar"

\begin{enumerate}
    \item Dokumentasjon med detaljert funksjonsbeskriving som inneheld:
        \begin{itemize} 
        \item Beskriving av anleggets verkemåte
        \item \gls{blokkdiagram}
        \item \gls{forrigling}, \gls{alarm}- og \gls{IO}-lister
        \item Oversikt over objekta
        \item Elektriske teikningar og \gls{PID}
        \item Vedlikehaldsmanual
        \end{itemize}
    \item Programmering, vi skal:
        \begin{itemize}
        \item Bruke den ny funksjonsbeskriving som grunnlag for programmeringa.
        \item Anvende open kjeldekode og ikkje låse oss til spesifikke leverandørar.
        \item Følge \gls{IEC} 61131\textemdash3 standard for programmering.
        \end{itemize}
    \item Simulering med verifisering
        \begin{itemize}
        \item Utvikle eit simuleringsverktøy for å teste og verifisere programmet
        \end{itemize}
\end{enumerate}

\newpage
\section{Mål}
Vi har satt følgande mål for systemet:

\begin{itemize}
    \item Programmere fire \gls{IEC} blokker: \gls{MB},\gls{MA},\gls{SBE} og \gls{SBV}
    \item Utvikle eit program som er enkelt og vedlikehalde
    \item Implementera tilstandsmaskin for alle prosess sekvensane
\end{itemize}

\section{Ny funksjonalitet og sensorar}
Følgande funksjonar og sensorar er ikkje nødvendig for programmets grunnleggande verkemåte, Men 
er ønska av \gls{Sunnfjord Kommune} om tiden strekker til.  
\begin{itemize}
    \item Temperatur, nivå og trykksensorar(reintvan inn).
    \item Ventil tilbakemeldingar og oksygenmåling
    \item Mengdemåling for overlaup og frekvensstyring av pumper
    \item Integrasjon av \gls{MJK} prøvetakar og energimåling
\end{itemize}

