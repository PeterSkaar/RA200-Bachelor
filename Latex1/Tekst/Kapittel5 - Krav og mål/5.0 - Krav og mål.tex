\chapter{Krav og mål}
\thispagestyle{fancy}
\label{sec: 5}

\gls{Renasys} hadde som ønskje at vi etablerte kravspesifikasjonen til oppgåva 
i samråd med \gls{Sunnfjord Kommune}.\newline
Vi hadde eit møte med kommunen der dei kom med nokre konkrete ønskje til oppgåva
og løysningsforslaget vi hadde presentert.

\begin{enumerate}
    \item Unngå høge lisenskostandar i planlegginga av PLS og val av programmeringsverktøy.
    \item Opprette ein funksjonsbeskrivelse som forklarer korleis anlegget verkar.
    \item Undersøkje og planlegge forbetringar av anlegget ved hjelp av ny sensorikk.
\end{enumerate}

\section{Krav}
Vi tok utgangspunkt i ønska frå \gls{Sunnfjord Kommune} for å danne kravspesifikasjonen.
Deretter utarbeida vi ei liste med krav som naturleg delte seg i tre
hovuddelar: dokumentasjon, programmering og simulering med verifisering. 
Denne oppdelinga vart framlagt for oppdragsgivarane og godkjent. 

\begin{enumerate}
    \item Dokumentasjon:
    \begin{itemize}
        \item Dokumentere anlegget og opprette ein funksjonsbeskrivelse.
        \item Dokumentere det nye styresystemet.
    \end{itemize}
    \item Programmering, vi skal:
    \begin{itemize}
        \item Programmere etter den nyoppretta funksjonsbeskrivelsen.
        \item Anvende open kjeldekode og ikkje låse anlegget til ein leverandør.
    \end{itemize}
    \item Simulering med verifisering
    \begin{itemize}
        \item Simulere og teste det nye programmet
    \end{itemize}
\end{enumerate}


\section{Mål}
Utanom kravspesifikasjonane som var avklart med arbeidsgivar sat vi oss
også nokre personlege mål for oppgåva.

\begin{itemize}
    \item Utvide eigen kompetansen ved bruk av eit nytt programmeringsverktøy.
    \item Programmere eigne funksjonsblokker som anvender relevante industristandardar.
    \item Levere eit program som er lett å anvende og enkelt å vedlikehalde.
    \item Implementere tilstandsmaskin som styringsform.
\end{itemize}

%\section{Ny funksjonalitet og sensorar}
%Følgande funksjonar og sensorar er ikkje nødvendig for programmets grunnleggande verkemåte, men 
%er ønska av \gls{Sunnfjord Kommune} om tiden strekker til.  
%\begin{itemize}
    %\item Temperatur, nivå og trykksensorar(reintvan inn).
    %\item Ventil tilbakemeldingar og oksygenmåling
    %\item Mengdemåling for overlaup og frekvensstyring av pumper
    %\item Integrasjon av \gls{MJK} prøvetakar og energimåling
%\end{itemize}

