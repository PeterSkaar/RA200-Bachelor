\newpage
\section{HØGBELASTNINGSMODUS}

Høgbelastningsmodus blir aktivert ved stor til renning til anlegget. 
Dersom til renningen er større ein anleggets kapasitet i normal drift vil sekvenstidene til reaktorane blir redusert for å auke kapasiteten. 
Alle tider på høgkapasitetsmodus kan endrast i operatørpanelet.
Her er eit eksempel på sekvenstider:

Det tilførte avlaupsvatnet vil i slike situasjonar være svært uttynna, med lave konsentrasjonar av organisk materiale.
Den nødvendige biologiske ned brytningstida (reaksjonstida) kan derfor reduserast. 
Det viktige i slike situasjonar er å behalde sedimenterings¬tida konstant, slik at ein forhindrar slamflukt.

\begin{table}[h]
    \centering
    \begin{tabular}{|l|c|c|}
    \hline
    \rowcolor{myblack} % Assuming 'myblack' is meant to be 'black'
    \textcolor{purewhite}{Delsekvens} & \textcolor{purewhite}{Normal sekvens Minutter} & \textcolor{purewhite}{Høybelastnings sekvens Minutter} \\ \hline
    \rowcolor{lightgray} 1. Innpumping & 45 & 45 \\ \hline
    \rowcolor{purewhite} 2. Reaksjon & 180 & 90 \\ \hline 
    \rowcolor{lightgray} 3. Sedimentering & 90 & 90 \\ \hline
    \rowcolor{purewhite} 4. Uttapping & 30 & 30 \\ \hline
    \rowcolor{lightgray} 5. Pause & 0 - uendelig & 0 - uendelig  \\ \hline
    \end{tabular}
    \caption{Normal og Høgbelastningsmodus tider}\label{table:Normal Og Høgbelastningsmodus}
\end{table}