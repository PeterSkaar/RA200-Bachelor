\newpage
\section{Dokumentasjon}
\thispagestyle{fancy}

Som tidlegare spesifisert i kapittel 3 var ein stor del av oppgåva vår og dokumentere
anlegget og verkemåten til anlegget. Vi bestemte oss for å gjere ei dokumentasjonsfornying,
altså at vi henta alt det som var av tidlegare dokumentasjon og kombinerte det med vår nyerfarte kunnskap.

Vi oppretta ein funksjonsbeskrivelse som bygger vidare på driftsinstruksen som var levert av 
Watercare i 2003. Dokumentet er tiltenkt ein slags bruksanvisning på heile anlegget,
der sikkerheit, prosess, verkemåte og programmering er sentrale tema.

Funksjonsbeskrivelsen skal være forståeleg for alle intresserte parter og inneheld 
mange av våre nye figurar og avsnitt. Dokumentet innheld detaljert kvifor og korleis
ting heng saman og gir derfor ein forståelse for anlegget som ikkje var mogleg med dokumentasjonen som var tilgjengeleg før.

Funksjonsbeskrivelsen inneheld alt om reinseanlegget på Sande men er delt opp i kapittel for å ikkje
overvelde lesar med mykje unødvendig informasjon. Desto lenger ned i dokumentet ein kjem jo meir avansert blir informasjonen,
og programmering av anlegget ligger under udjupa teknisk beskrivelse.

Dokumentet er delt opp i desse kapittela

\begin{enumerate}
    \item Introduksjon
    \item Verkemåte
    \item Teknisk beskrivelse
    \item Drift og vedlikehald
    \item Feilsøking
    \item Utdjupa teknisk beskrivelse
    \item Teknisk underlag
\end{enumerate}

Alt som er nemnt her i kapittel 4 står meir detaljert under kapittel Teknisk beskrivelse i funksjonsbeksrivelsen, og
er sentralt for å best forstå korleis anlegget fungerer.

Funksjonsbeskrivelsen i sin heilhet ligger som vedlegg (SETT INN VEDLEGG)