\chapter{Konklusjon}
\thispagestyle{fancy}

Resultatet av bacheloroppgåva utfyller dei krav og mål som var satt på førehand.
Oppdragsgivar har fått ein ny verdifull dokumentasjon, som beskrivelser teikningsunderlag, komponentkartlegging,
illustrasjonar og generell styring som vil gi betre forståelse av verkemåten anlegget.
Det blir levert eit program som inneheld eit godt utgangspunkt for eit fullverdig og ferdig testa program.

I resultatet ser vi at ikkje alle funksjonar vi ønskte å inkludere på førehand, er representert
Programmet som er levert med oppgåva har moglegheit til å bli det nye styresystemet på reinseanlegget i Sande
dersom ein invisterar noko meir tid til funksjonalitet og testing.

Denne oppgåva har bidratt med å belyse viktigheita med samarbeid, kommunikasjon og planlegging av tid.

