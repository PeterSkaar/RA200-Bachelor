\chapter{Konklusjon}
\thispagestyle{fancy}

%% Skreve meir folkeleg COPILOT
Gjennom denne bacheloroppgåva har vi oppnådd ein djupare forståing av korleis Sande avlaupsreinseanlegg fungerer og kva utfordringar dei står ovanfor. 
Vi har utvikla eit nytt styresystem som sikrar at reinseanlegget følgjer moderne programmeringsstandardar. 
Sjølv om tilnærminga vår har vore teoretisk, har vi lagt eit solid grunnlag for framtidig praktisk implementering.

Ved å velje løysingsalternativ C, har vi unngått å overføre eksisterande feil og har kunne utforske og implementere betre løysingar. 
Dette har ført til eit fleksibelt og teknisk moderne styresystem som vil kunne vera til god nytte for \gls{Sunnfjord Kommune}.

Sjølv om vi ikkje har gjennomført ein fullskala test, har simuleringane vist at systemet fungerer som forventa. 
Vi anbefaler at ytterlegare funksjonalitet og testing blir gjennomført før systemet eventuelt blir tatt i bruk.

Uansett kva \gls{Sunnfjord Kommune} skulle velje å gjere med det nye styresystemet, har dei fått ein ny dokumentasjonspakke 
som samlar viktig informasjon på ein strukturert måte.

Bacheloroppgåva har gitt \gls{Renasys} og \gls{Sunnfjord Kommune} tilgang til funksjonsblokker som kan nyttast i ulike PLS-styresystem. 
Gjennom å nytte \gls{IEC} \gls{PAS} 63131-standarden har vi oppnådd ei løysing som gir oppdragsgivar fleksibilitet og gode verktøy for framtidige behov.

Til slutt har prosjektet understreka kor viktig samarbeid, kommunikasjon og nøye planlegging er.
Vi er takknemlege for moglegheita til å bidra til eit så viktig område som offentleg infrastruktur, og ser fram til å sjå kva arbeidet kan føre til.

