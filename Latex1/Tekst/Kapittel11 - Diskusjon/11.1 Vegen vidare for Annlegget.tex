\section{Vegen videre for anlegget}
\thispagestyle{fancy}

%
% Føler vi driter i heile ASI-Bussen, Er den i det heile relevant??
% Så klasker vi inn eit delkapittel her der vi bygger vidare på ønsker 
% til sunnfjord kommune og å undersøke og planlegge ny sensorikk og
% får vist får Automasjonskompentanse ved å skrive om det.

%Vanskelig første setning
Da vi forventar at anlegget går igjennom ein større ombygging, da med tanke på at Renasys og kommunen er i dialog dialog om utviding 
av det eksisterande anlegget med ny Renasys teknologi, er dette noko som bør vurderast opp mot programmet 
og om det blir behov for å endre eller legge til noko. 

Styreskapet der den gamle PLS står i, har då noko redusert plass om ein byrjar å utvide anlegget med fleire modular. 
Dette er moment som må takast med videre om ein ønsker å oppgradera anlegget.

Videre for å oppnå eit betre anlegg, har vi i lag med oppdragsgivar identifisert fleire komponentar som burde 
vert implementert på sikt. Vi tar ikkje stilling til nye lover og forskrifter som trer i kraft ved oppgradering av anlegget.

Programmet er programmert med at tiltenkte tilleggsmål ikkje skal vere vanskeleg å implementere. 
Men dette forutsette at det blir gjort oppgraderingar på reinseanlegget før det blir programmert. 
Våra tilråding for å skape eit robust program og anlegg er at desse funksjonane er anbefalt å implementere

\subsection{Anbefalingar sensorikk}

- Flowmåler, inntak, til hver reaktor, sivebed og uttapping for kontroll over mengder
- Oksigenmåling for kontroll over reaksjonsfasen
- temperaturmålera i hver reaktor, samt kjellar(forst?)
- energimåling (Oversikt over forbruk)
- ventilar utstyrt med tilbakemeldinger for sikkerhet
- nødstoppbryter i styrestraumen med tilbakemelding til pls for alarmliste
- integrere mjk prøvetaker


LISTE OPP TILLEGSMÅL OG BEGRUNNING
Tillegsmåla som vart satt av oss og oppdragsgivar

% Fjernet gammel tekst
%%Dette er eit naturleg punkt å vurdere tilleggsmåla, om dei ønsker å implementere noko av desse.
%Det gamle programmet brukar i dag ASiMASTER-bus for tilkopling av ein del av komponentane mot PLS. 
%Dette er noko eigar av anlegget må ta stilling til om ein ønsker å fortsette med eller gjere om. Alle dei fleste 
%ventilar er styrt over bus, og om ein ønskjer å bruke ventilar med tilbakemelding på opne og lukke signal, 
%må det nokre ombyggingar til sidan det berre er laga til for utgangar frå PLS ved ventilane.


\newpage

