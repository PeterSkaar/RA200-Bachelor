\section{Vegen videre for anlegget og programmet}
\thispagestyle{fancy}



Da vi forventar at anlegget går igjennom ein større ombygging,  da med tanke på Renasys og Kommunen sin dialog om utviding av det eksisterande anlegget med ny Renasys teknologi, er dette noko som bør vurderast opp mot programmet og om det blir behov for å endre eller legge til noko. 
Dette er eit naturleg punkt å vurdere tilleggsmåla, om dei ønsker å implementere noko av desse.
Det gamle programmet brukar i dag ASiMASTER-bus for tilkopling av ein del av komponentane mot PLS. 
Dette er noko eigar av anlegget må ta stilling til om ein ønsker å fortsette med eller gjere om. Alle dei fleste ventilar er styrt over bus, og om ein ønskjer å bruke ventilar med tilbakemelding på opne og lukke signal, må det nokre ombyggingar til sidan det berre er laga til for utgangar frå PLS ved ventilane.
Styreskapet der den gamle PLS står i, har då noko redusert plass om ein byrjar å utvide anlegget med fleire modular. 
Dette er moment som må takast med videre om ein ønsker å oppgradera anlegget. 

\textbf{SKRIVE EIT AVSNITT PÅ TILRÅDINGAR TIL ANLEGGET FRAMOVER I TID}
        Skrive noe om at anlegget burde ha verifisering av parametera, slik at 
            ikkje operatør setter ulovlige verdier.
        Skrive om at kanskje det hadde vert en lur ide med en transit i mellom sekvensane, å sjekke tilstanden på objekta før den går videre

