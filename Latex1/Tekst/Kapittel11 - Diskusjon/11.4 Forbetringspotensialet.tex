\section{Lærdom og forbetringspotensial}
\thispagestyle{fancy}

Oppgåva har bydd på ei bratt læringskurve og når vi ser tilbake på arbeidet ser vi fleire moglegheiter til forbetring.

Etter læring og dokumentering av anlegget var vi ivrige på å byrje med programmering.
Vi valde å starte fortløpande og så lære litt undervegs.
Dette gjorde at vi på fleire plasser var nøydd til å ta ting oppatt grunna at vi hadde misforstått standardane eller kvarande.
I etterpåklokskapenslys skulle vi ha lest meir om \gls{IEC} blokkene og tatt oss betre tid før vi byrja å programmere.

Vi kan også sjå i etterkant av programmeringa at vi kunne ha nytta funksjonalitetane til \gls{IEC} blokkene betre.
Med ordentleg bruk av alle dei funksjonalitetane som vi programmerte kunne ein moglegvis ha redusert tilstandslogikk
til eit par logiske setningar og moglegvis fjerna dei i sin heilheit.

Arbeidet med namngiving på inngangar, utgangar og parameter hadde også moglegheit for forbetring.
I starten var vi flinke å sette av den naudsynte tida for å få namn korrekt, men ettersom tida blei knappare utover i oppgåva
blei namngivinga dårlegare. Dette gjorde at vi var nøydd å ta alle desse namna oppatt. \newline
Vi skulle også sett oss ned i forkant å diskutert kva språk kommentarar, namn og forklaringar skulle vere i programmet.

Vi ser også at vi moglegvis sette oss litt for detaljert inn i reinseanleggets verkemåte.
Sjølv om denne delen av oppgåva var vesentleg, ser vi no at noko av arbeidet vi brukte tid på kanskje ikkje var naudsynt.

Gjennom denne bacheloroppgåva har vi henta inn mykje ny kunnskap og erfaring.
Vi har blandt anna lært og utdjupa oss innan:

\begin{itemize}
    \item Offentleg infrastruktur
    \item Kundekommunikasjon
    \item Programmering av større styresystem i \gls{PLS}
    \item Programmering av \gls{PLS} i \gls{Codesys}
    \item Standardane \gls{IEC} 61131-3 og \gls{IEC} \gls{PAS} 63131
    \item \Gls{github} intergrering
    \item Rapportskriving i \gls{latex}
\end{itemize}






