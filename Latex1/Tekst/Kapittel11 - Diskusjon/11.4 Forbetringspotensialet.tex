\section{Lærdom og forbetringspotensial}
\thispagestyle{fancy}

Oppgåva har bydd på eit bratt lærekurve og når vi ser tilbake på arbeidet som er utført
ser vi fleire moglegheitar for forbetring.

Etter læring og dokumenteringa av anlegget var vi ivrige på å byrje med kodinga.
Vi valgte å starte fortløpade og så lære litt undervegs.
Dette gjorde at vi på fleire plasser var nødt til å ta ting oppigjen grunna at vi hadde missforstått standarar og enkelte gangar missforstått kvarande.
I etterklokspaenslys skulle vi gjerne ha lest litt meir om \gls{IEC} blokkene og tatt litt betre tid før vi byrja å programmere dei.

Vi kan også sjå i etterkant av programmeringa at vi kunne ha nytta funksjonalitetane til IEC blokkene betre.
Med ordentleg bruk av alle dei funksjonalitetane som vi programerte kunne ein kanskje ha redusert all tilstandslogikk
til eit par logiske setningar og kanskje fjerna den i sin heilheit.

Navngiving på inngangar, utgangar og parameter har også moglegheit for forbetring.
I starten var vi flinke og satt av den nødvendige tida for å få navn korrekt, men ettersom tida blei knappare utover i oppgåva
blei namngivinga dårlegare. Dette gjorde at vi var nøyd å ta alle desse namna igjen. \newline
Vi skulle også satt oss ned på forkant av programmering å blitt enige om kva språk kommentarar, navn og forklaringar skulle komme på.
Her var vi ikkje konsekvente på, og programmet endte opp med litt bokmål, litt nynorsk og ein del engelsk.

Vi ser også at vi gikk litt for detaljert i reinseanleggets virkemåte.
Sjølv om denne delen av oppgåva var vesentlig ser vi no at noko av arbeidet vi brukte tid på kanskje ikkje var naudvendig.

Igjennom denne bacheloroppgåva har vi henta inn mykje ny kunnskap og erfaring.
Oppgåva har gitt oss utfordringar og læringskurver vi før ikkje trudde var moglege.
Vi har blandt anna lært og utdjupa oss innan:

\begin{itemize}
    \item Offentleg infrastruktur
    \item Kunde kommunikasjon
    \item Programmering av større styresystemer i PLS
    \item Programmering av PLS i Codesys
    \item IEC 61131-3 og IEC PAS 63131
    \item Github intergrering
    \item Rapportskriving i Latex
\end{itemize}

