\section{Lærdom og forbetringspotensial}
\thispagestyle{fancy}

Oppgåva har bydd på ei bratt læringskurve og når vi ser tilbake på arbeidet ser vi fleire moglegheiter til forbetring.

Etter læring og dokumentering av anlegget var vi ivrige på å byrje med programmering.
Vi valde å starte fortløpande og så lære litt undervegs.
Dette gjorde at vi på fleire plasser var nøydd til å ta ting oppatt grunna at vi hadde misforstått standardane eller kvarande.
I ettertid innser vi at vi burde ha lest meir om \gls{IEC} blokkene og tatt oss betre tid før vi byrja å programmere.

I etterkant av programmeringa ser vi at vi kunne ha utnytta funksjonaliteten til IEC-blokkene betre. 
Ved full bruk av alle dei funksjonane vi implementerte, kunne vi potensielt ha forenkla tilstandslogikken til berre eit par logiske setningar, 
og kanskje fjerna den i sin heilheit.

Ved å optimalisere bruk av blokkene kunne programmet vore betre og meir effektivt. 
Dersom det skulle vere naudsynt med nyprogrammering eller endringar, ville vi ha gjort ting anleis og utnytta all funksjonalitet.
Blokkene kunne bidratt til betre handtering av forriglingar, ved t.d. bruk av FSL, FSH, FDH, og FDL inngangar. \newline
Undersøkt forbetring av varsling og generell alarmreduksjon ved bruk av ``blocking'' og ``suppression'' for komponentar som ikkje er relevante i gjeldande tilstand.
Vi ville også ha implementert ny bruk av inngang XE på \gls{SBE} t.d. for termiske motorvern, og nytta aktivering frå tilstandsmaskin på ein anna måte.

Arbeidet med namngiving på inngangar, utgangar og parameter hadde også moglegheit for forbetring.
I starten var vi flinke å sette av den naudsynte tida for å få namn korrekt, men ettersom tida blei knappare utover i oppgåva
blei namngivinga dårlegare. Dette gjorde at vi var nøydd å ta alle desse namna oppatt. \newline
Vi skulle også sett oss ned i forkant å diskutert kva språk kommentarar, namn og forklaringar skulle vere i programmet.

Vi ser også at vi moglegvis sette oss litt for detaljert inn i reinseanleggets verkemåte.
Sjølv om denne delen av oppgåva var vesentleg, ser vi no at noko av arbeidet vi nytta tid på kanskje ikkje var naudsynt.

Gjennom denne bacheloroppgåva har vi henta inn mykje ny kunnskap og erfaring.
Vi har blandt anna lært og utdjupa oss innan:

\begin{itemize}
    \item Offentleg infrastruktur
    \item Kundekommunikasjon
    \item Programmering av større styresystem i \gls{PLS}
    \item Programmering av \gls{PLS} i \gls{Codesys}
    \item Standardane \gls{IEC} 61131-3 og \gls{IEC} \gls{PAS} 63131
    \item \Gls{github} intergrering
    \item Rapportskriving i \gls{latex}
\end{itemize}






