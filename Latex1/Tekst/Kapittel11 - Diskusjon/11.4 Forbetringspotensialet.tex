\section{Lærdom og forbetringspotensial}
\thispagestyle{fancy}

Oppgåva har bydd på eit bratt læringskurve og når vi ser tilbake på arbeidet som er utført, 
så ser vi fleire moglegheiter til forbetring.

Etter læring og dokumenteringa av anlegget var vi ivrige på å byrje med programmeringa.
Vi valde å starte fortløpande og så lære litt undervegs.
Dette gjorde at vi på fleire plasser var nøydd til å ta ting opptatt grunna at vi hadde misforstått standardane og enkelte gangar misforstått kvarande.
I etterpåklokskapenslys skulle vi gjerne ha lest meir om \gls{IEC} blokkene og tatt oss betre tid før vi byrja å programmere dei.

Vi kan også sjå i etterkant av programmeringa at vi kunne ha nytta funksjonalitetane til \gls{IEC} blokkene betre.
Med ordentleg bruk av alle dei funksjonalitetane som vi programmerte kunne ein kanskje ha redusert all tilstandslogikk
til eit par logiske setningar og kanskje fjerna dei i sin heilheit.

Namngiving på inngangar, utgangar og parameter har også moglegheit for forbetring.
I starten var vi flinke å sette av den naudsynte tida for å få namn korrekt, men ettersom tida blei knappare utover i oppgåva
blei namngivinga dårlegare. Dette gjorde at vi var nøydd å ta alle desse namna oppatt. \newline
Vi skulle også sett oss ned i forkant å diskutert kva språk kommentarar, namn og forklaringar skulle komme på i programmet.

Vi ser også at vi sette oss litt for detaljert inn i reinseanleggets verkemåte.
Sjølv om denne delen av oppgåva var vesentleg, ser vi no at noko av arbeidet vi brukte tid på kanskje ikkje var nødvendig.

Igjennom denne bacheloroppgåva har vi henta inn mykje ny kunnskap og erfaring.
Oppgåva har gitt oss utfordringar og ei læringskurve vi før ikkje trudde var mogleg.
Vi har bl.a. lært og utdjupa oss innan:

\begin{itemize}
    \item Offentleg infrastruktur
    \item Kundekommunikasjon
    \item Programmering av større styresystem i \gls{PLS}
    \item Programmering av \gls{PLS} i \gls{Codesys}
    \item \gls{IEC} 61131-3 og \gls{IEC} \gls{PAS} 63131
    \item \Gls{github} intergrering
    \item Rapportskriving i \gls{latex}
\end{itemize}






