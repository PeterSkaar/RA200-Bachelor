\section{Vegen vidare for programmet}
\thispagestyle{fancy}

Programmeringsarbeidet hadde eit så stort omfang at vi ikkje klarte å ferdigstille prosjektet.
Etter å ha innsett at vi ikkje ville nå målet, måtte vi gjere nokre val angåande prioritering av ressursar. 
Vi valde å fokusere på å gjennomføre ei god samla simulering i staden for å arbeide vidare med ferdigstilling av funksjonar.

\subsection{Programmering}

Vegen vidare for programmet er ferdigstilling av funksjonane som ikkje er komplette, og 
implementering av funksjonar som er oppretta men ikkje tatt i bruk. 
Det gjennstår arbeid på feilhandtering og korleis anlegget skal reagere på spesifikke feilsituasjonar.

Her er punkter vi ønskjer å arbeide vidare med.

\begin{itemize}
    \item Styring av pumpehuset er forberedt men ikkje programmert. 
    \item Ferdigstilling av høgbelastningsmodus.
    \item Spesefikk feilhandtering
\end{itemize}

\subsection{Simulering}

Det gjennstår framleis noko simulering og testing før ein kan ansjå programmet ferdig verifisert.
Til dømes er funksjonsblokkene for slamuttapping og sivbedrotasjon testa separat, men ikkje simulert saman med
tilstandsmaskinene og resterande program.

Under testing av eit slikt program bør ein aktivt leite og prøve å identifisere feil og manglar for å gjere programmet best mogleg.
For blokkparameterer manglar eksempelvis nokre beskyttelsar mot ulovlege parameterverdiar.

Ved å investere meir tid i vidare simuleringsfase sikrar vi at programmet oppfyller forventningane til \gls{Sunnfjord Kommune} og at ein eventuell
praktisk innkøyringsfase går effektivt. \newline
Programmet er no sett opp i simuleringsmodus, med simuleringsblokker og endra parameterverdiar, 
og det må tilbakestillast til original tilstand før det kan nyttast. 




\newpage