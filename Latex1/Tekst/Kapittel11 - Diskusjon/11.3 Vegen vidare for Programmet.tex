\section{Vegen vidare for programmet}
\thispagestyle{fancy}

Programmeringsarbeidet hadde eit så stort omfang at vi ikkje klarte å ferdigstille programmet.
Etter å ha innsett at vi ikkje ville nå målet, måtte vi gjere nokre val angåande prioriteringa av ressursane våre. 
Vi valde å fokusere på å gjennomføre ei god samla simulering av programmet i staden for å arbeide vidare med ferdigstilling av funksjonar.

\subsection{Programmering}

Vegen vidare for programmet er først og fremst ferdigstilling av funksjonane som ikkje er komplette, og 
implementering av funksjonar som er laga til men ikkje tatt i bruk. 
Det står også at arbeid på feilhandtering og korleis anlegget skal reagere på spesifikke feilsituasjonar.

Her er punkter vi ønskjer å arbeide vidare med.

\begin{itemize}
    \item Styring av pumpehuset er forberedt men ikkje programmert. 
    \item Ferdigstilling av høgbelastningsmodus.
    \item Generell feilhandtering
    \item Automatisk prøvetaking er ikkje blitt programmert.
\end{itemize}

\subsection{Simulering}

Det gjennstår framleis noko simulering og testing av programmet før ein kan ansjå det som ferdig verifisert.
Til dømes er funksjonsblokkene for slamuttapping og sivebedrotasjon testa separat, men ikkje simulert i sin heilskap saman med
tilstandsmaskinene og resterande program.

Under testing av eit slikt program som dette bør ein aktivt leite og prøve å identifisere feil og manglar for å gjere programmet best mogleg.
For blokkparameterer manglar eksempelvis fleire beskyttelsar mot ulovlege parameterverdiar, noko som vil bli oppdaga gjennom meir omfattande simulering.

Ved å investere meir tid i vidare simuleringsfase sikrar vi at programmet oppfyller forventningane til \gls{Sunnfjord Kommune} og at ein eventuell
praktisk innkøyringsfase går effektivt. \newline
Programmet er no sett opp i simuleringsmodus, og det må tilbakestillast til original tilstand før det kan brukast. 
Simuleringsmodus har endra fleire parameterverdiar og er kopla opp i mot simuleringsblokker.  



\newpage