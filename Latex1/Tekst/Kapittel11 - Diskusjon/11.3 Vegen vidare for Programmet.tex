\section{Vegen vidare for programmet}
\thispagestyle{fancy}

Programmeringsarbeidet hadde eit for stort omfang til at vi klarte å ferdigstille programmet. 
Vi måtte ta nokre val angående prioriteringen av ressursene våre etter vi innsåg at vi ikkje ville komme heilt i mål.
Vi valte å fokusere på å få til ei god samla simulering av programmet istadenfor å arbeide vidare med ferdigstilling av funksjonar.

\subsection{Programmering}

Vegen vidare for programmet er først og fremst ferdigstilling av fleire av funksjonane som ikkje er komplette og 
implementering av funksjonar som er laga til men ikkje tatt i bruk. 
Det gjennstår også arbeid på feilhandtering og korleis anlegget skal aggere på spesefikke feilsituasjonar.

Følgande kjem punkter vi ynskjer å arbeide vidare med.

\begin{itemize}
    \item Styring av pumpehuset er forbredt men ikkje programmert. 
    \item Ferdigstilling av høgbelastningsmodus.
    \item Generell Feilhandtering
    \item Automatisk prøvetaking er ikkje blitt programmert.
    %\item Implementere beskyttelse av parametere.
\end{itemize}

% Begge desse er vel forsåvidt berre MB blokker inn, så må det skje ting i spesefikk feilhantering istaden
%\item Sensor for vann på golv er ikkje programmert.
%\item Tilbakemelding frå pressostat på kompressor er ikkje programmert.

\subsection{Simulering}

Det gjennstår forsatt mykje simulering og testing av programmet før ein kan ansjå det som ferdig verifisert.
Eksempelvis er funksjonsblokkene for slamuttapping og sivebedrotasjon testa seperat, men ikkje simulert i heilhet med
tilstandsmaskinene og resterande program.

I testing av eit program som dette bør ein aktivt leite og prøve å finne feil og manglar for å gjere programmet best mogleg.
For blokkparameterer manglar eksempelvis fleire beskyttelsar mot ulovlege parameterverdiar, men slik vil ein oppdage ved meir simulering.

Ved å invistere meir tid i vidare simuleringsfase sikrar vi at programmet oppfyller forventningane til Sunnfjord kommune og at ein eventuell
praktisk innkøyringsfase går effektivt. \newline
Programmet er no satt opp i simuleringsmodus, og det må tilbakestillast til original tilstand før det kan brukast. 
Simuleringsmodus har endra fleire parameterverdiar og er kopla opp i mot simuleringsblokker.  


\newpage