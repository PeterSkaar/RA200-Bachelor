\section{Måloppnåing}
\thispagestyle{fancy}

\subsection{Krav}
Når ein tar eit tilbakeblikk på kravspesifikasjonen under kapittel \ref{sec: 5} sit vi med ein god følelse.
Krava som vart godkjend i lag med oppdragsgivar var relativt opne, og 
mykje av grunnen til dette var at oppgåva ikkje var utlyst til oss, men at vi etterlyste ei oppgåve i frå \gls{Renasys}.
Oppgåva hadde dermed ingen ferdigdefinerte krav eller mållinjer, som gjorde at vi kunne definere oppgåva slik vi ville.
Sjølv om krava var opne, føler vi at vi har svart på dei punkta som var essensielle og levert det som var forventa.

Det er og eit faktum at vi ikkje kom i mål med alt arbeidet og at enkelte detaljer av både dokumentasjon, program og simulering
kunne våre betre og meir utfyllande. Men med den tida vi hadde til rådigheit, føler vi at vi har levert etter krav.

\subsection{Mål}
Våre personlege mål under oppgåva dreia seg mest om å lære nye program og auke vår kompetanse. Under dette prosjektet har vi lært mykje nytt. 
Programmeringsdelen av oppgåva har bydd på opplæring i nye program, samt pugging og lesing av standard og normer.
Alle delane av denne bacheloroppgåva har bydd på utfordringar og meistring, og når vi ser tilbake på dei personlege måla våre er vi godt fornøgde.

\subsection{Forprosjekt}
Ein viktig del av slutten på eit prosjekt er å sjå tilbake på starten.
Klarte vi å legge ein god plan og klarte vi å følge planen vi la?
Dette er sentrale spørsmål ein er nøydt å ta stilling til dersom ein ynskjer å lære mest mogeleg.

Når vi samanliknar forprosjektet og bacheloroppgåva, ser vi god gjenspegling. 
I løpet av arbeidsprosessen og etterkvart som me fekk meir kunnskap, vart det naturleg å gjere nokre justeringar og tilpassingar. 
Likevel kjenner vi at forprosjektet la ein solid plan for oss, og at me har følgt denne planen i utviklinga av bacheloroppgåva



\newpage