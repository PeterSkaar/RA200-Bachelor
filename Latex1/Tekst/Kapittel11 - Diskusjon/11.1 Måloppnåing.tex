\section{Måloppnåing}
\thispagestyle{fancy}

\subsection{Krav}
Når ein tar eit tilbakeblikk på kravspesifikasjonen i kapittel 5 sit vi igjen med ein god følelse.
Krava som vart godkjent ilag med oppdragsgivar var relativt opne og 
mykje av grunnen til dette er at oppgåva ikkje var utlyst til oss, men at vi etterlyste ei oppgåve ifrå Renasys.
Oppgåva hadde dermed ingen ferdigdefinerte krav eller mållinjer noko som gjorde at vi kunne definere oppgåvå slik vi ville.
Sjølv om krava var opne føler vi at vi har svart på dei punkta som var essensielle og levert det som var forventa.

Det er også eit faktum at vi ikkje kom i mål med alt arbeidet og at enkelte detaljer av både dokumentasjon, program og simulering
kunne blitt forbetra og utvida, men med den tida vi hadde føler vi at vi har levert etter krav.

\subsection{Mål}
Våre personlege mål under oppgåva dreia seg mykje om læring og å utfordre det nye, og under dette prosjektet har vi lært mykje nytt. 
Programmeringsdelen av oppgåva har bydd på opplæring i nye programmer og pugging og lesing av normer og standarar.
Alle delane av denne bacheloroppgåva har bydd på utfordringar og meistring og når vi ser tilbake på dei personlege måla våre er vi veldig fornøyde.

\subsection{Forprosjekt}
Ein viktig del av slutten på eit prosjekt er å sjå tilbake på starten.
Klarte vi å legge ein god plan og klarte vi å følge planen vi la?
Dette er sentrale spørsmål ein er nødt å ta stilling til dersom ein ynskjer å lære mest mogeleg.

Når vi samanliknar forprosjektet og bacheloroppgåva, ser vi ein god gjenspegling av kvarandre. 
I løpet av arbeidsprosessen og etterkvart som me fekk meir kunnskap, vart det naturleg å gjere nokre justeringar og tilpassingar. 
Likevel kjenner vi at forprosjektet la ein solid plan for oss, og at me har følgt denne planen i utviklinga av bacheloroppgåva


\newpage