\section{Måloppnåing}
\thispagestyle{fancy}

\subsection{Krav}
Når ein tar eit tilbakeblikk på kravspesifikasjonen skildra under kapittel \ref{sec: 5} sit vi med ein god følelse.
Krava som vart godkjend saman med oppdragsgivar var relativt opne.
Mykje av grunnen til dette var at vi etterlyste oppgåva frå \gls{Renasys} og at det ikkje var planlagt noko kravspesifikasjon i forkant. \newline
Oppgåva hadde dermed ingen ferdigdefinerte krav eller mållinjer, som gjorde at vi kunne definere oppgåva slik vi ville.

Det er og eit faktum at vi ikkje kom i mål med alt arbeidet og at enkelte detaljer av dokumentasjon, program og simulering
kunne vore betre med meir tid.
Vi føler uansett at vi har svart på dei punkt som var essensielle og levert over det som var forventa.



\subsection{Mål}
Våre personlege mål under oppgåva handla om å lære nye program og auke eigen kompetanse. Under dette prosjektet har vi lært mykje nytt. 
Programmeringsdelen av oppgåva har bydd på læring i nye program, samt pugging og lesing av standard og normer.
Alle delane av oppgåva har bydd på utfordringar og meistring, og når vi ser tilbake personlege mål, er vi godt nøgde.

\subsection{Forprosjekt}
Ein viktig del av slutten på eit prosjekt er å sjå tilbake på starten.
Klarte vi å legge ein god plan og klarte vi å følgje planen vi la?
Dette er sentrale spørsmål ein er nøydd å ta stilling til dersom ein ønsker å lære mest mogeleg.

Når vi samanliknar forprosjektet og bacheloroppgåva, ser vi god gjenspegling. 
I løpet av arbeidsprosessen og etterkvart som vi fekk meir kunnskap, var det naturleg å gjere nokre justeringar og tilpassingar. 
Likevel kjenner vi at forprosjektet la ein solid plan for oss, og at vi har følgt denne planen i arbeidet på bacheloroppgåva.

Rapport for forprosjekt er tilgjengeleg via vedlegg. (Vedlegg K)

\newpage