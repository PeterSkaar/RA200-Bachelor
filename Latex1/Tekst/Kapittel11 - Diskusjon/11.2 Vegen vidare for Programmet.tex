\section{Vegen vidare for anlegget og programmet}
\thispagestyle{fancy}

Programmeringsarbeidet hadde eit for stort omfang til at vi klarte å ferdigstille programmeringa. 
Vi måtte ta nokre val om kor vi skulle prioritere ressursane våra videre, da vi såg undervegs at vi ikkje kom i mål. 
Vi valte å fokusere på å få til ei god samla simulering av programmet istedenfor å arbeide vidare med ferdigstilling av funksjonar.
Denne fasen er veldig kritisk for å dokumentere og verifisere våra program, da det dukkar opp ukjente problemstillingar under denne fasen som er viktig å handtere.

Vegen vidare for programmet er først og fremst ferdigstilling av de funksjonane som ikkje er laget og implementering av funksjonar som ikkje har blitt implementert. 
Desse er som følger

\begin{itemize}
    \item Pumpehus er ikkje blitt programmert. 
    \item Slamuttapping er laget til, men ikkje implementert.
    \item Sivebedrotasjon er laget til, men ikkje implementert.
    \item Høgbelastning er laget til, men ikkje implementert.
    \item Sensor for vann på golv er ikkje programmert.
    \item Tilbakemelding frå pressostat på kompressor er ikkje programmert.
    \item PULS MJK er ikkje blitt programmert.
    \item Gjennomgang av alarmliste med eigar av annlegget.
   % \item Vurdering av implementering av tilleggsmål
    \item Implementere beskyttelse av parametere.
\end{itemize}

Anlegget har ikkje noko beskyttelse mot inntasting av ulovlege parameterverdiar.
Det er ein kjent problemstilling at ein operatør som har moglegheit til å stille på desse parameterane kan vere uheldig å sette ein parameter som ikkje gir meining. 
Det er derfor viktig å lage til ein beskyttelse mellom HMI og programmet slik at ein ikkje klarar dette. 
Det kan implementerast ved at ein sjekkar dei parameterane som blir tasta inn er innafor ein gyldig range, før ein sendar di videre til programmet. 

Programmet står nå satt opp i simuleringsmodus, og det må tilbakestillast til original tilstand før det kan brukast. 
Simuleringsmodus har endra alle parameterane og er kopla opp i mot simuleringskode.  

\newline