\section{Vegen videre for anlegget}
\thispagestyle{fancy}

\subsection{Ombygging}

Dersom den teoretiske bacheloroppgåva vår skal realiserast i praksis trenger anlegget fysiske oppgraderingar.
Reinseanlegget idag oppnår ikkje de




Da vi forventar at anlegget går igjennom ein større ombygging, da med tanke på at Renasys og kommunen er i dialog om utviding 
av det eksisterande anlegget med ny Renasys teknologi, er dette noko som bør vurderast opp mot programmet 
og om det blir behov for å endre eller legge til noko. 

Styreskapet der den gamle PLS står i, har då noko redusert plass om ein byrjar å utvide anlegget med fleire modular. 
Dette er moment som må takast med videre om ein ønsker å oppgradera anlegget.

Videre for å oppnå eit betre anlegg, har vi i lag med oppdragsgivar identifisert fleire komponentar som burde 
vert implementert på sikt. Vi tar ikkje stilling til nye lover og forskrifter som trer i kraft ved oppgradering av anlegget.

Programmet er programmert med at tiltenkte tilleggsmål ikkje skal vere vanskeleg å implementere. 
Men dette forutsette at det blir gjort oppgraderingar på reinseanlegget før det blir programmert. 
Våra tilråding for å skape eit robust program og anlegg er at desse funksjonane er anbefalt å implementere

- nødstoppbryter i styrestraumen med tilbakemelding til pls for alarmliste
- integrere mjk prøvetaker

\newpage

\subsection{Anbefalingar sensorikk}

For å betre anlegget og styresystemet ytterlegare vil det være nyttig å få inn meir instrumentering. 
Auka sensorikk vil forbetre styresystemet ved å samle meir og nøyaktig data om tilstanden og ytelsen til systemet. 
Med meir data vil styresystemet bli meir automatisk og vil kunne tilpassast varierande forhold meir effektivt. 
Systemet vil og ha moglegheita til å oppdage avvik og agere tidlegare utan manuelle inngrep, noko som igjen aukar automasjonsevna.

Basert på vår nyerfarte kunnskap av annlegget og styresystemet presenterer vi ein
anbefaling på oppgradering av instrumentering som ville forbetre reinseanlegget. 
Sjølv om all sensorikk vil være verdifull, har vi vurdert kost-nytte og anbefaler kun sensorikk som vil gi tilstrekkeleg verdi.

\begin{itemize}
    \item \textbf{Strøymingsmålar} \newline
        Strøymingmålar i anlegget vil gi nøyaktige tall på mengder vatn som er i anlegget.
        Dette vil gi bedre kontroll på aktivering av høgbelastningsmodus og kontroll og rapportering av driftsdata.
        Fleire strøymningsmålarar er moglege men minimumsanbefaling er vatn inn og ut av anlegget.
    \item \textbf{Energimåling} \newline
        Måling av energi vil gi moglegeit å analyse energiforbruket for å redusere kostnadar og effektivisere prosessar.
        Komponentspesefikk energimåling kan også brukast til overvåking av utstyr for å oppdage slitasje og feil.
    \item \textbf{Reaktormålingar} \newline
        Oksygen, PH og temperatur er alle kritiske verdiar for å oppnå god biologisk reinsing i reaktorane. \newline
        Ved å ha kontroll på desse parameterane vil reaktoren kunne finjusterast for å effektivitere den biologiske reinseprosessen.
    \item \textbf{Tilbakemeldingar} \newline
        Anlegget har begrensa tilbakemeldingar på utstyr, spesielt ingen innan ventilstyring.
        Tilbakemelding er essensielt for prosesstyring, feiloppdagelse og sikkerheit.
        Utan tilbakemeldingar er reinseanlegget meir sårbar for feil som ikkje blir oppdaga før det er for seint 
        og slike situasjonar vil kunne resultere i nedetid, overlaup og andre utilsikta hendelser.
    \item \textbf{Forbetra nivåmåling} \newline
        Ei forbetring av dei akutelle nivåmålingane i anlegget vil gi meir nøyaktige mål på slammengde og slamnivå i reaktorane.
        Det vil då være mogleg å finne det eksakte skilljet mellom slam og reinsa vatn etter ein sedimenteringssekvens som igjen
        gir bedre oversikt over tilstanden til den biologiske reinseprosessen.
    \item \textbf{Oppløysning} \newline
        Oppløysning på analoge målingar er idag kunn 0-1000 bit på 4-20mA.(Appendix xXx) \newline
        Oppgradering av oppløysning vil gjere kvar eksisterande og nye analoge målingar
        meir nøyaktig som igjen gir moglegeit for betre styring og regulering.
\end{itemize}
\newpage

