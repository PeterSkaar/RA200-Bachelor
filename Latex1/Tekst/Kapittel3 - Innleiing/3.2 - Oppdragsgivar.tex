\section{Oppdragsgivar}
\textbf{Renasys AS i samarbeid med Sunnfjord kommune}

Renays\citep{Renasys} kan skildrast som ein innovativ og nyskapande oppstarts bedrift som arbeider med banebrytande teknologi innan mekanisk finpartikkelfiltrering av avlaupsvatn.
Bedrifta har 15 tilsette fordelt på forskjellige lokasjonar, dei har kontor på Øyrane i Førde og Sandnes i Rogaland. 
Etter å ha arbeidd konfidensielt over lengre tid, gjekk Renasys offentleg ut med teknologien sin i løpet av 2023. 
Dei tilbyr reinsetjenester til kommunar og interkommunale selskap innan avlaup og maritim sektor.

Samarbeidet med Sunnfjord kommune\citep{SunnfjordKommune} er retta mot ``Mission Zero'' som er eit ambisiøst mål om 
null utslepp, null avfall, null energi og ein generell forbetring av avlaupssektoren i Noreg. \newline
Sunnfjord kommune er ansvarleg for vann, veg og avlaup i sitt område og har engasjert 
Renasys for å utforske forbetringar ved reinseanlegget på Sande.
\newline
\newline


\begin{figure}[htbp]
    \centering
    \begin{subfigure}[b]{0.3\textwidth}
        \centering
        \includegraphics[width=1\textwidth]{Bilder/renasys.png}
    \end{subfigure}
    \hfill
    \begin{subfigure}[b]{0.3\textwidth}
        \centering
        \includegraphics[width=0.7\textwidth]{Bilder/SK.png}
    \end{subfigure}
    \caption{Logo oppdragsgivar}\label{fig:Oppdragsgivar}
\end{figure}

\section{Om Vedlegga}
Til denne rapporten er det utarbeidd eit eige dokument som inneheld alle relevante vedlegg. 
Desse vedlegga omfattar ytterlegare tekniske detaljar knytte til oppgåva vår.\newline
Vedlegga er organiserte for å gje lesaren ein djupare innsikt i arbeidet vi har utført.