\chapter{Forord}
\thispagestyle{romanpages}

Denne bacheloroppgåva er skreven som ein del av utdanninga vår i Automatiseringsteknikk med robotikk ved Høgskulen på Vestlandet, campus Førde.
Vi starta arbeidet tidleg januar, og oppgåva vert avslutta med ein presentasjon i juni.

Oppgåva er skreven på vegne av \gls{Renasys}\citep{Renasys} og \gls{Sunnfjord Kommune}\citep{SunnfjordKommune} og den omhandlar 
avlaupsreinseanlegget på Sande i Sunnfjord. 
Ein spesiell takk til dykk som har gjort denne oppgåva mogleg og
bidratt med gode faglege diskusjonar samt omvisningar på reinseanlegget på Sande og i Førde.

Vi ynskjer å rette ein stor takk til rettleiarane våre Olav Sande, Bjarte Pollen og Joar Sande ved \gls{HVL} Førde, som har bidratt med gode råd gjennom prosjektet.
Takk også til \gls{HVL} og MidTechnolegies\citep{MIDT} for programvarelisensar brukt i oppgåva.

Familie og vener fortener ein spesiell takk for deira støtte gjennom denne perioden.

Vi håpar sluttresultatet vil bidra til ei betre forståing og kan løyse litt av utfordringane \gls{Sunnfjord Kommune} har hatt med reinseanlegget dei siste åra.
For oss har det vore ein lærerik prosess, der vi har fordjupa oss innan avlaupsreisning, samt at vi har utvikla ferdigheitane våres i \gls{latex}, 
\gls{PLS} programmering i \gls{Codesys}, teikning av blokk diagram i \gls{SCD} med meir. Vi er sikre på at dette vil vere gode erfaringar å ta med oss vidare.
