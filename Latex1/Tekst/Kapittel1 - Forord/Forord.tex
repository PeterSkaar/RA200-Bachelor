\chapter{Forord}
\thispagestyle{romanpages}

Denne bacheloroppgåva er utarbeida som ein del av vår utdanning i Automatiseringsteknikk med robotikk ved Høgskulen på Vestlandet, campus Førde.
Vi starta arbeidet tidleg januar, og oppgåva vert avslutta med ein presentasjon i juni.

Oppgåva er skreven på vegne av \gls{Renasys} \citep{Renasys} og \gls{Sunnfjord Kommune} \citep{SunnfjordKommune} og den omhandlar 
avlaupsreinseanlegget på Sande i Sunnfjord. 
Ein spesiell takk til dykk som har gjort denne oppgåva mogleg og
bidratt med gode faglege diskusjonar samt omvisningar på reinseanlegget på Sande og i Førde.

Vi ønskjer å rette ein stor takk til rettleiarane våre Olav Sande, Bjarte Pollen og Joar Sande ved \gls{HVL} Førde, som har bidratt med gode råd gjennom prosjektet.
Takk også til \gls{HVL} \citep{HVL} og MIDTechnolegies \citep{MIDT} for programvarelisensar nytta i oppgåva \citep{MIDTToolbox} \citep{Microsoft}.

Familie og vener fortener ein spesiell takk for deira støtte gjennom denne perioden.

Vi håpar sluttresultatet vil bidra til betre forståelse og kan løyse nokon av utfordringane \gls{Sunnfjord Kommune} har hatt med reinseanlegget dei siste åra.
For oss har det vore ein lærerik prosess der vi har fordjupa oss innan avlaupsreinsing samt at vi har utvikla ferdigheitane i \gls{latex} \citep{MikTeX} \citep{VisualStudio}, 
\gls{PLS} programmering i \gls{Codesys} \citep{Codesys} og teikning av blokkdiagram med \gls{SCD}. Vi er sikre på at dette vil vere gode erfaringar å ta med vidare.
 