\chapter{Forord}
\thispagestyle{romanpages}

Denne bacheloroppgåva er skreven som ein del av utdanninga vår i Automatiseringsteknikk og robotikk ved Høgskulen på Vestlandet, campus Førde.
Vi starta arbeidet tidleg Januar, og oppgåva vert avslutta med ein presentasjon i Juni.

Oppgåva skreven på vegne av \gls{Renasys}\citep{Renasys} og \gls{Sunnfjord Kommune}\citep{SunnfjordKommune} og den omhandlar 
avlaups reinseanlegget på Sande i Sunnfjord. skriv meir..

% Takke opplegg
 
Vi ønsker og rette ein stor takk til rettleiarane våre Olav Sande, Bjarte Pollen og Joar Sande ved HVL Førde, som har bidratt med gode råd gjennom prosjektet.

Ein spesiell takk går også til oppdragsgivarane våre, Renasys og Sunnfjord kommune som har gjort denne oppgåve mogleg. Dei har bidratt med gode faglege diskusjonar samt omvisningar på reinseanlegget på Sande og i Førde.
Takk også til HVL og MidTechnolegies for programvare lisensar.

Familie og vener fortener ein spesiell takk for deira støtte gjennom denne intense perioden.

Vi håpar sluttresultatet vil bidra til ei betre forståing og kan løyse litt av utfordringane Sunnfjord Kommune har hatt med reinseanlegget dei siste åra.
For oss har det våre ein lærerik prosess, der vi har dykka inn i ``SBR'' teknologien, samt at vi har utvikla ferdigheitane våres i \gls{latex}, 
pls programmering i \gls{Codesys}, teikning av blokk diagram i CFC med meir. Vi er sikre på at dette vil vere gode erfaringar og ta med oss vidare 
inn i arbeidslivet.

% Bør innehalde
% - Hvorfor arbeidet ble utført
% - Personlige motivasjoner eller inspirasjoner for prosjektet
% - Hvordan prosjektet utviklet seg, eventuelle utfordringer underveis
% - eventuelle spesielle forhold eller omstendigheter rundt forskningen

-- Flytte introduksjon av oss hit ? \newline
-- Flyttes til appendix? All kode for denne bacheloroppgåva er samla i ein felles github (SETT INN LINK)
