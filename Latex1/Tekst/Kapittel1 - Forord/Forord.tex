\chapter{Forord}
\thispagestyle{romanpages}

Denne bacheloroppgåva er skreven som ein del av utdanninga vår i Automatiseringsteknikk og robotikk ved Høgskulen på Vestlandet, campus Førde.
Vi starta arbeidet tidleg Januar, og oppgåva vert avslutta med ein presentasjon i Juni.

Oppgåva skreven på vegne av \gls{Renasys}\citep{Renasys} og \gls{Sunnfjord Kommune}\citep{SunnfjordKommune} og den omhandlar 
avlaups reinseanlegget på Sande i Sunnfjord. 
Ein spesiell takk til dykk som har gjort denne oppgåva mogleg og
bidratt med gode faglege diskusjonar samt omvisningar på reinseanlegget på Sande og i Førde.

Vi ønsker og rette ein stor takk til rettleiarane våre Olav Sande, Bjarte Pollen og Joar Sande ved HVL Førde, som har bidratt med gode råd gjennom prosjektet.
Takk også til HVL og MidTechnolegies for programvarelisensar brukt i oppgåva.

Familie og vener fortener ein spesiell takk for deira støtte gjennom denne perioden.

Vi håpar sluttresultatet vil bidra til ei betre forståing og kan løyse litt av utfordringane Sunnfjord Kommune har hatt med reinseanlegget dei siste åra.
For oss har det våre ein lærerik prosess, der vi har dykka inn i avlaupsreisning, samt at vi har utvikla ferdigheitane våres i \gls{latex}, 
pls programmering i \gls{Codesys}, teikning av blokk diagram i SCD med meir. Vi er sikre på at dette vil vere gode erfaringar å ta med oss vidare.
