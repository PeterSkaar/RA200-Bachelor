\newpage
\section{Val og Drøfting}

Når vi starta og definere forskjellige løysningsalternativ i forprosjektet var vi
ikkje heilt sikre på kva retning vi skulle velge. Løysningsalternative virka alle gode,
men med varierande arbeidsmengde. Ilag med rettleiar frå HVL 
diskuterte vi dei ulike løysningane og kom fram til at løysningsalternativet der arbeidsmengden best
tilsvarte ei bacheloroppgåve var løysningsalternativ: Styresystem A til Å.

Dette var hovudgrunnen til at vi valgte løysningsalternativ: Styresystem A til Å.
Ilag med dette er det også fleire aspekter som gjer denne løysningen til den beste for anlegget.

Ved å følgje eit løysningsalternativ som bygger på en arbeidsprosess for et nytt anlegg, 
kan vi sikre oss mot å ta med snarvegar og uoptimal kode. 
Dette gir oss også muligheten til å utforske og finne bedre løysningar, 
samtidig som vi oppgraderar koden med ein meir moderne oppbygging og heilheit.
Videre legger dette til rette for ein enklere og meir fullstendig dokumentasjonsjobb, 
som er essensiell for å opprettholde kvaliteten og bærekrafta til prosjektet over tid.

Vi blei også anbefalt av intern rettleiar å halde oppgåva teoretisk.
Det var spørsmål og reinseanlegget på Sande følgte gjelande sikkerheitskrav og normer
og om start av eit praktisk inngrep på anlegget ville utløyse krav om forbetringar.
Dette var noko vi ikkje ønska å ta stillig til å vår bacheloroppgåve, og det blei avklart
med både Renasys og Sunnfjord kommune at oppgåva ville bli teoretisk.
