\section{Val og drøfting}

Då vi starta med å definere dei ulike løysningsalternativa i forprosjektet, var vi
ikkje heilt sikre på kva retning vi skulle velje. Alle løysningsalternativa syntest å vere gode,
men med varierande arbeidsmengde. Vi kom fram til at løysningsalternativ C hadde ei arbeidsmengd
som passa best for ei bacheloroppgåve.

Dette var hovudgrunnen til at vi valte løysningsalternativ C.
I tillegg til dette er det fleire aspekt som gjer denne løysinga til den beste for anlegget.

Ved å følgje eit løysningsalternativ som bygger på ein arbeidsprosess for eit nytt anlegg, 
kan vi forhindre å overføre feil og manglar frå eksisterande program. 
Dette gir oss også moglegheita til å utforske og finne betre løysningar, 
samtidig som vi oppgraderer styresystemet med ein forbetra oppbygging og heilskap.
Vidare legg dette til rette for ein enklare og meir fullstendig dokumentasjonsjobb, 
som er essensiell for å oppretthalde kvaliteten og berekrafta til prosjektet over tid.

Med tanke på arbeidsmengd, så valde vi å halde oppgåva teoretisk.
Det var uvisst om reinseanlegget på Sande følgde gjeldande sikkerheitskrav og normer,
og om starten av eit praktisk inngrep på anlegget ville utløyse krav om forbetringar.
Dette var noko vi ikkje ønska å ta stilling til i vår bacheloroppgåve, og det blei avklart
med både Renasys og Sunnfjord kommune at oppgåva skulle haldast teoretisk.

