\newpage
\section{Løysningsforslag}

Fellesfaktor for løysningsalternativa til problemstillinga vil vere å fornye
styresystemet med ein ny styringseining (\gls{PLS}). 
Dette vil bidra til å løyse mange av utfordringane som anlegget har i dag.\newline
Det er fleire måtar å utføre dette på, men vi vel å fokusere på dei mest relevante.

\subsection{Løysningsalternativ A:~Bytte \gls{PLS}}
Den enklaste løysninga er å oppgradera det eksisterande styresystem ein til ein. 
Dette inneber å bytte ut den eksisterande \gls{PLS}-en med ein nyare \gls{PLS} frå same leverandør, som inneheld lik programvare.\newline
Dette vil redusere faren for kritisk komponentsvikt, men sidan \gls{PLS}-programmet er det same, 
vil det likevel ikkje bli enklare å gjere endringar i styresystemet.

\subsection{Løysningsalternativ B:~Bytte \gls{PLS}, konvertere program}
Eit steg vidare frå løysningsalternativ A, er å konvertere det eksisterande programmet over til eit meir
framtidsretta programmeringsspråk der kompetansen er lettare tilgjengeleg. Dette alternativet inneber å konvertere logikk for logikk,
med dei nødvendige tilpassingane som trengst for den nye språkets unike krav og funksjonar.

Utfordringa med dette alternativet er at ein risikerar å overføre eventuelle feil frå eksisterande
program over til det nye programmeringsspråket. Det kan også vere problematisk å gjere endringar
i det konverterte programmet, ettersom den overordna strukturen ikkje blir bevart eller riktig tolka.

\subsection{Løysningsalternativ C:~Bytte \gls{PLS}, nytt program}
Eit tredje og meir komplett løysningsalternativ bygger på dei to førre alternativa, 
men forheld seg til problematikken med endring av program.
Dette alternativet involverer ikkje valet av \gls{PLS} eller bruk av eksisterande program.

Ved å setje seg inn i anleggets teknologiske verkemåte vil vi kunne få eit bilete av korleis anlegget skal fungere teoretisk.
På denne måten vil ein kunne opprette ein ny funksjonsbeskrivelse for anlegget og bygge eit nytt program basert på denne beskrivelsen.
Programmeringsmessig vil dette vere som å starte på ny, men dette vil gi sikkerheit for at anlegget driftast optimalt og at det nye programmet følgjer
dagens standard. Dette tilsvara ein arbeidsprosess av eit nytt anlegg.

\newpage
\subsection{Løysningsalternativ D:~Nytt anlegg, ny teknologi}
I tillegg til ein full gjennomgang av styresystemet, 
kan ein undersøkje nye løysningar for å optimalisere, forbetre og utvide heile reinseanlegget.
I samråd med Renasys arbeider Sunnfjord kommune med å undersøkje moglegheitene og kostnadane for eit slikt alternativ.
Første steg i ein slik prosess er uansett eit løysningsalternativ som forholder seg til problema rundt styresystemet.


