\newpage
\section{Løysningsforslag}

Fells faktor for løysningsalternativa til reinseanlegget vil vere å fornye
styresystemet med ein ny styringseining (PLS). 
Dette vil bidra til å løyse mange av utfordringane som anlegget har i dag.
Det er fleire måtar å utføre dette på, men vi vel å fokusere på dei mest relevante.
\newline
\subsection{Løysningsalternativ A: Bytte PLS}
Den enklaste løysninga er å oppgradera det eksisterande styresystem ein til ein. 
Dette inneber å bytte ut den eksisterande PLS-en med ein nyare PLS frå same leverandør, som inneheld lik programvare.
Dette vil minske faren for kritisk komponentsvikt. Sidan PLS-programmet er det same, 
vil det likevel ikkje bli enklare å gjere endringar i det.

\subsection{Løysningsalternativ B: Bytte PLS, konvertere kode}
Eit steg vidare frå løysningsalternativ A, er også å konvertere det eksisterande programmet over til eit meir
framtidsretta programmeringsspråk der kompetansen er lettare tilgjengeleg. Dette alternativet inneber å konvertere logikk for logikk,
med dei nødvendige tilpassingane som trengst for den nye plattformas unike krav og funksjonar.
\newline \newline
Utfordringa med dette alternativet er at ein risikera å dra med seg eventuelle feil frå eksisterande
kode over til det nye programmeringsspråket. Det kan også vere problematisk å gjere endringar
i den konverterte koden ettersom den overordna strukturen i programmet
kanskje ikkje blir fullt ut bevart eller riktig tolka i det nye programmeringsspråket.

\subsection{Løysningsalternativ C: Bytte PLS, ny kode}
Eit tredje og meir komplett løysningsalternativ bygger på dei to førre alternativa, men forheld seg til problematikken med endring av program.
Dette alternativet involverer ikkje valet av PLS eller bruk av den gamle koden.
\newline \newline
Ved å sette seg inn i anleggets teknologiske verkemåte vil vi kunne få eit bilete av korleis anlegget skal fungere teoretisk.
På denne måten vil ein kunne opprette ein ny funksjonsbeskrivelse for anlegget og bygge eit nytt program basert på denne beskrivelsen.
Programmerings messig vil dette vere som å starte på nytt, men dette vil gi sikkerheit for at anlegget driftast optimalt og at det nye programmet følger
dagens standard. Dette tilsvara ein arbeidsprosess av eit nytt anlegg.

\newpage
\subsection{Løysningsalternativ D: Nytt anlegg, ny teknologi}
I tillegg til ein full gjennomgang av styresystemet, 
kan ein undersøke nye løysningar for å optimalisere, forbetre og utvide heile reinseanlegget.
I samråd med Renasys arbeider Sunnfjord kommune med å undersøke moglegheitene og kostnadane for eit slikt alternativ,
men første steg i ein slik prosess er uansett eit løysningsalternativ som forholder seg til problema rundt styresystemet.


