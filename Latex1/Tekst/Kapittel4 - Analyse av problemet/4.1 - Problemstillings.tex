\section{Problemstilling}
Reinseanlegget er teknisk utdatert og treng fornying. Styresystemet, som no er over tjue år gammalt,
består hovudsakeleg av eldre og utgåtte komponentar. Med eldre komponentar aukar risikoen for svikt, 
og det kan vere vanskeleg å finne passande reservedelar.

WaterCare AS \citep{WaterCare}, som opphavleg leverte styresystemet har i seinare tid blitt avvikla. 
Dette gjer at kompetansen innan styresystemet og moglegheita for å gjere endringar i det er utfordrande.
Grunna desse utfordringane har ikkje anlegget klart å halde tritt med den teknologiske utviklinga, 
og mindre problem har gradvis auka til større utfordringar.

Samstundes med desse faktorane er dokumentasjonen til reinseanlegget mangelfull, noko som gjer at enkle arbeidsoppgåver blir utfordrande og tidkrevjande.
I verste tilfelle kan styresystemet til anlegget svikte og med dei utfordringane nemnt ovanfor vil det være krevjande
å få anlegget tilbake i drift. \newline 
Dette utgjer ei kritisk utfordring innan offentleg infrastruktur og kan ikkje oversjåast.
\newline

\begin{figure}[htbp]
    \centering
    \begin{subfigure}[b]{0.5\textwidth}
        \centering
        \includegraphics[width=1\textwidth]{Bilder/BeijerSkjerm.JPG}
        \caption{Beijer \gls{HMI}}\label{fig:BeijerSkjerm}
    \end{subfigure}
    \hfill
    \begin{subfigure}[b]{0.3\textwidth}
        \centering
        \includegraphics[angle=-90,width=1\textwidth]{Bilder/Styreskap.JPG}
        \caption{Styreskap}\label{fig:Styreskap}
    \end{subfigure}
    \caption{Styresystem}\label{fig:Styresystem}
\end{figure}
