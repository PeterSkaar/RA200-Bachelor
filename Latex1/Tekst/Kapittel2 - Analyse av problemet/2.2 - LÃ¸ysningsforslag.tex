\newpage
\section{Løysningsforslag}
Fellsfaktor for løysningsalternativ til reinseanlegget vil vere å modernisere 
styresystemet med ein ny moderne styringseining (PLS). 
Dette vil være med å løyse mykje av utfordringane som anlegget har idag. 
Det er fleire måtar og utføre dette på men vi velger å ta med dei som er mest relevant.
\newline
\subsection{Løysningsalternativ: Bytte PLS}
Den enklaste løysninga er å oppgradere eksisterande styresystem ein til ein. 
Dette vil seie å skifte ut den eksisterande PLS'en med ein moderne PLS frå same levrandør som køyrer lik software.
Dette vil minke faren for kritisk komponentsvikt men sidan PLS-programmet er det same vil det ikkje gjere det enklare å foreta
software endringar.

\subsection{Løysningsalternativ: Bytte PLS, oversette kode}
Eit steg vidare frå løysningsalternativ bytte PLS, er å også oversette gammal software over til eit programmeringsspråk
som er meir egna og der kompetansen er større. Ved denne løysningen ville ein oversatte logikk for logikk, litt som å
oversette mellom engelsk og norsk.\newline
Problemet med dette alternativet er at ein risikerer å ta med seg dårleg kode over på det nye språket og
det vil være problematisk å gjere software endringar ettersom heilheita i programmet ikkje blir tolka.

\subsection{Løysningsalternativ: Styresystem A til Å}
Ein tredje meir komplett løysning bygger på dei to førre, men forholder seg rundt problematikken med endigar i software.
Dette løysningsalternativet forholder seg ikkje til PLS eller den gamle koden.\newline
Ved å sette seg inn i anleggets teknologisk verkemåte vil ein kunne få eit bilete av korleis anlegget skal fungere teoretisk.
På denne måten vil ein kunne oprette ein ny funksjonsbeskrivelse til anlegget og bygge ein software utifrå denne funksjonsbeskrivelsen.
Dette vil være som å starte på ny, men det vil gi sikkerheit på at anlegget driftast optimalt og at software følger
dagens standard. Dette tilsvarer ein arbeidsprosess av eit nytt anlegg.

\subsection{Løysningsalternativ: Nytt anlegg, moderne teknologi}
I tillegg til ein full gjennomgang av styresystemet, 
kan ein undersøke nye løysningar for å optimalisere, forbetre og utvide heile reinseanlegget.
Ilag med Renasys arbeider Sunnfjord kommune med å undersøke moglegheiten og kostnadane for eit slikt alternativ,
men første steg i ein slik prosess er uansett eit løysningsalternativ som forholder seg til problema rundt styresystemet.
