% Definer sidemargene
\newgeometry{top=2.5cm,bottom=2.5cm,right=3cm,left=5cm}

% Definer seksjonstitler
\titleformat{\section}
  {\normalfont\Large\bfseries}{\thesection}{1em}{}
\titleformat{\subsection}
  {\normalfont\large\bfseries}{\thesubsection}{1em}{}

% Definer header og footer
\fancyhf{}
\fancyhead[L]{\includegraphics[height=2cm]{mango.jpg}} % Bytt ut 'logo.png' med filnavnet til din logo
\fancyhead[R]{\thepage\~av~\pageref{LastPage}}
\fancyfoot[L]{SPEC}
\fancyfoot[C]{CONFIDENTIAL}
\fancyfoot[R]{Peter Søreide Skaar \\ Student engineer/Operator \\ p.skaar@gmail.com}
\renewcommand{\headrulewidth}{0pt} % Ingen linje i header
\renewcommand{\footrulewidth}{0pt} % Ingen linje i footer
\pagestyle{fancy}


% Tittelområde
\begin{center}
  \Large \textbf{Oppdatering PLS-program} \\
  \large RA100 Førde \\
  Enkel funksjonsbeskrivelse for sampling \\
\end{center}

% Systeminformasjon-seksjon
\section{systeminformasjon}
Volumetrisk prøvetaking for \textit{etc.}

% Systembeskrivelse-seksjon
\section{Systembeskrivelse}
For å kunne starte prøvetaking remote er \textit{etc.}

% Tabell
\begin{tabularx}{\textwidth}{|X|X|X|}
  \hline
  Start: & Pulser: & Sample: \\
  \hline
  \textit{etc.} & \textit{etc.} & \textit{etc.} \\
  \hline
\end{tabularx}

% Resten av dokumentet følger her...

