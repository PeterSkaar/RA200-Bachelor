\section{Generell verkemåte}

Eit avlaupsreinseanlegg er bygd opp av tre hovuddelar: Primær, sekunder og tertiærreinsing,
samt ein del for behandling av slam \citep{Regjeriga}.
Alle desse delane kan løysast på forskjellige måtar, men hovudoppgåvene er dei same
i alle reinseanlegg.

``Primærreinsing'' handlar om å skilje organisk og uorganisk materiale.
I eit avlaupsreinseanlegg tilsvarer dette å skilje avlaupsvatnet, 
som ein vil behandle frå, sand, Q-tips, våtserviettar
og anna uønska material som ein ikkje ønskjer vidare i prosessen.\newline
``Primærreinsing'' er eit viktig steg for å bevare pumper og anna prosessutstyr.

``Sekundærreinsing'' handlar om å fjerne mest mogleg suspanderte stoff og organisk materiale.
Det tilsvarer å skilje organisk materiale frå vatn.
``Sekundærreinsing'' er i kvart anlegg avhengig av kva ``reinseprinsipp'' som er brukt. Dette tilsvarer
kva teknolgisk metode som nyttast for å utføre dette steget.
``Sekundærreinsing'' refererast generellt som biologisk reinsing.

``Tertiærreinsing'' handlar om å fjerne resterande forureiningar i vatnet.
Dette steget varierer veldig frå anlegg til anlegg og eg er
avhengig av kva krav reinseanlegget har som krav på sitt utsleppsvatn.

Slamebehandling handler om å fjerne og behandle det oppbygde organiske materialet (slam)
som skillast ut i sekundær og tertiærreinsing 

\begin{figure}[htbp]
    \centering
    \includegraphics[width=1\textwidth]{Figurar/Generellverkemåte.png}
    \caption{Generell verkemåte for eit avlaupsreinseanlegg}\label{fig:GenerellVerkemåte}
\end{figure}

Då vi hadde fått oversikt over generell verkemåte til eit avlaupsreinseanlegg retta vi fokuset mot Sande
Dette valde vi å dele opp i tre hovuddelar. Kva reinseprisipp er brukt, korleis Sande reinseanlegg
fungerar i praksis og om anlegget har nokre særeigna preg.
