\section{Generel verkemåte}

Eit avlaupsreinseanlegg er bygd opp av 3 hovuddelar: Primær, sekunder og tertiærreinsing,
samt ein del for behandling av slam.
Alle desse delane kan løysast på forskjellige måtar, men hovudoppgåvene er dei same
i alle reinseanlegg.

``Primærreinsing'' handlar om å skille organisk og uorganisk materiale.
I eit avlaupsreinseanlegg tilsvarer dette å skille avlaupsvatnet, 
som ein vil behandle frå, sand, Q-tips, våtserviettar
og anna uønska material som ein ikkje ynskjer vidare i prosessen. \newline
``Primærreinsing'' er eit viktig steg for å beveare pumper og anna prosessutstyr.

``Sekunderreising'' handlar om å fjerne mest mulig suspanderte stoffer og organisk materiale.
Det tilsvarer å skilje det meste av organisk materialet frå vatnet.
``Sekunderreising'' er i kvart anlegg avhengig av kva `reinseprinsipp' som er brukt. Dette tilsvarer
kva teknolgisk metode som nyttast for å utføre dette steget.
``Sekunderreising'' refererast oftast til biologiskreinsing.

``Tertiærreising'' handlar om å fjerne resterande forureiningar i vatnet.
Dette steget varierer veldig frå anlegg til anlegg og eg er
avhengig av kva krav reiseanlegget har som krav på sitt utslippsvatn.

Slamebehandling handler om å fjerne og behandle det oppbygde organiske materialet (Slam)
som skillast ut i sekundær og tertiærreinsing 

\begin{figure}[htbp]
    \centering
    \includegraphics[width=1\textwidth]{Figurar/Generellverkemåte.png}
    \caption{Generell verkemåte for eit avlaupsreinseanlegg}\label{fig:GenerellVerkemåte}
\end{figure}

Når vi hadde satt oss inn i den generelle verkemåten til eit avlaupsreinseanlegg
valgte vi å fokusere mot Sande.
Dette valgte vi å gjere i tre hovuddelar. Kva reinseprisipp er brukt, korleis Sande reinseanlegg
fungerar i praksis og om anlegget har nokre særeigna preg.
