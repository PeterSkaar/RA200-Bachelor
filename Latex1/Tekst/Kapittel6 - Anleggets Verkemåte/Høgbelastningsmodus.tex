\newpage
\section{Høgbelastningsmodus}

Høgbelastningsmodus blir aktivert ved stor til renning til anlegget. 
Dersom til renningen er større ein anleggets kapasitet i normal drift vil sekvenstidene til reaktorane blir redusert for å auke kapasiteten. 
Alle tider på høgkapasitetsmodus kan endrast i operatørpanelet.
Her er eit eksempel på sekvenstider:

Det tilførte avlaupsvatnet vil i slike situasjonar være svært uttynna, med lave konsentrasjonar av organisk materiale.
Den nødvendige biologiske ned brytningstida (reaksjonstida) kan derfor reduserast. 
Det viktige i slike situasjonar er å behalde sedimenterings¬tida konstant, slik at ein forhindrar slamflukt.

\begin{table}[h]
    \centering
    \begin{tabular}{|l|c|c|}
    \hline
        \rowcolor{myblack} % Assuming 'myblack' is meant to be 'black'
        \textcolor{purewhite}{Sekvenser} & \textcolor{purewhite}{Normal modus} & \textcolor{purewhite}{Høybelastnings modus} \\ \hline
        \rowcolor{lightgray} 1. Innpumping & 45 min & 45 min \\ \hline
        \rowcolor{purewhite} 2. Reaksjon & 180 min & 90 min \\ \hline 
        \rowcolor{lightgray} 3. Sedimentering & 90 min & 90 min \\ \hline
        \rowcolor{purewhite} 4. Uttapping & 30 min & 30 min \\ \hline
        \rowcolor{lightgray} 5. Pause & 0 min -> uendeleg & 0 min -> uendeleg  \\ \hline
    \end{tabular}
    \caption{Normal og Høgbelastningsmodus tider}\label{table:Normal Og Høgbelastningsmodus}
\end{table}

%skriblerier fra vegard
Høgbelastningsprogramm belager seg på at det er utregna ein hydraulisk belastning basert på nivåendringer i mottaktstanken
Det er ikkje til å ungå at ved oppgradering av anlegget, at flow måling må inn for ein mer nøyaktig måling.

