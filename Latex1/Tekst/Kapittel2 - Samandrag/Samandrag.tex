\chapter{Samandrag}
\thispagestyle{romanpages}

Bacheloroppgåva, som vi har løyst saman med \gls{Renasys} og \gls{Sunnfjord Kommune}, handlar om utbetring av styresystemet på 
avlaupsreinseanlegget i Sande i Sunnfjord. 
Anlegget er teknisk utdatert noko som gjer at ei oppgradering av styresystemet må undersøkast.

Rapporten presenterer fire ulike løysningsalternativ
der ulik grad av dybde blir vurdert. \newline
Rapporten legg til grunn val av løysningsalternativ C som presenterer planlegging av eit nytt styresystem.
Vidare undersøkar rapporten generell verkemåte til eit avlaupsreinseanlegg og forklarer og dokumenterar reinseanlegget på Sande.

Oppgåva gir innblikk i dei relevante stega i planlegging av eit nytt styresystem der eigen dokumentasjon er nytta som grunnlag for vidare arbeid. 
Utføring av programmering, simulering og testing er løyst med eigne funskjonsblokker og metodar i programmeringsverktøyet \gls{Codesys},
der annerkjende \gls{IEC} standardar er undersøkt og tatt i bruk.
Vidare blir styresystemet og programmeringsarbeidet skildra og dokumentert.

Resultatet av bacheloroppgåva gir vår arbeidsgivar ny og forbetra dokumentasjon av anlegget 
og grunnlaget for eit nytt, fleksibelt og teknisk moderne styresystem. 
Rapporten undersøker også generelle oppgraderingar og ytterligare forbetringspotensial for reinseanlegget.