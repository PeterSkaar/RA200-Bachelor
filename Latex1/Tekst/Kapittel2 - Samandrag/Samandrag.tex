\chapter{Samandrag}
\thispagestyle{romanpages}

Bacheloroppgåva, som vi ynskjer å løyse i saman med \gls{Renasys} og \gls{Sunnfjord Kommune}, handlar om utbetring av styresystemet på 
avlaupsreinseanlegget i Sande i Sunnfjord. 
Anlegget er teknisk utdatert noko som gjer at oppgradering av styresystemet undersøkjast.

Rapporten legg til rette for fire ulike løysningsalternativ
der ulik grad av dybde blir presentert og vurdert. \newline
Rapporten legg til grunn val av løysningsalternativ C der alternativet bygger på planlegginga av eit nytt styresystem. 
Reinseanlegget må undersøkjast og 
verkemåten til eit generelt reinseanlegget og verkemåten til dette anlegget må beskrivast og dokumenterast i forkant.

Oppgåva gir innblikk i dei relevante stega i planlegging av eit nytt styresystem der eigen dokumentasjon er brukt som grunnlag for vidare arbeid. 
Utføring av programmering, simulering og testing er løyst med eigne funskjonsblokker og metodar i programmeringsverktøyet \gls{Codesys},
der annerkjente \gls{IEC} standarar er undersøkt og tatt i bruk.
Vidare blir styresystemet og programmeringsarbeidet beskreve og dokumentert.

Resultatet av bacheloroppgåva gir vår arbeidsgivarar eit betre dokumentert anlegg

Resultatet av bacheloroppgåva gir vår arbeidsgivar ny og forbetra dokumentasjon av anlegget 
og grunnlaget for eit nytt, fleksibelt og teknisk moderne program. 
Rapporten undersøkjer også generelle oppgraderingar og ytterligare forbetringspotensial 
for å forbetre reinseanlegget og programmet ytterligare.