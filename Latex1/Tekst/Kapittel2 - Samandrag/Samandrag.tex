\chapter{Samandrag}
\thispagestyle{romanpages}

Bacheloroppgåva, som vi ynskjer å løyse saman med Renasys og Sunnfjord kommune, handlar om utbetring av styresystemet på avlaupsreinseanlegget i Sande i Sunnfjord. 
Anlegget er teknisk utdatert noko som gjer at styresystemet på anlegget må oppgraderast.

Rapporten legger til rette for fire ulike løysningsalternativer av problemstillinga,
der ulike grader av dybder blir presentert og vurdert. \newline
Vidare grunnlegger rapporten val av løysningsalternativ C der alternativet bygger på planlegginga av eit nytt styresystem. 
Reinseanlegget må undersøkjast og 
verkemåten til eit generelt reinseanlegget og verkemåten til dette anlegget må beskrivast og dokumenterast i forkant.

Oppgåva gir innblikk i dei relevante stega i planlegging av eit nytt styresystem der eigen dokumentasjon av anlegget er brukt som underlag. 
Utføring av programmering, simulering og testing er løyst med eigne funskjonsblokker og metodar i programmeringsverktøyet Codesys,
der annerkjente IEC standarar er undersøkt og tatt i bruk.
Vidare blir styresystemet og programmeirngsarbeidet beskreve og dokumentert.

Resultatet av bacheloroppgåva gir vår arbeidsgivarar eit betre dokumentert anlegg, 
grunnlaget for ny teknisk programvare der simulering og programmering er påbegynt men ikkje ferdig. \newline
Forbetringspotensialet til anlegget og program er diskutert, der rapporten legger til rette for generelle oppgraderingar
som undersøker å forbetre reinseanlegget og programmet ytterligare.