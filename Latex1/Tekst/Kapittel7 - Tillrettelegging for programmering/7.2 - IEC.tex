\section{IEC}
\thispagestyle{fancy}
\label{sec:5.2}


\gls{IEC} \citep{IEC} er ein internasjonal, ikkje statleg orginasjon som utviklar og publiserer tekniske standardar innan elektrofag. 
Norge er representert i IEC ved Norsk Elektrotekniske Komité (NEK) \citep{IEC-SNL}. 
IEC har standarar som dekker programmering av PLS som går heilt tilbake til 1993\citep{Wiki-93}. 
Den nåverande standaren som omfamnar PLS er IEC 61131\citep{IEC-61131}. Dette er ein standard spesielt designet for programmerbare kontrollarar, og er delt opp i 10 delar, der del 3 tar for seg programmeringsspråk. 

Vårt program er i hovudsak tiltenkt og programmerast etter IEC 61131-3 og IEC \gls{PAS} 63131\citep{IEC-63131}. 
Der IEC PAS 63131 er ein standard utarbeida av IEC som gir oss grunnlag for å lage \gls{SCD} samt å bruke forhandsdefinerte funksjonstemplater for funksjonsblokker. 
IEC PAS 63131 er laga med formål at leverandørindustrien og oljeselskap skal ha eit felles rammeverk for bruk på norsk sokkel, og er utarbeida etter NORSOK I-005:2013.
Ved å bruke desse standardane så gir det oss eit robust og fleksibelt rammeverk for å programmere anlegget. 
Ved bruk av dei forhandsdefinerte funksjonsblokkane har vi moglegheit til å enkelt knytte i hop fleire delar av programmet våra, og ha fleksibilitet ved å enkelt kunne endre og legge til funksjonar i programmet. 
Dette er noko vi har heile vegen har fokusert på, då vår sluttkunde har ytra eit ønske om fleire tilleggsfunksjonar i programmet, utover det som er der i dag. 
\newpage
