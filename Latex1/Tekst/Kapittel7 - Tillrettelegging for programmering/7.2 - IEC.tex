\section{IEC}
\thispagestyle{fancy}
\label{sec:5.2}


\gls{IEC} \citep{IEC} er ein internasjonal, ikkje statleg organisasjon som utviklar og publiserer tekniske standardar innan elektrofag. 
Norge er representert i \gls{IEC} ved Norsk Elektrotekniske Komité (\gls{NEK}) \citep{IEC-SNL}. 
\gls{IEC} har ein standard som dekker programmering av \gls{PLS} som går heilt tilbake til 1993\citep{Wiki-93}. 
Den nåverande standaren som omfamnar PLS er IEC 61131\citep{IEC-61131}. Dette er ein standard spesielt laga for programmerbare kontrollarar, og er delt opp i ti delar, der del tre tar for seg programmeringsspråk. 

Vårt program er i hovudsak tiltenkt programmert etter \gls{IEC} 61131-3 og \gls{IEC} \gls{PAS} 63131\citep{IEC-63131}. 
Der \gls{IEC} \gls{PAS} 63131 er ein standard utarbeida av \gls{IEC} som gir oss grunnlag for å lage \gls{SCD} samt å bruke forhandsdefinerte funksjonstemplater for funksjonsblokker. 
\gls{IEC} \gls{PAS} 63131 er laga med formål at leverandørindustrien og oljeselskap skal ha eit felles rammeverk for bruk på norsk sokkel, og er utarbeida etter NORSOK I-005:2013.
Ved å bruke desse standardane så gir det oss eit robust og fleksibelt rammeverk for å programmere anlegget. 
Ved bruk av dei forhandsdefinerte funksjonsblokkane har vi moglegheit til å enkelt knytte saman fleire delar av programmet vårt. 
Dette gir oss fleksibilitet ved å enkelt kunne endre og legge til funksjonar i programmet. 
Dette er noko vi har fokusert på, då vår sluttkunde har ytra eit ynskje om moglegheit for tilleggsfunksjonar i programmet. 
\newpage

