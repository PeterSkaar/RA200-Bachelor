\section{IEC}
\thispagestyle{fancy}
\label{sec:5.2}


\gls{IEC} \citep{IEC} er ein internasjonal, ikkje statleg organisasjon som utviklar og publiserer tekniske standardar innan elektrofag. 
Noreg er representert i \gls{IEC} ved Norsk Elektrotekniske Komité (\gls{NEK}) \citep{IEC-SNL}. 
\gls{IEC} har ein standard som dekker programmering av \gls{PLS} som går heilt tilbake til 1993 \citep{Wiki-93}. 
Den nåverande standarden som omfamnar PLS er IEC 61131 \citep{IEC-61131}. Dette er ein standard spesielt laga for programmerbare kontrollarar, og er delt opp i ti delar, der del tre tar for seg programmeringsspråk. 

Vårt program er i hovudsak tiltenkt programmert etter \gls{IEC} 61131-3 og \gls{IEC} \gls{PAS} 63131 \citep{IEC-63131}. 
\Gls{IEC} \gls{PAS} 63131 er ein standard som gir oss grunnlag for å nytte forhandsdefinerte funksjonstemplat for funksjonsblokker. 
\gls{IEC} \gls{PAS} 63131 er laga med formål at leverandørindustrien og oljeselskap skal ha eit felles rammeverk for bruk på norsk sokkel, og er utarbeida etter NORSOK I-005:2013 \citep{NORSOK}.
Ved å bruke desse standardane gir det oss eit robust og fleksibelt rammeverk for å programmere anlegget. 
Ved bruk av forhandsdefinerte funksjonstemplat har vi moglegheit til å enkelt knytte saman fleire delar av programmet. 
Dette gir oss fleksibilitet ved å enkelt kunne endre og legge til funksjonar i programmet.  
\newpage

