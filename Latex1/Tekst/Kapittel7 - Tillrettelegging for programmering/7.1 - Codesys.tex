\section{Codesys}
\thispagestyle{fancy}
Codesys\citep{Codesys} laga av Codesys Group tilbyr ein open kildekode løysning for prosjeket, og har ingen lisenskostnadar for sluttkunden. 
I tillegg så kan prosjektfilane brukast på fleire typar PLS einingar. 
Dette gir våra sluttkunde fleksibilitet i korleis dei ønska å implementere våra løysningsforslag til deira anlegg.

Codesys nyttar programeringsspråkstandaren satt av IEC 61131 som blant anna Structured Text (ST), Sequential Function Chart (SFC) og Ladder Diagram (LD). 
I vårt program planlegger vi å skrive koden i strukturert tekst (ST) og deretter, 
kople blokkene samen av eit grafisk-basert språk for å gi koden god lesbarheit.

Codesys har nyleg fått støtte for integrering av Github i programvaren. 
Dette gjer det mykje enklare for oss å halde versjonskontroll
og enklare for fleire av gruppemedlemmene å kode saman på same prosjekt. 
Github har vi god erfaring med og har nyttast på fleire andre prosjekter. 

Vidare så har Codesys støtte for bibliotekar igjennom CODESYS Store. 
Med å bruke kjente bibliotek så får vi tilgang ei samling av gjenbrukbar kodeblokker, funksjoner og komponentar som vi kan nytte.
Dei biblioteka ynskjer å bruke er som følger:

\begin{itemize}
    \item CODESYS Building Automation \citep{BuildingAutomation}
    \item SysTime \citep{DateAndTime}
    \item Util \citep{DateAndTime}
\end{itemize}


\newpage


%% CODESYS is a software platform for industrial automation technology. 
%% The core of the platform is the IEC-61131-3 programming tool "CODESYS Development System".