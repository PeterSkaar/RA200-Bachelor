\section{Codesys}
\thispagestyle{fancy}
Codesys \citep{Codesys} laga av Codesys Group tilbyr ein open kildekode løysning for prosjeket, og har ingen lisenskostnadar for sluttkunden \citep{CodesysLisens}. 
I tillegg så kan prosjektfilane brukast på fleire typar PLS einingar \citep{CodesysPLS}. 
Dette gir våra sluttkunde fleksibilitet i korleis dei ønska å implementere våra løysningsforslag til deira anlegg.

Codesys nyttar programeringsspråkstandaren satt av \gls{IEC} 61131-3 som blant anna \gls{ST}, ``Sequential Function Chart'' (\gls{SFC}) og ``Ladder Diagram'' (\gls{LD}). 

Codesys har nyleg fått støtte for integrering av \gls{github} i programvaren \citep{CodesysGIT}. 
Dette gjer det mykje enklare for oss å halde versjonskontroll
og enklare for gruppemedlemma å programmere saman. 
\gls{github} har vi god erfaring med og har nyttast på fleire andre prosjekter. 

Vidare så har Codesys støtte for bibliotekar igjennom CODESYS Store \citep{CodesysStore}. 
Med å bruke kjende bibliotek så får vi tilgang til ei samling av gjenbrukbare kodeblokker, funksjonar og komponentar som vi kan nytte.
Dei biblioteka vi ynskjer å bruke er som følger:

\begin{itemize}
    \item CODESYS Building Automation \citep{BuildingAutomation}
    \item SysTime \citep{DateAndTime}
    \item Util \citep{Util}
\end{itemize}


\newpage
