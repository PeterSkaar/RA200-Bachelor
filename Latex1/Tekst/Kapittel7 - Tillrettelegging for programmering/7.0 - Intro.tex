\chapter{Tillrettelegging for programmering}
\thispagestyle{fancy}
\label{sec:7} 

Tidlegare skildra under kapittel \ref{sec: 5}, ønskja vi å nytte ei løysning som ikkje hadde høge lisenskostnadar for vår arbeidsgivar. 
Sunnfjord kommune var interessert å ikkje låse seg til ein fast leverandør, men heller ha moglegheita til å ha valet mellom fleire 
leverandørar innan \gls{PLS}. Dette ilag med moglegheiten til å utvide vår eigen kompetanse 
gjorde at vi såg vekk ifrå Siemens \gls{TIA}-portal \citep{Siemens}, som vi hadde lært gjennom \gls{PLS} emnet ELE304.
Programmet ønskja vi å programmere i strukturert tekst (\gls{ST}), men undersøkte også 
typar av grafiske diagrambaserte språk for å betre vise samanhengar.

Det var også viktig at alle på gruppa kunne programmere samtidig og at ein felles programmeringsstandard skulle nyttast.
Det var viktig at parallelt arbeid ikkje skulle gi synkroniseringsproblem og at det fantest ei god løysning for dette.

