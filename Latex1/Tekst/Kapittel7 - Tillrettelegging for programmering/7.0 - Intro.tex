\chapter{Tillrettelegging for programmering}
\thispagestyle{fancy}
\label{sec:5} 

Som tidlegare beskrevet under krav og mål, ønskte vi å bruke ei løysning som ikkje hadde høge lisenskostnadar for vår sluttkunde. 
Sunnfjord kommune var interessert i å ikkje låse seg til ein fast leverandør, men heller ha moglegheita til å ha valet mellom fleire 
leverandørar innan PLS. Dette ilag med moglegheiten til å breie vår eigen kompetanse 
gjorde at vi såg vekk ifrå Siemens TIA-portal, som vi hadde lært igjennom PLS emnet, 
og begynte å sjå i andre endar.
Programmet ønska vi å skrive i Structured Text (ST) men undersøkte også 
typar av grafiske-diagram baserte språk for å lettare vise samanhengar i programmet.

Det var også viktig at alle på gruppa kunne programmere samtidig og at ein felles programmeringsstandard skulle nyttast.
Det var viktig at parallelt arbeid ikkje skulle by på synkroniserings problem og at det fantest ei god løysning for dette.

