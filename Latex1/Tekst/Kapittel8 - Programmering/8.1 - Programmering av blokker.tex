\section{IEC Funksjonsblokker} \label{IEC Seksjon}
\thispagestyle{fancy}

Vi gjore eit utval av funksjonsblokk maler basert på dei komponentane vi hadde identifiserte i reinseanlegget.
For å laga eit robust program valde vi og fokusere på: \gls{MA}, \gls{MB}, \gls{SBE} og \gls{SBV}.
Sjølv om funksjonsblokk malane kunne verke noko overkvalifiserte valde vi likevel å ha dei med, då nokon av tilleggsfunksjonane eventuelt kunne nyttast seinare
og at det gav oss ein klar retning å arbeide mot.

\gls{IEC} har sentrale begreper som vi ynskjer å utdjupe nærmare. Grunna mangel på gode norske begreper 
og for å forhindre forvirring har vi valgt å beskrive begrepa slik dei er definert i normen.

\begin{itemize}
    \item \textbf{Lock:} Action overruling any other signal while being true
    \item \textbf{Force:} Action overruling any other signal
    \item \textbf{Disable Transition:} Transistion high/low function not avaliable
    \item \textbf{Blocking:} Prevention of certain functions or operations 
    \item \textbf{Suppression:} Disable alarm annunciation as well as any associated automatic actions
\end{itemize}

Alle funksjonsblokk maler som er designet etter \gls{IEC} \gls{PAS} 63131 har alle moglegheit for handtering og varsling av feil 
Desse funksjonane er veldig viktig i videre arbeid mot feilhandtering og alarmliste.

\newpage

\subsection{Monitor Binary}
\gls{MB} funksjonsblokka (Vedlegg C.1) blir brukt til automatisk overvåking, alarmhandtering, framvising og låsing av binære prosess variablar.
Funksjonsblokka inkluderer alarm ``suppression'' og ``blocking'' funksjonalitet. Den har moglegheit for invertering av 
inngangssignal og moglegheit for tidsforseinking av utgangssignal via parameter.

Funksjonsblokka er brukt i programmet for å overvaka alle digitale nivåvipper og trykkbrytarar i prosessen.


\begin{figure}[htbp]
    \centering
    \begin{subfigure}[b]{0.45\textwidth}
        \centering
        \includegraphics[width=1\textwidth]{Bilder/MBBlokkIEC.png}
        \caption{IEC}\label{fig:Monitor Binary blokk IEC}
    \end{subfigure}
    \hfill
    \begin{subfigure}[b]{0.45\textwidth}
        \centering
        \includegraphics[width=0.7\textwidth]{Bilder/MBBlokkIProgrammet.png}
        \caption{Bruk i programmet}\label{fig:Monitor Binary blokk i programmet}
    \end{subfigure}
    \caption{Monitor Binary}\label{fig:Monitor Binary}
\end{figure}

\subsection{Monitor Analogue}
\gls{MA} funksjonsblokka (Vedlegg C.2) er brukt for skalering, visning, overvåking og alarmhandtering av \newline
analoge inngangsvariablar i ein prosess.
Funksjonsblokka inneheld ``suppression'' og ``blocking'' funksjonalitet.

Funksjonsblokka er brukt i programmet for å overvåke analoge trykknivågivarar samt å skalere og vise desse som ein fyllingsgrad i prosent.

\begin{figure}[htbp]
    \centering
    \begin{subfigure}[b]{0.45\textwidth}
        \centering
        \includegraphics[width=1\textwidth]{Bilder/MABlokkIEC.png}
        \caption{IEC}\label{fig:Monitor Analogue blokk IEC}
    \end{subfigure}
    \hfill
    \begin{subfigure}[b]{0.45\textwidth}
        \centering
        \includegraphics[width=0.7\textwidth]{Bilder/MABlokkIProgrammet.png}
        \caption{Bruk i programmet}\label{fig:Monitor Analogue blokk i programmet}
    \end{subfigure}
    \caption{Monitor Analogue}\label{fig:Monitor Analogue}
\end{figure}

\newpage

\subsection{Switch Binary Eletrical} 

\gls{SBE} funksjonsblokka (Vedlegg C.3) blir brukt for binærkontroll (av/på) av straumningselement for elektrisitet, varme eller væske. Den
kontrollerte komponenten kan være av typen motor, pumpe, varmeelement, vifte osv.
Funksjonsblokka er i dette programmet brukt til å styre motorar, pumper og blåserar.

Funksjonsblokka beskriver korleis ein styrer ein komponent.
Det er utgangen Y, som sender ein opne/stenge kommando (høg/lav) til komponenten. Blokka har fleire funksjonar, der den
tar utgangen og samanliknar med tilbakemelding \gls{XGH} som gjer korrekt \gls{BCL}/\gls{BCH} status. 

Funksjonsblokka inkluderar alarm ``suppression'', ``blocking'', ``safeguarding'' og ``transition'' funksjonalitet.

\begin{figure}[htbp]
    \centering
    \begin{subfigure}[b]{0.45\textwidth}
        \centering
        \includegraphics[width=1\textwidth]{Bilder/SBEBlokkIEC.png}
        \caption{IEC}\label{fig:Switch Binary Eletrical blokk IEC}
    \end{subfigure}
    \hfill
    \begin{subfigure}[b]{0.45\textwidth}
        \centering
        \includegraphics[width=0.5\textwidth]{Bilder/SBEBlokkIProgrammet.png}
        \caption{Bruk i programmet}\label{fig:Switch Binary Eletrical blokk i programmet}
    \end{subfigure}
    \caption{Switch Binary Eletrical}\label{fig:Switch Binary Eletrical}
\end{figure}

\newpage

\subsection{Switch Binary Valve}

\gls{SBV} funksjonsblokka (Vedlegg C.4) skal brukast til binær av/på kontroll av eit straumningselement ved å endra straumen av medium (varme eller væske). 
Typisk komponentar som styrast er bl.a.\ ventilar og spjeld.
Funksjonsblokka er i dette programmet brukt til å styre ventilar.

Funksjonsblokka styrer ventilen ved hjelp av dei binære inngangane \gls{XH} og \gls{XL}.\@
Desse inngangane styrer ein utgang Y, som sender opne/stenge-kommando (høg/låg) til ventilaktivatoren.
Alternativt kan dei pulsmodulerte utgangane \gls{YH} og \gls{YL} kan også nyttast.

Funksjonsblokka har også inngangar \gls{XGH} og \gls{XGL} som gjer tilbakemelding om ventilen er heilt open eller stengd
som bekrefter ventilen sin posisjon.

%\textbf{Kontrollfunksjonane i funksjonsblokka inkluderar:}
%\begin{itemize}
    %\item Generering av feilstatus (YF) om det oppstår ein intern eller ekstern feil.
    %\item Blokka set utgangen Y i samsvar med parameter når feil blir oppdaga.
    %\item Blokka set utgangen Y basert på tilbakemelding i ``outside mode'' når ingen eksterne inngangar blir brukte (XOH/XOL).
%\end{itemize}

Funksjonsblokka inkluderar alarm ``suppression'', ``blocking'', ``safeguarding'' og ``transition'' funksjonalitet.

\begin{figure}[htbp]
    \centering
    \begin{subfigure}[b]{0.45\textwidth}
        \centering
        \includegraphics[width=1\textwidth]{Bilder/SBVBlokkIEC.png}
        \caption{IEC}\label{fig:Switch Binary Value blokk IEC}
    \end{subfigure}
    \hfill
    \begin{subfigure}[b]{0.45\textwidth}
        \centering
        \includegraphics[width=0.5\textwidth]{Bilder/SBVBlokkIProgrammet.png}
        \caption{Bruk i programmet}\label{fig:Switch Binary Value blokk i programmet}
    \end{subfigure}
    \caption{Switch Binary Value}\label{fig:Switch Binary Value}
\end{figure}

\newpage



\newpage

%\begin{figure}[htbp]
%    \centering
%    \begin{subfigure}[b]{0.45\textwidth}
%        \centering
%        \includegraphics[width=1\textwidth]{Bilder/4_20mA_Scaling.png}
%        \caption{Skalering av mA mot prosent}\label{fig:Skalering av mA mot prosent}
%    \end{subfigure}
%    \hfill
%    \begin{subfigure}[b]{0.45\textwidth}
%        \centering
%        \includegraphics[width=0.95\textwidth]{Bilder/27327_prosent_Scaling.png}
%        \caption{Skalering av prosent til verdi}\label{fig:Skalering av prosent til verdi}
%    \end{subfigure}
%    \caption{Dei forskjellige skaleringane av inngangssignal}\label{fig:Skalering av prosent til verdi}
%\end{figure}


