\section{Utfordringer}
\thispagestyle{fancy}

\subsection{Blokker}
Vi opplevde noko utfordringar rundt det å ha fleire blokker som styrer samme komponentar og det å  
skrivar til ein felles global variabel.
Denne variabelen blir då satt sann og usann i frå fleire plassar i programmet, noko som gjer at 
variablenen sin tilstand vil vere tilfeldig basert på korleis Codesys sin kompilator leser koden.

Dette kan vi løyse ved å bruke ein egen global variabel for kvar blokk, 
og så skrive til ein funksjonsblokk som styrar den endelege globale utgangsvariabelen.
Det er fleire plassar i programmet vi møter denne utfordringa, som blant anna med pumpestyring, 
da kvar reaktor skal kunne styre same pumpe. Same løysning vil gjelde i denne situasjonen, 
der vi må lage ei blokk som tar inngangar frå begge pumpene og setter utgangen til riktig tilstand.  

\subsection{Mangel på sensorikk}

Anlegget har begrensa mengde med sensorikk noko som har gjort til at vi er nøyd til å
beregne, estimere og programmere oss rundt mangel på sensorar.

Som eksempel skal anlegget ha moglegheit til å gå inn i ein
høgbelastningsmodus dersom tilstrøymninga på anlegget er høg, 
men anlegget har ingen form for strøymningsmålar. \newline
Vi har derfor måtte berekna eit teoretisk tilstrøymningsverdi basert på stigninga
av volumet i mottakstanken.

Sjølv om slike løysningar vil fungere, vil det ikkje være optimalt for anlegget.
Estimering av prosessverdiar vil redusere nøyaktigheita i styringa og
mykje unødvendig prosessorkraft vil bli brukt på beregningar som enkelt kunne bli
erstatta av ein sensor. Det har også gått med mykje ekstra tid på programmeringa 
av slike funksjonar som eksempelet over.
Det har også krevd betydelig ekstra tid å programmere slike funksjoner, 
som det overnevnte eksempelet viser


% Trenger dette stå her?
%Anlegget har ikkje nokon form for gjennomstrøymingsmålar 
%og har difor ein teoretisk utrekna strøymingsverdiar. 
%Verdiane er utrekna frå tankvolum, og det gjelder for mottakstank og reaktortank. 
%Dette kan vere noko uheldig da dette gir oss noko manglande nøyaktigheit i målingane våra, 
%som forplantar seg vidare til databehandlinga.
%Vi har tatt utgangspunkt i planteikningane for å få så nøyaktige mål som mogleg, 
%i tillegg til at vi har fått verifisert tank (mottakstank) 
%innvending med video for å konstatere at våra mål og utrekningar er korrekte.


