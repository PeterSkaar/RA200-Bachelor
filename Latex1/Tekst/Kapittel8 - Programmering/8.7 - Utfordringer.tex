\section{Utfordringar}
\thispagestyle{fancy}

\subsection{Utgangsblokker med signalkonflikt}
Nokre komponentar skulle nyttast i fleire tilstandar og vi opplevde 
utfordringar med å ha fleire utgangsblokker for ein komponent.
Orginalt var dette løyst med fleire kall av ei utgangsblokk i forskjellige \gls{CFC} vindauge.
Kalla av dei ulike blokkene skreiv forskjellige signal til komponenten, og
sendte høg signal blei kontinuerleg overskreve.

Vi møtte denne utfordringa fleire plassar, t.d. i pumpestyringa,
der kvar reaktor skulle kunne styre den same pumpa.
Dette løyste vi ved å nytte ein unik utgangsvariabel for kvar blokk, 
og deretter skrive til ei blokk som samla signala til komponenten og satt utgangen til riktig verdi.

\subsection{Mangel på sensorikk}

Anlegget har begrensa mengde sensorikk. 
Dette har ført til at vi må berekne, estimere og programmere rundt denne mangelen.

Som døme har anlegget ein funksjon for å aktivere
høgbelastningsmodus ved høg tilstrøyming, 
men anlegget har ingen form for strøymningsmålar.\newline
Grunna dette har vi vore nøydd til å kalkulere ein teoretisk tilstrøyming basert på endringa av volumet i mottakstanken.

Sjølv om slike løysningar kan fungere, er det ikkje optimalt for anleggets drift.
Estimering av prosessverdiar vil redusere nøyaktigheit i styresystemet og
unødvendig prosessorkraft vil bli nytta på berekningar som enkelt kunne vore erstatta av ein sensor.\newline
Det det også krevd ekstra tid å programmere desse funksjonane.


