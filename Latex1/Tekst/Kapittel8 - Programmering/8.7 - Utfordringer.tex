\section{Utfordringer}
\thispagestyle{fancy}

\subsection{Blokker}
Vi opplevde noko utfordringar rundt det å ha fleire blokker som styrer den same komponenten, samt det å  
skriva til ein felles global variabel.
Denne variabelen kan då bli skriven ``true'' eller ``false'' i frå fleire plassar i programmet, noko som gjer at 
variablenen sin tilstand vil vere tilfeldig basert på korleis kompilatoren til \gls{Codesys} les koden.

Vi møtte denne utfordringa fleire plassar i programmet, t.d. i pumpestyringa,
der kvar reaktor skulle kunne styre den same pumpa.
Dette kunne vi løyse ved å bruke ein unik global variabel for kvar blokk, 
og deretter skrive til ei funksjonsblokk som styrer den globale utgangsvariabelen som går til komponenten.\newline
For pumpestyringa laga vi ei blokk som tok inngangen frå begge pumpene og satt utgangen til riktig tilstand.  

\subsection{Mangel på sensorikk}

Anlegget har ei avgremsa mengd med sensorikk, 
noko som har ført til at vi må berekne, estimere og programmere rundt mangelen på sensorar.

Som døme har anlegget ein funksjon for å aktivere
høgbelastningsmodus ved høg tilstrøyming av veske til annlegget, 
men anlegget har ingen form for strøymningsmålar.\newline
Difor har vi vore nøydde til å kalkulere ein teoretisk tilstrøymingsverdi basert på auken av volumet i mottakstanken.

Sjølv om slike løysningar kan fungere, er det ikkje optimalt for anleggets drift.
Estimering av prosessverdiar vil redusere nøyaktigheita i styresystemet og fører til at
mykje unødvendig prosessorkraft vil bli brukt på berekningar som enkelt kunne vore erstatta av ein sensor.\newline
Vidare har det også krevd mykje ekstra tid å programmere desse funksjonane.



% Trenger dette stå her?
%Anlegget har ikkje nokon form for gjennomstrøymingsmålar 
%og har difor ein teoretisk utrekna strøymingsverdiar. 
%Verdiane er utrekna frå tankvolum, og det gjelder for mottakstank og reaktortank. 
%Dette kan vere noko uheldig da dette gir oss noko manglande nøyaktigheit i målingane våra, 
%som forplantar seg vidare til databehandlinga.
%Vi har tatt utgangspunkt i planteikningane for å få så nøyaktige mål som mogleg, 
%i tillegg til at vi har fått verifisert tank (mottakstank) 
%innvending med video for å konstatere at våra mål og utrekningar er korrekte.


