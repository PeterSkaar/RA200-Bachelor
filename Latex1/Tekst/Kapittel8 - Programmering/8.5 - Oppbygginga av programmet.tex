\newpage
\section{Oppbygging av programmet}
\thispagestyle{fancy}

\subsection{Programmeringsmetode}
For å sette i sammen alle funksjonsblokkene, som vi hadde skreve i \gls{ST}, valde vi å bruke \GLS{Codesys} Continuous Function Chart (\gls{CFC}).
\gls{CFC} er er ein grafisk programmeringsmetode som bruker symbol og koplingar for å gjere programmet  meir visuelt.

Alle sammensettningar av blokker valgte vi å gjere i \gls{CFC}. Ved å bruke ein grafisk metode sikra vi oss god lesbarheit og
visuell forståelse i programmet. 

Alle inngangar og utgangar er leslege og enkle og forstå. \gls{CFC} i lag med god dokumentasjon vil kunne gi personar utan programmeringsbakgrunn
god forståelse av korleis programmet er bygd opp, utan å måtte lese kodelinjer.
\gls{CFC} gir eit godt grunnlag for feilsøking og analyse, dette bygger derfor vidare på filosofien med eit enkelt og fleksibelt program.

Dersom antall koplingar og linjer gjorde programmet vanskeleg å lese var det også
mogleg å opprette ``source'' og ``links'' som oppretta ein trådlaus forbindelsane gjennom ein unik ID.

\begin{figure}[htbp]
    \centering
    \includegraphics[width=1\textwidth]{Bilder/ReaktorPRG.png}
    \caption{Eksempel \gls{CFC} - Styring reaktor 1}\label{fig:CFCReaktor}
\end{figure}

\newpage

\subsection{Hovuddel}

Programmet er delt opp i tre hovuddelar, ei tilstandsmaskin for kvar reaktor og ein del for samling av felles reaktor funksjonar.
Alle delane har eit kall i MainTask og blir utført kvar \gls{PLS} syklus. Tilstandsmaskina har det overordna ansvaret og passar på kva 
funskjonsblokk med tilstandslogikk som køyrer.

Felles funksjonar er ein samling av funksjonsblokker og utrekningar som er felles og er uavhengig av reaktorane med tilstandsmaskin.
Driftsovervåking er ein sentral del av felles funksjoner, der gangtid og mengde prosessert vatn er eksempel på funksjonar og utrekningar
som blir gjennomførte.

I nokre tilfeller, som ved rullering av sivbed, var vi avhengig av at både tilstandsmaskin for reaktor 1 og reaktor 2 hadde den same informasjonen.
Dette løyste vi ved å lage ei funksjonsblokk ``fbSivbedRotation'' i felles funksjonar som henter inn og behandlar antall slamuttak for å så rotere sivbed når ei git grense var nådd.
Denne informasjonen blir derretter sendt til kvar tilstandsmaskin og sørger for at begge reaktorane har same aktive sivbedet.


\begin{figure}[htbp]
    \centering
    \includegraphics[width=1\textwidth]{Figurar/Oppbygging_Program.png}
    \caption{Illustrasjon oppbygging av programmet}\label{fig:OppbyggingProgram}
\end{figure}


\newpage

\subsection{Styring tilstandslogikk}

Som nemnt tilegare er det tilstandslogikken som samarbeider opp mot \gls{IEC} blokkene. Dette samarbeidet valge vi også å gjere i eit
\gls{CFC} vindu som igjen gjorde kall og koplingar visuelle. Dette \gls{CFC} vinduet fikk navn etter kva sekvens i SBR-prosessen
den hadde ansvar for å styre. 

Oppbyggningen av desse sekvensstyringane er gjort med inngangsblokker (\gls{MA} og \gls{MB}) øvst og utgangsblokker (\gls{SBE} og \gls{SBV}) i botn.
I mellom desse kjem sjølve tilstandslogikk blokkene som inneholder stryingslogikken.

\begin{figure}[htbp]
    \centering
    \includegraphics[width=1\textwidth]{Bilder/Heile_innpump.png}
    \caption{Eksempel \gls{CFC} - styring innpumping}\label{fig:CFCInnpumping}
\end{figure}

I denne figuren er ekstra inngangar, parameter inngangar og ekstra utgangar fjerna for å bedre visualisere kopllinga og samarbeidet mellom
\gls{IEC} blokkene og funksjonsblokk for tilstandslogikk.

\newpage

