\section{Tilstandsmaskin}
\thispagestyle{fancy}

Då \gls{IEC}-blokkene var ferdige, byrja vi på tilstandsmaskina som skulle styre sjølve \gls{SBR}-prosessen. 
Vi hadde allereie danna oss eit bilete, men kunne no byrje å nytte kunnskapen frå anleggets verkemåte
til å grovt fylle inn dei hendelsane og aksjonane som foregikk mellom tilstandane. 
Tilstandsmaskina er bygd opp av dei fem reaktorsekvensane som eksisterar i eit \gls{SBR}-anlegg.

Dette er ein enkel modell av korleis tilstandsmaskina er programmert, men gir eit godt innblikk i \newline funksjonaliteten. \newline \newline \newline \newline \newline

\begin{figure}[htbp]
    \centering
    \includegraphics[width=1\textwidth]{Figurar/Simpel tilstandsmaskin.png}
    \caption{Enkel model av tilstandsmaskin}\label{fig:SimpelTilstandsmaskin}
\end{figure}


\newpage

I sjølve programmeringa av tilstandsmaskina blei den oppretta som ei eiga funksjonsblokk, noko som gav oss moglegheita å nytte blokka for begge reaktorane.
Tilstandsmaskina er laga med fem inngangar og seks utgangar, og baserer seg på ``switch/case'' logikk.

Tilstandsmaskina sender ut høg på den respektive utgangen som samsvarer med reaktortilstanden den er i. Dersom tilstandsmaskina får tilbake
høg på den respektive tilstandsinngangen avanserer \newline tilstandsmaskina.
Det er også mogleg å hente ut aktiv tilstand ved hjelp av ein heiltallsverdi (1-5). \newline \newline \newline \newline

\begin{figure}[htbp]
    \centering
    \includegraphics[width=0.6\textwidth]{Bilder/Tilstandsmaskin.png}
    \caption{Tilstandsmaskin implementert i programmet}\label{fig:TilstandsmaskinIProgram}
\end{figure}


