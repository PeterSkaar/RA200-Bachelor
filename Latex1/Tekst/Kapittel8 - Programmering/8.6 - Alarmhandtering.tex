\newpage
\section{Alarm og feilhandtering}
\thispagestyle{fancy}

% Skriver litt om introduksjon om alarm og feilhandtering
Alarm og feilhandtering er ein sentral del av eit velfungerande styresystem. Det er viktig
at anlegget effektivt handterer og varslar om uønska situasjonar slik at 
driftspersonell kan varslast og tiltak kan iverksettast.

% Skrive litt om codesys alarmhantering
For å varsle om alarm og feil i vårt program har vi nytta oss av \gls{Codesys} sine
innebygde alarmhandteringsfunksjonar. Desse funksjonane gir oss god mulighet for å 
kontrollere, gruppere og prioritere alarmar samt sende informative tekstar til driftspersonell.

% Skriver om alarmar både gamle og nye.
IEC blokkene som vi har laga gir oss moglegheiten til å detektere fleire forskjellige typar varslingar
som feil, alarmar og forvarsel.
Dei forkjellige blokkene har ein ulik mengde med feil som skal kunne detekterast og varslast.
Feil blir detektert og ein boolsk variabel YF settast sann, 
samt at ein integer verdi YFI indikerar kva type feil som har opptrådd.\newline
Alle desse moglege typane varslingar, som er vidare beskreve i Appenix xXx, 
har vi samla i programmet vårt ilag 
med dei aktive varslingane som er på Sande reinseanlegg idag.

Vi har valgt å dele opp alle varslingane i fire forskjellige grupper.

\begin{itemize}
    \item \textbf{Feil}          (Der systemet ikkje fungerar, t.d sensorfeil og blokkfeil)
    \item \textbf{Alarmar}       (Kritisk prosessparameter, t.d straumbrudd og veldig høge nivå)
    \item \textbf{Forvarsel}     (Prosessparameter på veg mot kritisk nivå)
    \item \textbf{Informasjon}   (Prosessinformasjon med nytteverdi)
\end{itemize}

Ved å dele opp varslingar i fire forskjellige grupper med ulikt prioriteringsnivå vil systemet
og driftspersonell enklare kunne forstå samanhengen og alvoret på varslinga. \newline
Korleis dei ulike alarmane er definert og prioritert må evaluerast med ilag med Sunnfjord kommune.


\newpage