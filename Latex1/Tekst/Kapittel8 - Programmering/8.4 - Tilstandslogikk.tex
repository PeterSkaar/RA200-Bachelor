\newpage
\section{Tilstandslogikk}
\thispagestyle{fancy}

Styring av logikk som skulle skje i kvar tilstand, valde vi å samle i ei funksjonsblokk som vart definert som tilstandslogikk, og fikk navn etter tilstanden
den skulle styre t.d. reaksjon.

Kvar tilstandslogikkblokk får inn ``external enable'' XE frå tilstandsmaskina som startar tilstandslogikken. Når sekvensen er ferdig sender
funksjonsblokka høg på utgang Y som returnerast til tilstandsmaskina som avanserer til neste tilstand og tilstandslogikkblokk.

Det er tilstandslogikken som har ansvar for å samarbeide med \gls{IEC}-blokkene som handterar inngangssignaler
start/stopp av elektrisk utstyr og kontrollerer feil og tilbakemeldingar. \newline \newline \newline

\begin{figure}[htbp]
    \centering
    \begin{subfigure}[b]{0.3\textwidth}
        \centering
        \includegraphics[width=1\textwidth]{Bilder/fbInnpumping.png}
        \caption{Innpumping}\label{fig:fbInnpumping}
    \end{subfigure}
    \hfill
    \begin{subfigure}[b]{0.3\textwidth}
        \centering
        \includegraphics[width=1\textwidth]{Bilder/fbReaksjon.png}
        \caption{Reaksjon}\label{fig:fbReaksjon}
    \end{subfigure}
    \caption{Tilstandslogikkblokker implementert i programmet}\label{fig:ReaksjonsFasen}
\end{figure}



