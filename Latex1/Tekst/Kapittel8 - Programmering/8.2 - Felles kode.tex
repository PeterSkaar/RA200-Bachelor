\section{Generelle funksjonsblokker}
\thispagestyle{fancy}

Undervegs i programmeringa av \gls{IEC}-blokkene såg me også nødvendigheita av nokre generelle funksjonsblokker
som kunne gjenbrukast fleire gonger i programmet. Dette er hensiktsmessig sidan ein då slepp å skrive lik 
funksjonalitet fleire gonger, men heller kan kalle ei funksjonsblokk som kan utføre den same jobben.

Sjå vedlegg B for meir dokumentasjon av funksjonsblokkene.

\subsection{fbTimer}
Denne funksjonsblokka kan brukast når ein treng ei tidsforsinking i programmet.
Her kan ein nytte tidsforsinka inn, tidsforsinka ut, eller ein kombinasjon av begge.

\subsection{fbAnalougeAlarm}
Denne funksjonsblokka blir brukt til å overvake, tidsforsinke, behandle grenseverdiar, 
gi alarmar og legge til hysteresar på ferdig skalerte analoge inngangsverdiar.

\subsection{fbDigitalAlarm}
Denne funksjonsblokka kan overvake, tidsforsinke og gi alarmer. Det er valbart om blokka skal aktiverast på høg eller låg
inngang basert på ein parameter.

\subsection{fbSwap}
Denne funksjonsblokka mottar ein inngangsverdi og byter sekvensielt mellom å bruke to utgangsverdiar. Det vil sei at blokka hugsar på kva utgang som vart brukt sist,
og vil bruke den andre utgangen ved neste kall. Blokka har også moglegheit for feilhandtering.

\subsection{fbCalculations}
Denne funksjonsblokka utfører nokre rekneoppgåver som ligg i bakgrunnen og køyrer kontinuerleg. 
Det er hovudsakleg utrekningar av volum i reaktorane, mottakstank og drenert volum i frå reaktortankane.

\subsection{fbTimeMeter}
Denne funksjonsblokka er ei blokk som tel tida så lenge den er kalla på, og lagrar verdien i forskjellige tidsformat for seinare bruk.
Den blir brukt i programmet til å telle gangtid for forskjellige eletriske komponentar.

\newpage

\subsection{fbHighLoad}
Denne funksjonsblokka blir brukt til å overvake den antatte innstraumningen i mottakstanken, basert på nivåendringar.
Blokka beregner gjennomsnittleg tilstraumning ved å måle tanknivået kvart minutt over totalt 30 minutt.
Denne beregninga blir samanlikna med eit parameter  for å sette anlegget i høgbelastningsmodus.

\subsection{fbSivbedRotation}
Anlegget har fire ventilar til sivebedet som det er viktig å oppretthalde rotasjon mellom. Anlegget tappar frå ein reaktor om gongen, 
og denne funksjonsblokka vel kva ventil som skal aktiverast.
Rotasjonen blir angitt av ein parameter som fastset kor mange syklusar kvar ventil skal nyttast. 
Det er også mogleg å ta eit sivbed ut av rotasjonen, slik at ventilen ikkje blir inkludert i syklusen.

\subsection{fbDataprocessing}
Denne funksjonsblokka blir brukt i samband med innsamlinga av driftsdata. 

\subsection{fbProcessedWater}
Denne funksjonsblokka er ei eiga blokk for å halde oversikt over driftsdata som gjeld vatn som har blitt behandla av reaktorane. 
Denne blokka gir informasjon om totalt behandla vatn i forskjellige tidsformat.

\newpage

