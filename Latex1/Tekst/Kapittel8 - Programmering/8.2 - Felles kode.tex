\section{Generelle funksjonsblokker}
\thispagestyle{fancy}

% Legge inn forklaring av "felleskoden", utrekningar', sivebedrotasjon', dataprossesing', Høgbelastningsprogram', ProvessedWater.
% Alt under her er er ikkje skrevet i stein.

Undervegs i programmeringa av \gls{IEC}-blokkene såg me også nødvendigheita av nokre generelle funksjonsblokker
som kunne gjenbrukast fleire gonger i programmet. Dette er hensiktsmessig sidan ein då slepp å skrive lik 
funksjonalitet fleire gonger, men heller kan kalle ei \gls{FB} som kan utføre den same jobben. \newline
Sjå appendix xXx for meir dokumentasjon.

\subsection{fbTimer}
Timer \gls{FB} kan brukast når ein treng ei tidsforsinking i programmet.
Her kan ein nytte tidsforsinka inn, tidsforsinka ut, eller ein kombinasjon av begge.

\subsection{fbAnalougeAlarm}
Analogue alarm \gls{FB} blir brukt til å overvake, tidsforsinke, behandle grenseverdiar, 
gi alarmar og legge til hysteresar på ferdig skalerte analoge inngangsverdiar.

\subsection{fbDigitalAlarm}
Digital alarm \gls{FB} kan overvake, tidsforsinke og gi alarmer. Det er valbart om blokka skal aktiverast på høg eller låg
inngang basert på ein parameter.

\subsection{fbSwap}
Swap \gls{FB} mottar ein inngangsverdi og byter sekvensielt mellom å bruke to utgangsverdiar. Det vil sei at blokka hugsar på kva utgang som vart brukt sist,
og vil bruke den andre utgangen ved neste kall. Blokka har også moglegheit for feilhandtering.

\subsection{fbCalculations}
Calculations \gls{FB} utfører nokre rekneoppgåver som ligg i bakgrunnen og køyrer kontinuerleg. 
Det er hovudsakleg utrekningar av volum i reaktorane, mottakstank og drenert volum i frå reaktortankane.

\subsection{fbTimeMeter}
Time meter \gls{FB} er ei blokk som tel tida så lenge den er kalla på, og lagrar verdien i forskjellige tidsformat for seinare bruk.
Den blir brukt i programmet til å telle gangtida for forskjellige eletriske komponentar.

\subsection{fbHighLoad}
High Load \gls{FB}-blokka blir brukt til å overvake den antatte innstraumningen i mottakstanken, basert på nivåendringane i tanken.
Blokka kalkulerer gjennomsnittleg tilstraumning ved å måle tanknivået kvart minutt over totalt 30 minutt.
Denne kalkuleringa blir samanlikna med eit parameter, PXR 'setpunkt', for å sette anlegget i høgbelastningsmodus.
BBlokka estimerer også den antatte tilstraumminga per time.

\subsection{fbSivbedRotation}
Anlegget har fire ventilar til sivebedet som det er viktig å oppretthalde rotasjon mellom. Anlegget tappar frå ein reaktor om gongen, 
og \gls{FB} Sivbedrotation vel kva ventil som skal aktiverast.
Rotasjonen blir angitt av ein parameter som fastset kor mange syklusar kvar ventil skal nyttast. 
Det er også mogleg å ta ein ventil ut av rotasjonen, slik at han ikkje blir inkludert i syklusen.

\subsection{fbDataprocessing}
fbDataprocessing blir brukt i samband med innsamlinga av driftsdata. 
Når nokon av objekta som er meinte å ha driftsdata er i drift eller aktive, blir desse sendt til fbDataprocessing. Derifrå kallar den opp fbTimeMeter, som lagrar og tel driftsdata for objektet.

\subsection{fbProcessedWater}
fbProcessedWater er ein eigen blokk for å halde oversikt over driftsdata som gjeld vatn som har blitt behandla av reaktorane. 
Denne blokka gir informasjon om totalt behandla vatn for det noverande året, den noverande månaden, 
den noverande veka, den noverande dagen, føregåande år, førre månad, førre veke og førre dag.



