\section{Generelle funksjonsblokker}
\thispagestyle{fancy}

Undervegs i programmering av \gls{IEC}-blokkene såg vi det naudsynt av nokre generelle funksjonsblokker
som kunne gjenbrukast fleire gonger. Dette er hensiktsmessig då ein slepp å programmere lik 
funksjonalitet fleire gonger.

Sjå vedlegg for meir dokumentasjon av funksjonsblokkene. (Vedlegg B)

\subsection{fbTimer}
Denne funksjonsblokka gir moglegheit for tidsforsinking.
Her kan ein nytte tidsforsinking på, tidsforsinking av, eller ein kombinasjon av begge.

\subsection{fbAnalougeAlarm}
Denne funksjonsblokka nyttast til å overvake, behandle grenseverdiar, 
gi alarmar og leggje til hysterese på ferdig skalerte analoge inngangsverdiar.
Funksjonsblokka nyttast fleire gonger i \gls{MA}-blokka.

\subsection{fbDigitalAlarm}
Denne funksjonsblokka nyttast til å overvake og gi alarmar på digitale inngangsverdiar. 
Det er valbart om blokka skal aktiverast på høg eller låg inngang basert på ein parameter.

\subsection{fbSwap}\label{sec:1}
Denne funksjonsblokka mottar ein inngangsverdi og rullerar mellom å nytte to utgangsverdiar. Blokka hugsar på kva utgang som vart nytta sist,
og vil nytte den andre utgangen ved neste kall. Blokka har også moglegheit for feilhandtering.

\subsection{fbCalculations}
Denne funksjonsblokka utfører nokre rekneoppgåver. 
Det er hovudsakleg utrekningar av volum i reaktorane, mottakstank og drenert volum i frå reaktortankane.

\subsection{fbTimeMeter}
Denne funksjonsblokka tel tida så lenge den er kalla på, og lagrar verdien i forskjellige tidsformat for driftsdata.
Den blir nytta til å telle gangtid for forskjellige elektriske komponentar.

\newpage

\subsection{fbHighLoad}
Denne funksjonsblokka blir nyttast til å overvake den antatte tilstrøyming på anlegget basert på nivåendringar i mottakstank.
Blokka bereknar gjennomsnittleg tilstrøyming ved å måle tanknivået kvart minutt over totalt 30 minutt.
Denne berekninga blir samanlikna med eit parameter for å sette anlegget i høgbelastningsmodus.

\subsection{fbSivbedRotation}\label{sec:2}
Denne funksjonsblokka nyttast for å styre og rotere mellom anleggets fire sivbedetventilar under uttapping av slam. 
Parameter angir antall uttappingar per sivbed/ventil.
Det er mogleg å ta eit sivbed ut av drift, slik at ventil ikkje blir inkludert i rotasjonen.

\subsection{fbDataprocessing}
Denne funksjonsblokka blir nytta i samband med innsamlinga av driftsdata. 

\subsection{fbProcessedWater}
Denne funksjonsblokka nyttast for oversikt over driftsdata for behandla vatn. 
Blokka gir informasjon i forskjellige tidsformat.

\newpage

