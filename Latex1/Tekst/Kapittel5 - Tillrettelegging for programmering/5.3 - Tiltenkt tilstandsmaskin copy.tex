\section{Tilstandsmaskin}
\thispagestyle{fancy}

Vi identifiserte tidleg i prosessen at vi hadde eit ønskje om å lage ein tilstandsmaskin som hadde den overordna styringa over kvar reaktor. 
Sidan ein SBR tank er prinsipielt oppbygd av forskjellige sekvensar, virka det logisk å lage til ein tilstandsmaskin som gjekk igjennom dei forskjellige tilstandane basert på logikk som vart plassert i dei forskjellige tilstandane. 
Det vart tidleg i prosjektet utarbeida mykje dokumentasjon om verkemåte til anlegget som gjorde at vi kunne enkelt programmere ein tilstandsmaskin som vart intregert mot dei forskjellige sekvensane i anlegget. 
Tilstandsmaskina tar inn signal i frå dei forskjellige tilstandane i anlegget og avansera til neste sekvens når logikken i tilstanden gir signal om at den er ferdig. 

\begin{figure}[htbp]
    \centering
    \includegraphics[width=1\textwidth]{Bilder/tilstandsmaskin_prinsipp.png}
    \caption{Tilstandsmaskin prinsipp}\label{fig:Tilstandsmaskin prinsipp}
\end{figure}

