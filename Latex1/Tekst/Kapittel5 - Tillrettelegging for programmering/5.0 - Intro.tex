\chapter{Tillrettelegging for programmering}
\thispagestyle{fancy}
\label{sec:5} 

Som tidlegare beskrevet under krav og mål, ønska vi å bruke ein løysning som ikkje har løypande lisenskostnadar for vår sluttkunde. 
Sunnfjord kommune var veldig interessert i å ikkje låse seg til ein fast leverandør, men heller ha moglegheita til å ha valet mellom fleire 
leverandørar innan levering av PLS.Dette gjorde at vi såg vekk ifrå Siemens TIA-portal som vi hadde lært
igjennom PLS emnet, og begynte å sjå i andre endar.
Programmet ønska vi å skrive i Structured Text (ST) men fekk også anbefalt typar av grafiske diagram baserte språk for å lettare
vise samanheng i programmet.

Det var også viktig at alle på gruppa kunne programmere samtidig og at ein felles programmeringsstandard skulle nyttast.
Det var viktig at parallelt arbeid ikkje skulle by på synkroniserings problem og at det var ei god løysning for dette.

