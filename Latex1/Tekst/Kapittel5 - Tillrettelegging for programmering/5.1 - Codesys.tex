\section{Codesys}
\thispagestyle{fancy}
Som tidlegare beskrive under krav og mål, ønska vi å bruke ein løysning som ikkje har løypande lisenskostnadar for våra sluttkunde. 
Kommunen var veldig interessert i å ikkje låse seg til ein fast leverandør, men heller ha moglegheita til å ha valet mellom fleire leverandørar innan levering av PLS, så valet falt på programeringsplattforma Codesys\citep{Codesys} laga av Codesys Group. 
Codesys tilbyr ein open kildekode løysning for prosjekta, og har ingen lisenskostnadar for sluttkunden. 
I tillegg så kan prosjektfilane brukast på fleire typar PLS einingar. 
Dette gir våra sluttkunde fleksibilitet i korleis dei ønska å implementere våra løysningsforslag til deira anlegg.

Codesys støttar programeringsspråkstandaren satt av IEC 61131 som blant anna Structured Text (ST), Sequential Function Chart (SFC) og Ladder Diagram (LD). I våra program er all logikk skrevet i strukturert tekst (ST), og bygd opp av ein blokk-basert programering med Continuous Function Chart (CFC). 
CFC er en grafisk programmeringsspråk som utvidar dei standardiserte språkene i standarden og gir koden ein god lesbarhet.

Codesys har nyleg fått støtte for integrering av Github i programmvaren, som gjer det mykje enklare for oss å halde versjonskontroll og enkelt for fleire av gruppemedlemmene å kode saman på same prosjekt. 
Denne funksjonaliteten har vi nytta oss av flittig, og har hjelpt oss med å kunne ha moglegheita til å jobbe individuelt med prosjektet. 
\newpage