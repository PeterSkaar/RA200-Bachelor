\newpage
\section{Praktisk Virkemåte}
\thispagestyle{fancy}

Sjølv om Sande reiseanlegg anvender SBR-teknologi så er det enkle spesefikke
punkter der dette reiseanlegget avviker ifrå normen. 
Sande reinseanlegg bruker eksempelvis ein spesiel form for slambehandling sett i norsk perspektiv.

Vanleg slambehandling i Noreg er oppsamling av slammet i tanker som handterast og tømmast av lokale etater.
På Sande blir ikkje slammet lagra, men jamnlig spreid ut over eit designert område. På dette området er
det planta siv som skal ta opp slammet og det resterande vatnet blir naturleg filtrert og drenert.
Desse områda kallast sivbed.

Sivbeda er konstruert med fleire dreneringslag som gjer at resterande vatn skiljast ut i designerte soner.
Her kan desse handterast vidare etter ønska behov. 
På Sande reiseanlegg er djupaste dreneringssona klassifisert som rein reject og blir sendt ut til resepient.
Den øvre dreneringssona er forsatt klassifisert som skitten og blir returnert til mottakstanken.

Grunna denne slambehandlingsmetoden er det på Sande reiseanlegg heller slamfjerning frå reaktor
i reaksjonsfasen. Dette blir gjort for å ha mindre konsentrert slam.

\begin{figure}[htbp]
    \centering
    \includegraphics[width=1\textwidth]{Figurar/Sivbed.png}
    \caption{Illustrasjon sivbed}\label{fig:HMI}
\end{figure}

