\newpage
\section{Mekanisk Ustyr}
\subsection{Tankar}

\begin{enumerate}
    \item \underline{Mottaktstank:} \newline
    Mottaktstank/utjamningstank har eit total volum på ca. 100 \(\text{m}^3\). Tanken er laga i betong og ligger som «kjellar» under anlegget.
    \item \underline{Reaktor:} \newline
    Reaktorane, 2x165 \(\text{m}^3\), er standard Brimer tankar produsert av Kvamsøy Plastindustri AS i glassfiberarmert polyester tilpassa vårt behov for tilslutning i botn og via flensar på tankvegg. 
    Tankane er dimensjonert for de laster vanlig drift tilsvara.  Anslutninger på tankane er tilpassa aktuelle røyrtypa, ventiler og medie.  
    Kvar tank har følgande inndeling av soner:
    \begin{itemize}
        \item \textbf{Bruksvolum} \newline
        Bruksvolumet er den aktive delen av tanken som fyllast ved kvar innpumping.
        \item \textbf{Slamsona} \newline
        Slamsona er den delen av tanken som er under utløpet, fråtrekket sikkerheitssonen.
        \item \textbf{Sikkerheitsona}
        Den tredje sonen er sonen mellom bruksvolumet og slamsonen. Den er til for å ta hand om varierande sedimenterings eigenskapar og overskotsslam.
    \end{itemize}
    \item \underline{Slamlager:} \newline
    ”Slamlageret” er et slammineraliseringsanlegg basert på siv bed og er et stort basseng plassert utanfor anlegget.    
    \item \underline{Kjemikalielagring:} \newline
    Kjemikalietanken er produsert i rotasjonsstøypt PEH frå Polimoon Cipax AS.
\end{enumerate}

\newpage
\subsection{Roterande Utstyr}
\begin{enumerate}
    \item \underline{Kloakkpumper} \newline
    På anlegget er det  montert fem pumper. Pumpene styres av trykkgivarar/flottørar som signalerer start og stopp. 
    Dei to matepumpene som pumper innløpet frå mottakstank til reaktorane er montert tørroppstilt i horisontal versjon på stativ i maskinrommet i kjellaren, med ventiler på kvar side for vedlikehald og service.
    I pumpehuset utanfor anlegget er det montert to ned dykka pumper på geidefeste for retur pumping av rejektvatn frå siv bed og for retur pumping av slam frå påfyllingsrørene.
    I maskinrommet er det montert en lett slukpumpe. 
    Det er nytta pumper frå ITT Flygt/xylem på anlegget.   
    \item \underline{Blåsemaskiner} \newline
    På dette anlegget er det nytta skrue/lobekompressor. Levert av NESSCO
    Blåsemaskinene er vald spesielt for dette anlegget med omsyn til kapasitet, energiøkonomi og vedlikehaldskostnader.
    \item \underline{Doseringspumpe} \newline
    For dosering av kjemikalium nyttast membranpumper. Kjemikaliar blir pumpa direkte inn i reaktorane.
    Kem levert av?
    Korleis endre dosering?
\end{enumerate}

\subsection{Ventilar}
På dette anlegget er det montert fleire ulike ventiltypar, tilpassa ønsket funksjon. Ventilar levert av Lohse

\underline{Membran} ventiler med automatisk drift er nytta som ventilar for utløp av reinsa vann. Ventilane er i PVC.
\newline \underline{Skyvespjelds} ventiler med automatisk drift er nytta for styring av innløp og slam. 
\newline \underline{Skyvespjelds} ventiler med manuell drift er nytta på alle prosess leidningar som serviceventilar. Ventilane er i syrefast stål. 
\newline \underline{Magnet} ventiler er hovudsakeleg brukt for å styre instrumentluft til automatiske ventiler.

\subsection{Røyr}
På dette reinseanlegget er det lagt vekt på å bruke rør i miljøvennlege material.  Det er derfor valt røyr i PP eller PEH som hovudregel.  Spesielle detaljer er i PVC.
Ved å utnytte tilgjengelege leverandørars produktsortiment og kompetanse er det utvikla eit røyrsystem som fyller de krav reinseanlegget stiller. 
Røyr og detaljer er samansett ved muffeskøyt, flens og krage, sveis eller lim.  Val av samansetnings metode er tilpassa krav til service og vedlikehald.