\newpage
\section{REAKTOR-SEKVENSAR}
Reaktorsekvensane er delt opp i fem sekvensar som er basert på SBR-teknologi beskrive i avsnitt xx.xx.
Sekvensane blir forklart i rekkjefølgje.

\subsection{Pause}
Ein reaktor vil være i pausesekvens så lenge det ikkje er bruk for reaktorens kapasitet. I pausesekvens vil reaktoren luftast periodisk gjennom tilhøyrande blåsar (PA01-BL01 / PA02-BL01) 
for å oppretthalde oksygeninnholdet i tanken og halde slammet aktivt, men samtidig ikkje bryte det heilt ned. Grad av periodisk lufting kan

Dersom følgande føresetnad er oppfylt går reaktoren over i innpumpingssekvens:
\begin{itemize}
    \item Nivågivar i mottaktstank (PP00-LT01) signaliserer innpumpingsnivå.
    \item Dersom nivågivar har feil vil flottør (PP00-LS02) fungere som backup.
    \item Nivågivar i respektiv reaktortank (PP01-LT01 / PP02-LT02) fungerer.
    \item Motorvern for pumpe ikkje slått ut.
\end{itemize}

\subsection{Innpumping}
Innpumpingsekvens byrjar ved å starte respektiv motor (PP01-PS01 / PP02-PS01) samt opne pneumatisk ventil (PP01-VP01 / PP02-VP01). 
Reaktor vil fyllast med avlaupsvatn så lenge nivågivar i mottakstank (PP00-LT01) eller flottør (PP00-LS02) signaliserer at det er nok vatn i mottakstanken. 
Startnivå for innpumping kan endrast frå operatørpanelet.

Dersom nivået i mottakstanken går under startnivå vil pumpe stoppe og ventil lukke. 
Dette medfører ikkje at innpumpingssekvensen er ferdig, men at den venter på meir vatn. 
Når nivågivar i mottaktstanken går over startnivå vil innpumping forsette. 

I Innpumpingssekvens vil reaktoren periodisk lufte reaktoren.
Systemet vil sørge for at dei to matepumpene vil ha tilnærma lik gangtid.
Dersom reaktor skulle overfyllast vil overlaup frå reaktor førast ned i mottaktstank.

Dersom følgande føresetnad er oppfylt går reaktoren over i reaksjonssekvens:
\begin{itemize}
    \item Nivågivar i reaktor (PP01-LT01 / PP02-LT02) signaliserer fullt bruksvolum eller makstid for innpumpingssekvens er nådd.
\end{itemize}

Lengda på sekvensen vil difor være bestemt av til-renninga opp mot makstid.
Når betingelse er oppfylt vil pumpe stoppe og pneumatisk ventil stenge.

\subsection{Reaksjon}
\underline{\textbf{Aerob}} \newline
Reaktor tilførast kontinuerleg oksygen frå respektiv blåser (PA01-BL01 / PA02-BL01). Lengde av aerob fase kan endrast frå operatørpanelet.

\underline{\textbf{Anoksisk}} \newline
Reaktor tilførast ikkje oksygen, respektiv blåser (PA01-BL01 / PA02-BL01) stopper. Lengde av anoksisk fase kan endrast frå operatørpanelet

\underline{\textbf{Simultanfelling}} \newline
Simultanfelling betyr kombinert biologisk og kjemisk reinsing. I slutten av reaksjonssekvensen tilsettast det kjemikaliar i reaktortanken. 
Doseringspumpe (CH00-PH01 / CH00-PH02) pumpar (kjemikalie) frå kjemikalietank CH00-BX01 og tilsett direkte til reaktortank.

Dosering av kjemikaliar er proporsjonalt med innpumpa råkloakk. Gangtida kontrollerast frå operatørpanelet, eller justerast direkte på doseringspumpa. 
Doseringmengda kan og skal justerast av driftsoperatør. Den skal justerast i forhold til målt fosfat-fosfor (orto-fosfat) på resipientprøven.

Dersom følgande føresetnad er oppfylt går reaktoren over i sedimenteringssekvens
\begin{itemize}
    \item Tid på reaksjonssekvens er ferdig.
\end{itemize}

\subsection{Sedimentering}
Sedimentering startar ved avslutta reaksjonsfase. I sedimenteringsfasen er eit roleg miljø nødvendig. Derfor skal den hydrauliske belastninga i tanken være lik null.
Dette medfører ingen innpumping, opne ventilar eller lufting av reaktor.

Dersom følgande føresetnad er oppfylt går reaktoren over i uttapping sekvens.
\begin{itemize}
    \item Tid på sedimenteringssekvens er ferdig.
\end{itemize}

\subsection{Uttapping}
Etter sedimenteringssekvensen vil slammet og SS være skilt ifrå vatnet. 
Vatnet på toppen av reaktoren kan no drenerast med sjølvfall mot resipient. 
Pneumatisk dreneringsventil (TW01-VP01) opnast og reinsa vatn drenerast ut.

Dersom følgande er oppfylt går reaktoren over i pausesekvens.
\begin{itemize}
    \item Dreneringstid for reaktor ferdig, eller nivågivar i reaktor (PP01-LT01 / PP02-LT02) signaliserer stoppnivå.
\end{itemize}