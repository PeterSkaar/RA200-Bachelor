TO DO LISTE




- Lage tilstandsdiagram over tisltandsmaskina vår (Visio mat)
- Oppdatere SCD, til siste skrik

Kap1
- Litt meir juice på renasys som bedrift og sunnfjord kommune
- Må komme tydleg fram! 
1. Gjekk detaljert inn i annleggets verkemåte
2. Progammerte anlegget frå verkemåte i (ST) og ikkje tidlegare program (Ladder)
3. 

Kap4
Første del : Programvare.
- Skrive inn forskjeller mellom Gammalt annlegg og nytt anlegget Programmerings messigt.
- Kva gjore me annaleis?
- Korleis er gammalt annlegg bygt opp, korleis er det nye annlegg bygd opp.
- t.d. Berekning av kubikk inn, Gjort på ein anna måte
- IEC Norm -> MA,MB,SBV,SBE blokker (Referere til heile dokumentet?)

Kap5 (Dokumentasjon)
- Interlock, t.d. Styringen av pumpene i innpuming, reaktorar kan ikkje gå i innpumping på likt



Skjelettet vi vil ha!

Romartall før kapittel?
- Framside
- Forord
- Samandrag
- Innhaldsliste
- Tabelliste
- Figurliste
- Ordliste
- Appendiksliste (Datasheet, Drawings, Video og Bilete osv A -> I)

kapittel
1. Innleeing
    1.1 Arbeidsgivar

2.  Analyse av problem
    2.1 Introduksjon til problem, med grunngjeving
    2.2 Forslag til løysning 
        2.2.1 Ein til Ein utskifting av PLC (Laste opp ladder logikk på anna PLC)
        2.2.2 Oversetting av kode frå Ladder til ST, bytte PLC
        2.2.3 Ny kode i ST ut i frå annleggets verkemåte og regjering av HW og SW oppdatering (evt ny sensorikk)
    Skrivast bra kort og konsist
    2.3 Val og Drøfting 
    2.4 Konklusjon - (Korleis me konkluderer ting, først teknologisk -> programmering)

3. Krav og mål
    3.1 Krav
    3.2 Hovudmål, Delmål
    3.3 Tilleggsmål -> Forbetring av anlegget
    - HMI

4. Anleggets Verkemåte (Sette seg inn i korleis anlegget fungerar)
    4.1 Teknologist Verkemåte SBR, korleis funkar dette (Sjå vekk frå sande) (Få inn nokk info, slik at ein kan kutte ut den mekansike delen) (Referansa viktig)
    - Forbehandling
    - utjamningstank
    - Sekvenser
    - osv
    4.2 Sande RA200 teoretisk verkemåte (Slik skal det funke) (Korleis er sande knytta opp mot teknologien)
    - Vatnets gang gjennom annlegget
    - Høgbelastningsmodus
    - Pumpehus
    - Huber
    - Gaula
    4.3 Sande RA200 praktisk verkemåte (Avvik frå teoretisk)(Slik fungerar det på sande)
    - Slammuttak i Reaksjon??
    - P&ID
    4.4 Dokumentasjon Fornyiing
    - FSD, 

5. Tillrettelegging for programering
    5.1 Codesys m/git, Programmerings språk (ST) - Hungarian notation
    5.2 IEC Standard (Generelt, templates osv, Arguemtering kvifor) (Vegen videre ved val av standariserte blokker)
    5.3 Tiltenkt tilstandsmaskin (Sekvenser -> Visio)
    5.4 SCD Generell forklaring -> Første utkast
    - Kvifor me tok dei vala me gjore.


6. Programmering
    6.1 Programmering av IEC Blokker (MB,MA,SBE,SBV) -> + fb blokker (fbTimer, fbAnalougeAlarm, fbCAC)
    - Til appendiks, Heile kode for t.d. fbTimer, fbCAC osv
   
    6.2 Tilstandsmaskin

    6.3 Oppbygningen av programmet (Peter Visio Layer)
    - Korleis me brukte IEC Blokker, fbTimer

    6.4 Tilstandslogikk
    - Bilete fbInnpumping
    - Felleslogikk

    Få forklart mest mogleg av programmet!
    Det som ikkje er forklart er heller ikkje gjort!

7. Dokumentasjon
    7.1 Blokkdokumentasjon IEC og eigne fb
    - Bruke mal for å dokumentere blokkene vi har laga, som appendiks
    7.2 Alarmliste
    7.3 Interlocker
    7.4 Objektliste (Liste over alle komponentar)
    7.5 Vedlikehalds dokumentasjon? 
    7.6 Programmerings verkemåte -> FSD
    ++?


8. Simulering / Verifisering
    8.1 Kontinuerleg simulering
    - Simuleringen som vart gjort undervegs i programmeringa
    8.2 Simuleringsblokker
    8.3 Full simulering av annlegget
    - Figur av simulering
    - Full scala test
    8.4 Vegen videre?? Treng me denne her?

9. Diskusjon (Oppsummering)
    9.1 Vegen vidare for Annlegget
    9.2 Vegen videre for Programmeringa
    9.3 Vegen videre for Dokumentasjon
    9.4 Vegen videre mot Tilleggsmål

10. Konklusjon 
    - Måloppnåing


Referanser

Appendikser
- Logg, Timelister
- Gannt
- Blokkdokumentasjon
- Github link -> Evt med pascal filer, slik at koden blir "lesleg" utan codesys
- Meir bilete?

Bilder og Video
- Video av simuleringa



TIL OLAV (SPYRJEMÅL)

- Dokumentering av blokker -> Litt i rapport, detalj i appendiks?
- Kor mykje kode skal me ha med i rapporten?
- 

