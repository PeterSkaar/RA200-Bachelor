\section{Analyse av problemet}
Hovudidé for løysningsforslag til reinseanlegget på Sande vil vere å modernisere 
styresystemet for reinseanlegget med ein moderne styringseining (PLS). 
Dette vil løyse mykje av utfordringane som anlegget har. Det er fleire måten å gjere dette på, 
men vi går igjennom dei tre som er mest relevant.

\section{Utforming av moglege løysningar}

\subsection{Løysnings alternativ 1 (ein til ein)}
Den enklaste løysninga er og oppgradera eksisterande styresystem ein til ein. 
Dette vil seie å skifta ut den eksisterande PLS med ein nyare PLS av same leverandør. 
Som vil vere lettare og vedlikehalde og gjere eventuelle framtidige modifikasjonar på. 
Ein vil og då sikre anlegget mot komponentsvikt sidan ein har tilgjengeleg reservedelar 
og tilgang til styringssystemprogrammet lett tilgjengeleg.

\subsection{Løysnings alternativ 2 (Styresystem A til Å)}
Ein anna løysning er i tillegg til å oppgradera eksisterande styresystem, så går ein eit steg videre og brukar PLS av ein ny produsent. 
Ein lagar då eit nytt PLS program i frå grunnen av, som er basert på anleggets verkemåte og ikkje basert på tidlegare program.
Ein må då starte med å kartlegge anleggets verkemåte ved å laga ein ny funksjonsbeskriving. 
Deretter må ein dekode programmet på det eksisterande styresystemet og bruke den nye funksjonsbeskrivinga til å dokumentere eventuelle avvik.


Deretter brukar ein funksjonsbeskrivinga til å programmere eit nytt styringsanlegg. 
Det vil bli programmert i programmerings verktøyet Codesys i strukturert tekst hermed «ST» eller «SCL». 
Slik at ein har moglegheit til å velja ulike PLS leverandørar som er meir tilpassa kunden sitt behov. 
Ein lagar då ny dokumentasjonspakke for dette styringssystemet.

\subsection{Løysnings alternativ 3 (Nytt anlegg, moderne teknologi)}
I tillegg til ein full gjennomgang av styresystemet, 
kan ein sjå på nye løysningar for å optimalisere og forbetre heile reinseanlegget.
Kommunen jobbar mot å bruke ny teknologi som kan koplast parallelt inn i prosessen til det eksisterande reinseanlegget.
Ein kan då forbetre og optimalisere den eksisterande prosessen opp mot utbygging av ny teknologi.