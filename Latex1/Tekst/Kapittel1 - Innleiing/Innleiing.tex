\chapter{Innleiing}
\thispagestyle{fancy}
Rapporten er skrevet for Sunnfjord Kommune via oppdragsgivar Renasys AS.
Oppgåva er fokusert rundt det noverande avløpsreinseanlegget på Sande «RA200». Reinseanlegget har hatt 
problem over lengre tid noko som har gjort at Sunnfjord kommune har sett på forskjellige moglegheiter for å 
forbetre anlegget, spesielt innan styresystemet.
Under arbeidet har bachelorgruppa henta informasjon og spesifikasjonar frå dei forskjellige aktørane for å 
danne eit bra bilete av arbeidet. Rapporten legger grunnlaget for bacheloroppgåva som skal skrivast om same 
tema.

\newpage
\section{Organisering av rapporten}
Rapporten er organisert etter standard HVL-mal.
Hensikta med kapittel ein er og gje lesaren ein betre forståelse for målet og problembeskrivinga. 
Vidare går vi igjennom krav og spesifikasjonar før vi legger fram løysingsalternativ og trekker ein konklusjon til løysning.

\section{Oppdragsgivar}

\subsection{Renasys AS}
Renasys AS er ein startup som arbeider med banebrytande teknologi innan mekanisk finpartikkelfiltrering av avløpsvatn. 
Renasys har gått offentleg med teknologien sin i løpet av 2023 og tilbyr no tenester til kommunar og interkommunale selskap. 
Renasys arbeider mot «Mission Zero» som innebærer null utslipp, null avfall og null energi.

\subsection{Sunnfjord kommune}
Etter kommunereformen i 2020 blei Sunnfjord kommune danna av tidlegare Gaular, Naustdal, Førde, og Jølster kommune. Sunnfjord kommune teknisk drift har ansvar for avfall, veg, vann og avløp i Sunnfjord kommune.