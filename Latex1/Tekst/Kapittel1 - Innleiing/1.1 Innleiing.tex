\section{Innleiing}
\thispagestyle{fancy}
Rapporten er skrevet for Sunnfjord Kommune via oppdragsgivar Renasys AS.
Oppgåva er fokusert rundt det noverande avløpsreinseanlegget på Sande «RA200». Reinseanlegget har hatt 
problem over lengre tid noko som har gjort at Sunnfjord kommune har sett på forskjellige moglegheiter for å 
forbetre anlegget, spesielt innan styresystemet.
Under arbeidet har bachelorgruppa henta informasjon og spesifikasjonar frå dei forskjellige aktørane for å 
danne eit bra bilete av arbeidet. Rapporten legger grunnlaget for bacheloroppgåva som skal skrivast om same 
tema.

\newpage
\section{Organisering av rapporten}
Rapporten er organisert etter standard HVL-mal.
Hensikta med kapittel ein er og gje lesaren ein betre forståelse for målet og problembeskrivinga. 
Vidare går vi igjennom krav og spesifikasjonar før vi legger fram løysingsalternativ og trekker ein konklusjon til løysning.

\section{Oppdragsgivar}

\subsection{Renasys AS og Sunnfjord kommune}
Renasys er ein startup som arbeider med banebrytande teknologi innan mekanisk finpartillekfiltering av avlaupsvatn.
Renasys har kontor på Øyrane i Førde og har gått offentleg med teknologien sin iløpet av 2023. Renasys tilbyr reinsetjenester til kommunar og interkommunale selskap innan avlaup og maritim sektorar.
\newline
Renasys og Sunnfjord kommune arbeider ilag mot Mission Zero som innebærer null utslepp, null avfall, null energi og ein generell modernisering av avlaupssektoren i Norge.
Sunnnfjord kommune er ansvarleg for vann, veg og avlaup i sitt område og har bede Renasys om å undersøke moglegheiter for forbetringar av reinseanlegget på Sande i Sunnfjord.