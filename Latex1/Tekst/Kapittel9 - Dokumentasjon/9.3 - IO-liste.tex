\section{IO-liste}
\thispagestyle{fancy}

Når det gjeld \gls{IO}-liste, har vi tatt utgangspunkt i den eksisterande IO frå styresystemet på Sande.\newline Sida vår løysning baserar seg på ei teoretisk løysing,
har vi nytta dei eksisterande sensorane i programmet vårt. Vi har ikkje konstruert løysinga mot ein eksisterande \gls{PLS}, sidan vi 
ikkje har ein fysisk \gls{PLS} å planleggje  mot. \gls{IO}-liste er lagt ved som (vedlegg J)

%Når det gjelder \gls{IO} liste, så har vi ikkje noko konkret å visa til, enn den gamle lista sidan våra løysningsforslag baserer seg på ein teoretisk løysning.
%Vi har difor tatt utgangspunkt i ein minimum I/O som allereie ligger til stades i frå det tidlegare anlegget (Vedlegg J) 
%(Vi har den gamle IO, men den nye er ikkje konstruet da vi ikkje har ein fysisk pls å planlegge mot.)

%\includepdf[pages=1, scale=1, pagecommand={\thispagestyle{empty}}]{Appendix/IOliste.pdf}
%\includepdf[pages=2, scale=1, pagecommand={\thispagestyle{empty}}]{Appendix/IOliste.pdf}



%I tillegg så har vi laget til moglegheit for fleire inngangar basert på ønsket tilleggsmål gitt av arbeidsgivar. 
%Dette er tilrettelagt for til dømes tilbakemelding av ventilar, flowmåler og temperaturgivera.
%Dette er funksjonalitet som er veldig enkle å legge til i ettertid, da programmet er bygget opp med dette i tankane.

