\section{Forrigling}
\thispagestyle{fancy}

For å sikre oss at utilsikta situasjonar skal oppskje i programmet har vi lagt inn nokre forriglingar
Desse forriglingane har vi koda i tilstandslogikken, 
men IEC blokkene gir oss også funksjonalitet som gjer det enkelt for oss å handheve forriglingar mellom komponentar.

Det ligger forrigling på styring av matepumpene til reaktorane. Det skal ikkje være mogleg at at begge pumpene går samtidig
og beskyttelse for dette er handtert i funksjonsblokka fbSwap. 
Det skal også ikkje forekomme at begge reaktorane kan være i innpumpingssekvens samtidig.
Dei to tilstandsmaskinene som styrer kvar sin reaktor er uavhengige av kvarandre så dette måtte vi løyse med ein
global variabel som blokkerer overgang frå tilstand pause til tilstand innpumping dersom ein reaktor allereie er i innpumpingssekvens,

I tillegg har vi overordna kontroll med at tilstandsmaskina gir signal til XE, 
slik at dei komponentane som ikkje er i bruk i sekvensen, ikkje er tilgjengeleg.



\newpage

%\begin{itemize}
%    \item Pumpemotorane er forigla mot å gå samstundes via fvSwap.
%    \item Reaktorane er forigla mot å vera i innpumpingssekvesen samstundes igjennom fbPumpLock 
%          og signal frå den andre tilstandsmaskina når den er i innpumping.
%\end{itemize}

%Sett inn tabell/bilde over forriglinger, kva forigla mot, korleis.

%Forrigling mellom innpumping og pause, slik at ein reaktor kun kan vere i innpumping samtidig.





