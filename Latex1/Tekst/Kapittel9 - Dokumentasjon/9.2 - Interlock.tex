\section{Forrigling}
\thispagestyle{fancy}

For å unngå utilsikta situasjonar i programmet har vi implementert nokre forriglingar.
Desse forriglingane har vi inkludert i tilstandslogikken.\newline
\gls{IEC} blokkene tilbyr også funksjonalitet som gjer det enkelt for oss å handheve forriglingar mellom komponentane.

Det ligg forrigling på styringa av matepumpene til reaktorane. Det skal ikkje være mogleg at begge pumpene går samtidig
og beskyttelse for dette er handtert i funksjonsblokka fbSwap(\ref{sec:1}).

Det skal heller ikkje inntreffe at begge reaktorane kan være i tilstand innpumping samstundes.
Sidan tilstandsmaskinene som styrer kvar sin reaktor er uavhengige av kvarandre, måtte vi løyse dette med ein
global variabel. Denne variabelen blokkerer overgangen frå tilstand pause til tilstand innpumping dersom ein reaktor allereie er i tilstand innpumping.

I tillegg har vi overordna kontroll med at tilstandsmaskina gir signal til \gls{XE}, 
slik at dei komponentane som ikkje er i bruk i tilstanden, ikkje er tilgjengeleg.

\newpage

%\begin{itemize}
%    \item Pumpemotorane er forigla mot å gå samstundes via fvSwap.
%    \item Reaktorane er forigla mot å vera i innpumpingssekvesen samstundes igjennom fbPumpLock 
%          og signal frå den andre tilstandsmaskina når den er i innpumping.
%\end{itemize}

%Sett inn tabell/bilde over forriglinger, kva forigla mot, korleis.

%Forrigling mellom innpumping og pause, slik at ein reaktor kun kan vere i innpumping samtidig.





