\section{Utfordringer}
\thispagestyle{fancy}

Problematikk rundt det å ha fleire blokker, som styrer samme komponetar da vi skriver til ein felles global variabel.
Denne variabelen blir då satt true og false i frå fleire plassar i programmet, noko som gjer at variablenen sinn tilstand vil vere tilfeldig, basert på korleis Codesys sin kompilator leser koden.

Dette kan vi løyse ved å bruke ein egen global variabel for hver blokk, og så skrive til ein funksjonsblokk som styrar den endelege globale variabelen.
Det er fleire plassar i programmet vi møter denna utfordringa, som blant anna med pumpestyring, da hver reaktor kan styre samme pumpe.
Samme løysning vil gjelde i denna situasjonen, der vi må lage ei blokk som tar inngangar frå begge pumpene og setter utgangen til riktig tilstand.

Dette er eit klassisk eksempel der vi har koda noko vi trur fungerar optimalt, men under testing så finner vi ut at det ikkje fungerar som vi har tenkt.
Løsninga setter nokre føringar for variabelhandtering videre i programmet, og vi står over eit val der valet våras gjør koden noko meir "innvikla", men vi opprettholder
funksjonaliteten i programmet slik vi opprinneleg hadde tenkt.

BARE RABLERIER FRA VEGARD
Volume utregninger har bydd på nokre utfordringer, da vi ikkje har noen form for flowmåling
Alt baserer seg på matte og nivåmålinger, og informasjon frå program.

Vi baserer våre volum målingar i mottaktstank på volume av kvadrat, v=b*l*h.

Høgbelastningsprogramm belager seg på at det er utregna ein hydraulisk belastning basert på nivåendringer i mottaktstanken
Det er ikkje til å ungå at ved oppgradering av anlegget, at flow måling må inn for ein mer nøyaktig måling.