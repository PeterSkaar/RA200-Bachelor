\section{IEC \gls{FB}}
\thispagestyle{fancy}

%Skriv OM:
%Programmering av IEC Blokker (MB,MA,SBE,SBV) 
%+ fb blokker (fbTimer, fbAnalougeAlarm, fbCAC)
%- Til appendiks, Heile kode for t.d. fbTimer, fbCAC osv

% Skrive litt om blokka sin funksjonalitet
% Korleis me brukte blokkene i programmet.

\glspl{FB} har fleir inngangnsfunkjonar som handterar ulike sikkerheit og driftsmodusar:
\begin{itemize}
    \item \textbf{Lock Safeguarding:} Beskyttar mot uautorisert driftsendring.
    \item \textbf{Force Safeguarding:} Hindrar tvangsstyring under visse tilhøve.
    \item \textbf{Force Disable Transition:} Deaktiverer overgang til tvangsstyring.
    \item \textbf{Force Blocking:} Blokkerer tvangsstyring.
    \item \textbf{Force Suppression:} Undertrykkjer tvangssignal.
\end{itemize}


\subsection{Monitor Analogue}
\gls{MA} funksjonsblokka er brukt for skalering, visning, overvåking og alarmhandtering av analoge inngangsvariablar i ein prosess.
Funksjonsblokka inneheld supression og blokking funksjonalitet.

funksjonsblokka er brukt i programmet for å overvåke analoge trykknivågivarar samt å skalere og vise desse som ein fyllingsgrad i prosent.

\begin{figure}[htbp]
    \centering
    \begin{subfigure}[b]{0.45\textwidth}
        \centering
        \includegraphics[width=1\textwidth]{Bilder/MABlokkIEC.png}
        \caption{IEC}\label{fig:Monitor Analogue blokk IEC}
    \end{subfigure}
    \hfill
    \begin{subfigure}[b]{0.45\textwidth}
        \centering
        \includegraphics[width=0.7\textwidth]{Bilder/MABlokkIProgrammet.png}
        \caption{Bruk i programmet}\label{fig:Monitor Analogue blokk i programmet}
    \end{subfigure}
    \caption{Monitor Analogue}\label{fig:Monitor Analogue}
\end{figure}
\newpage

\subsection{Monitor Binary}
\gls{MB} \gls{FB} blir brukt til automatisk overvåking, alarmhandtering, framvising og latching av binære prosess variablar.
\gls{FB} inkluderer alarm suppression og blokkerings funksjonalitet. Den har moglegheit for invertering av 
inngangssignal og moglegheit for tids forseinking av utgangssignal via parameter.

Funksjonsblokka er brukt i programmet for å overvaka alle digitale nivåfølerar i prosessen.


\begin{figure}[htbp]
    \centering
    \begin{subfigure}[b]{0.45\textwidth}
        \centering
        \includegraphics[width=1\textwidth]{Bilder/MBBlokkIEC.png}
        \caption{IEC}\label{fig:Monitor Binary blokk IEC}
    \end{subfigure}
    \hfill
    \begin{subfigure}[b]{0.45\textwidth}
        \centering
        \includegraphics[width=0.7\textwidth]{Bilder/MBBlokkIProgrammet.png}
        \caption{Bruk i programmet}\label{fig:Monitor Binary blokk i programmet}
    \end{subfigure}
    \caption{Monitor Binary}\label{fig:Monitor Binary}
\end{figure}

\newpage

\subsection{Switch Binary Value}

\gls{SBV} \gls{FB} skal brukast til binær av/på kontroll av eit flyt element ved å endra straumen av medium (varme eller væske). 
Typisk komponentar som styrast er bl.a. ventilar og spjeld.
\gls{FB} er i dette programmet brukt til og styre ventilar.

\gls{FB} styrer ventilen ved hjelp av dei binære inngangane XH og XL.
Desse inngangane styrer ein utgang Y, som sender opne/stenge-kommando(høg/låg) til ventilaktivatoren.
Alternativt kan dei pulsmodulerte utgangane YH og YL kan også nyttast.

\gls{FB} har også inngongar XGH og XGL som gjer tilbakemelding om ventiler er heilt open eller stengd,
som då bekrefter ventilen sin posisjon.

\textbf{Kontrollfunksjonane i \gls{FB} inkluderar:}
\begin{itemize}
    \item Generering av feilstatus (YF) om det oppstår ein intern eller ekstern feil.
    \item Den set utgangen Y i samsvar med parameter når feil blir oppdaga.
    \item Den set utgangen Y basert på tilbakemelding i ytremodus når ingen eksterne inngangar blir brukte (XOH/XOL).
\end{itemize}

\begin{figure}[htbp]
    \centering
    \begin{subfigure}[b]{0.45\textwidth}
        \centering
        \includegraphics[width=1\textwidth]{Bilder/SBVBlokkIEC.png}
        \caption{IEC}\label{fig:Switch Binary Value blokk IEC}
    \end{subfigure}
    \hfill
    \begin{subfigure}[b]{0.45\textwidth}
        \centering
        \includegraphics[width=0.5\textwidth]{Bilder/SBVBlokkIProgrammet.png}
        \caption{Bruk i programmet}\label{fig:Switch Binary Value blokk i programmet}
    \end{subfigure}
    \caption{Switch Binary Value}\label{fig:Switch Binary Value}
\end{figure}

\newpage

\subsection{Switch Binary Eletrical}

\gls{SBE} funksjonsblokka blir brukt for binærkontroll av straumningselement for elektrisitet, varme eller væske. Det
kontrollerte elementet er av typen motor, pumpe, varmeelement, vifte etc.

SBE blokka beskriver korleis ein kontrollarar ein enhet, for eksempel ein motor, pumpe, varmeelement, vifte etc.
Det er ein utgang Y, som gir ein opne/lukke (høg/lav) kommando til enheten. Blokka har fleire funksjonar, der den
tar output og samanliknar med tilbakemelding og gir korrekt BCL/BCH status. Den genererer også ein feil status på
YF om ein har ein ekstern feil inn.
Funksjonsblokka inkluderer alarm suppression, blocking, safeguarding og transition funksjonalitet.

\begin{figure}[htbp]
    \centering
    \begin{subfigure}[b]{0.45\textwidth}
        \centering
        \includegraphics[width=1\textwidth]{Bilder/SBEBlokkIEC.png}
        \caption{IEC}\label{fig:Switch Binary Eletrical blokk IEC}
    \end{subfigure}
    \hfill
    \begin{subfigure}[b]{0.45\textwidth}
        \centering
        \includegraphics[width=0.5\textwidth]{Bilder/SBEBlokkIProgrammet.png}
        \caption{Bruk i programmet}\label{fig:Switch Binary Eletrical blokk i programmet}
    \end{subfigure}
    \caption{Switch Binary Eletrical}\label{fig:Switch Binary Eletrical}
\end{figure}

\newpage

%\begin{figure}[htbp]
%    \centering
%    \begin{subfigure}[b]{0.45\textwidth}
%        \centering
%        \includegraphics[width=1\textwidth]{Bilder/4_20mA_Scaling.png}
%        \caption{Skalering av mA mot prosent}\label{fig:Skalering av mA mot prosent}
%    \end{subfigure}
%    \hfill
%    \begin{subfigure}[b]{0.45\textwidth}
%        \centering
%        \includegraphics[width=0.95\textwidth]{Bilder/27327_prosent_Scaling.png}
%        \caption{Skalering av prosent til verdi}\label{fig:Skalering av prosent til verdi}
%    \end{subfigure}
%    \caption{Dei forskjellige skaleringane av inngangssignal}\label{fig:Skalering av prosent til verdi}
%\end{figure}

